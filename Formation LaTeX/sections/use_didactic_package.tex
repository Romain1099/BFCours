\subsection{Introduction}


L'on dit toujours que le plus important lorsque l'on programme un logiciel ou un document, ce sont les \acc{contenants}. 

En effet, de bons contenants automatisent certaines fonctionnalités que l'auteur souhaite retrouver en tous temps. 
C'est précisément ce qui bloque de nombreux adeptes de \LaTeX. 

De nombreux packages sont disponibles et proposent des fonctionnalités plus ou moins équivalentes à celles développées dans \bfcours. 

Dans la suite de cette formation, nous utiliserons \bfcours\ par soucis d'homogénéité de la formation, mais il est tout à fait possible d'utiliser un autre package didactique à la place. 

Cela permet d'introduire la sous-section la plus importante de cette formation : \acc{la documentation}.

\subsection{Se documenter}

\begin{EXO}{Utiliser la documentation}{BF-3}
    Explorez l'univers de la communauté \LaTeX\ en découvrant deux packages didactiques comme alternatives à \bfcours. 

    \begin{tcbenumerate}[2]
        \tcbitem \tcbitempoint{1} Utiliser le package \encadrer[red]{pas-cours}.
        \begin{itemize}[label=$\bullet$]
            \item \acc{Accéder} à la documentation de \acc{pas-cours} sur \vocnoindexref{https://ctan.org/pkg/pas-cours}{CTAN - \frquote{pas-cours}}.
            \item \acc{Repérer les environnements didactiques}. 
            %\item Les \acc{utiliser} dans un exemple basique. 
        \end{itemize}
        \tcbitem \tcbitempoint{1} \acc{Utiliser} le package \encadrer[red]{profMaquette}. 
        \begin{itemize}[label=$\bullet$]
            \item Aller sur le site \vocnoindexref{https://ctan.org/}{https://ctan.org/}
            \item Chercher la documentation du package \frquote{profMaquette} de \acc{Christophe Poulain}. 
            %\item Construire un nouveau document latex utilisant ce package. 
        \end{itemize}
    \end{tcbenumerate}
    \exocorrection

    Cette correction n'est pas implémentée. 

    Il s'agit d'une alternative à l'utilisation de \bfcours.
\end{EXO}