
\subsection{Guidelines pour prompts structurés (2025)}

\begin{Definition}[Prompts structurés modernes]
    Les \voc{prompts structurés} représentent l'évolution des techniques de prompt engineering en 2025, combinant la précision du \acc{XML} pour la structure logique avec la lisibilité du \acc{Markdown} pour le contenu.
    
    Cette approche permet d'obtenir des réponses plus fiables et cohérentes des modèles de langage comme Claude d'Anthropic.
\end{Definition}

\begin{Methode}[Structure XML + Markdown]
    \begin{tcbraster}[raster columns=2,raster equal height=rows]
        \begin{tcolorbox}[title=Principe fondamental]
            \begin{itemize}[label=$\bullet$,itemsep=1.3em,leftmargin=*]
                \item \acc{XML} pour la \voc{structure logique}
                \item \acc{Markdown} pour le \voc{formatage du contenu}
                \item \voc{Balises sémantiques} descriptives
            \end{itemize}
        \end{tcolorbox}
        \begin{tcolorbox}[title=Exemple de base, colback=gray!5, colframe=gray!50]
            \ttfamily\small
            \xmltag{instructions}\\
            \hspace{1em}\#\# Tâche principale\\
            \hspace{1em}- Analyser le document\\
            \hspace{1em}- Extraire les points clés\\
            \hspace{1em}- **Formater** la sortie\\
            \xmlctag{instructions}
        \end{tcolorbox}
    \end{tcbraster}
\end{Methode}
\begin{Methode}[Ordre optimal des sections]
    L'ordre des sections dans un prompt structuré suit une progression logique du général au spécifique :
    
    \begin{tcbenumerate}[2]
        \tcbitem \xmltag{system_role} -- Qui est l'agent
        \tcbitem \xmltag{context} -- Situation et environnement  
        \tcbitem \xmltag{data} -- Données de référence
        \tcbitem \xmltag{rules} -- Règles et contraintes
        \tcbitem \xmltag{examples} -- Exemples few-shot
        \tcbitem \xmltag{instructions} -- Actions à effectuer
        \tcbitem \xmltag{output_format} -- Format de sortie
        \tcbitem \xmltag{critical_reminders} -- Points essentiels
    \end{tcbenumerate}
\end{Methode}

\begin{Exemple}[Prompt pour agent créateur de cartes \frquote{J'ai / Qui a}]
    Voici un exemple de prompt pour un agent spécialisé dans la création de cartes mathématiques :

    \begin{MultiColonnes}{2}
        \tcbitem \ttfamily\small
        \xmlbox{?xml version="1.0" encoding="UTF-8"?}\\[0.5em]
        \xmltag{agent_prompt}\\
        \hspace{1em}\xmltag{system_role}\\
        \hspace{2em}Expert LaTeX en création de jeux\\
        \hspace{2em}mathématiques pédagogiques\\
        \hspace{2em}Spécialiste tcolorbox et bfcours\\
        \hspace{1em}\xmlctag{system_role}\\[0.5em]
        \hspace{1em}\xmltag{context}\\
        \hspace{2em}\xmltag{level}Collège 4ème\xmlctag{level}\\
        \hspace{2em}\xmltag{topic}Calcul littéral\xmlctag{topic}\\
        \hspace{2em}\xmltag{cards_count}24\xmlctag{cards_count}\\
        \hspace{2em}\xmltag{objective}Jeu de révision\xmlctag{objective}\\
        \hspace{1em}\xmlctag{context}\\[0.5em]
        
        \tcbitem \ttfamily\small 
        \hspace{1em}\xmltag{instructions}\\
        \hspace{2em}\#\# Créer un jeu complet avec :\\
        \hspace{2em}1. **Cartes bouclées** (dernière $\rightarrow$ première)\\
        \hspace{2em}2. **Progression** de difficulté\\
        \hspace{2em}3. **Variété** des expressions\\[0.5em]
        \hspace{2em}\#\# Contraintes techniques :\\
        \hspace{2em}- Utiliser \textbackslash{}definecolor\{cardcolor\}\\
        \hspace{2em}- Format A4 avec 8 cartes/page\\
        \hspace{2em}- Police lisible (12pt minimum)\\
        \hspace{1em}\xmlctag{instructions}\\[0.5em]
        \hspace{1em}\xmltag{card_structure}\\
        \hspace{2em}\xmltag{front}J'ai : [expression]\xmlctag{front}\\
        \hspace{2em}\xmltag{back}Qui a : [question]\xmlctag{back}\\
        \hspace{1em}\xmlctag{card_structure}
    \end{MultiColonnes}
\end{Exemple}
\vspace{-0.5cm}\begin{Remarque}[Anti-patterns à éviter]
    \vspace{-0.5cm}\begin{Colonnes}[2]%[raster columns=2, raster equal height]
        \begin{tcolorbox}[title={\textcolor{red!75!white}{\textcolor{red}{\faTimes} À éviter}}, colback=red!5]
            \begin{itemize}[label=$\times$]
                \item Instructions négatives excessives (\frquote{Ne fais pas...})
                \item Balises XML invalides ou mal fermées
                \item Mélange contexte/instructions
                \item Exemples incohérents en format
                \item Instructions incohérentes
            \end{itemize}
        \end{tcolorbox}
        \begin{tcolorbox}[title={\textcolor{green!70!black}{$\checkmark$ Bonnes pratiques}}, colback=green!5]
            \begin{itemize}[label=$\checkmark$]
                \item Instructions positives claires
                \item Structure XML valide et cohérente
                \item Séparation nette des sections
                \item Exemples uniformes
                \item Concision, précision et cohérence
            \end{itemize}
        \end{tcolorbox}
    \end{Colonnes}
\end{Remarque}
\begin{Methode}[Techniques avancées]
    \begin{MultiColonnes}{2}
        \tcbitem \textbf{Chain of Thought} 

        Technique permettant de décomposer le raisonnement :
        
        \begin{tcolorbox}[colback=gray!5, colframe=gray!50]
            \ttfamily\small
            \xmltag{thinking_process}\\
            \hspace{1em}1. D'abord, analyser...\\
            \hspace{1em}2. Ensuite, vérifier...\\
            \hspace{1em}3. Finalement, produire...\\
            \xmlctag{thinking_process}
        \end{tcolorbox}
        \tcbitem \textbf{Scratchpad} 

        Pour les documents longs, utiliser un bloc-notes :

        \vfill
        \begin{tcolorbox}[colback=gray!5, colframe=gray!50]
            \ttfamily\small
            \xmltag{scratchpad}\\
            \hspace{1em}Points clés à retenir :\\
            \hspace{1em}- Information critique 1\\
            \hspace{1em}- Donnée importante 2\\
            \xmlctag{scratchpad}
        \end{tcolorbox}
    \end{MultiColonnes}
    \begin{tcolorbox}[title=Scratchpad permanent :,colback=gray!5, colframe=gray!50]
        \ttfamily\small
        \xmltag{scratchpad}\\
        \hspace{1em}Utilise le fichier \displayFilePath{mon\_repertoire\_IA/AI\_NOTES.xml} pour :\\
        \hspace{1em}- Noter ton plan d'action\\
        \hspace{1em}- Tenir à jour les étapes effectuées au fur et à mesure.\\
        \xmlctag{scratchpad}
    \end{tcolorbox}
    \encadrer[purple]{Cette dernière méthode est extrêmement efficace et permet aux agents de se succéder de façon fluide.}
\end{Methode}

\begin{Methode}[Prompt optimisé]
    Adapter la structure selon le type de tâche :
    
    \begin{MultiColonnes}{3}[colframe=\itemBaseColor,boxrule=0.4pt]
        \tcbitem[title=Pour du code,halign=left] 
        \begin{itemize}[label=$\bullet$,leftmargin=*]
            \item Structure de fichiers attendue
            \item Conventions de nommage
            \item Exemples d'entrée/sortie
        \end{itemize}
        \tcbitem[title=Pour des documents] 
        \begin{itemize}[label=$\bullet$]
            \item Templates de structure
            \item Styles d'écriture
            \item Ton et registre
        \end{itemize}
        \tcbitem[title=Pour des analyses] 
        \begin{itemize}[label=$\bullet$]
            \item Critères d'évaluation
            \item Métriques à calculer
            \item Format de rapport
        \end{itemize}
    \end{MultiColonnes}

    Pour un agent spécialisé en \LaTeX\ avec \acc{bfcours}, voici la structure optimale :
        \dirtree{%
            .1 \xmltag{latex_agent_prompt}.
            .2 \xmltag{system_role} \DTcomment{\color{green!75!black}Expert LaTeX/bfcours}.
            .2 \xmltag{context} \DTcomment{\color{green!75!black}Environnement pédagogique}.
            .2 \xmltag{bfcours_conventions}.
            .3 \xmltag{environments} \DTcomment{\color{green!75!black}Definition, Exemple, etc.}.
            .3 \xmltag{commands} \DTcomment{\color{green!75!black}\textbackslash voc, \textbackslash acc, etc.}.
            .3 \xmltag{layout} \DTcomment{\color{green!75!black}MultiColonnes, tcbraster}.
            .2 \xmltag{quality_requirements}.
            .3 \xmltag{imperative_rules} \DTcomment{\color{green!75!black}Règles strictes}.
            .3 \xmltag{design_principles} \DTcomment{\color{green!75!black}Principes de design}.
            .2 \xmltag{workflow_instructions}.\DTcomment{\color{green!75!black}Patterns de travail / chaîne de pensée}.
            .2 \xmltag{critical_reminders}.\DTcomment{\color{green!75!black}Points clés à respecter ( synthétique )}.
        }
\end{Methode}
\begin{bfbox}{Processus d'itération}
    \begin{tcbenumerate}[2]
        \tcbitem \textbf{Tester} avec des cas limites variés
        \tcbitem \textbf{Analyser} les échecs et réussites
        \tcbitem \textbf{Enrichir progressivement} la structure et les instructions
        \tcbitem \textbf{Affiner} le prompt
        \tcbitem \textbf{Conserver} une bibliothèque de prompts testés
        \tcbitem \textbf{Partager} les agents efficaces découverts
    \end{tcbenumerate}
    %\vspace{-0.6cm}
    \begin{center}
        \acc{Règle d'or :}
    \encadrer[\itemBaseColor]{Un prompt bien structuré}
            {\huge =}
            \encadrer[\itemBaseColor]{Moins de corrections} 
            {\huge +} 
            \encadrer[\itemBaseColor]{Meilleure productivité}\\

        \encadrer[cyan]{\frquote{La structure parfaite émerge de l'expérimentation continue.}}
    \end{center}

    
\end{bfbox}