\begin{Exemple}[Architecture d'agent avec MCP]
    \begin{tcolorbox}[
        title=Mode agentique MCP seulement,
        colback=blue!5!white,
        colframe=blue!50!black,
        width=\textwidth,
        boxrule=0.4pt,
        left=2pt,right=2pt,top=2pt,bottom=2pt
    ]
        \begin{tcbraster}[
            raster columns=5,
            raster equal height=rows,
            raster column skip=1cm,
            raster row skip=8pt
        ]
            % utilisateur
            \begin{tcolorbox}[
                blankest,
                halign=center,
                valign=center,
                height=3cm
            ]
                \tikzmark{mcp-user}utilisateur\tikzmark{mcp-user-end}
            \end{tcolorbox}
            % agent
            \begin{tcolorbox}[
                blankest,
                halign=center,
                valign=center
            ]
                \tikzmark{mcp-agent}agent\tikzmark{mcp-agent-end}
            \end{tcolorbox}            
            % Colonne outil 1
            \begin{tcolorbox}[
                blankest
            ]
                \begin{tcbraster}[
                    raster columns=1,
                    raster row skip=3pt
                ]
                    \begin{tcolorbox}[
                        blankest,
                        halign=center,
                        height=0.8cm
                    ]
                        \tikzmark{mcp-outil1-top}outil 1
                    \end{tcolorbox}
                    \begin{tcolorbox}[
                        blankest,
                        halign=center
                    ]
                        \parbox{3cm}{\centering \tikzmark{mcp-analyse1}analyse et\\poursuite de la\\tâche}\tikzmark{mcp-analyse1-end}
                    \end{tcolorbox}
                \end{tcbraster}
            \end{tcolorbox}
            % Colonne outil 2
            \begin{tcolorbox}[
                blankest
            ]
                \begin{tcbraster}[
                    raster columns=1,
                    raster row skip=3pt
                ]
                    \begin{tcolorbox}[
                        blankest,
                        halign=center,
                        height=0.8cm
                    ]
                        \tikzmark{mcp-outil2-top}outil 2
                    \end{tcolorbox}
                    \begin{tcolorbox}[
                        blankest,
                        halign=center
                    ]
                        \parbox{3cm}{\centering \tikzmark{mcp-analyse2}analyse et\\poursuite de la\\tâche}\tikzmark{mcp-analyse2-end}
                    \end{tcolorbox}
                \end{tcbraster}
            \end{tcolorbox}            
            % fin
            \begin{tcolorbox}[
                blankest,
                halign=center,
                valign=center
            ]
                \tikzmark{mcp-fin}Fin de la tâche
            \end{tcolorbox}
        \end{tcbraster}
        
        % Flèches
        \tikz[remember picture,overlay] {
            % utilisateur → agent
            \draw[->,thick] ([xshift=3pt]pic cs:mcp-user-end) -- ([xshift=-3pt]pic cs:mcp-agent);
            % agent → outil 1 (haut)
            \draw[->,thick] ([xshift=3pt]pic cs:mcp-agent-end) to[out=0,in=180] ([xshift=-3pt,yshift=3pt]pic cs:mcp-outil1-top);
            % outil 1 (analyse) → outil 2 (haut)
            \draw[->,thick] ([xshift=3pt]pic cs:mcp-analyse1-end) to[out=0,in=180] ([xshift=-3pt,yshift=3pt]pic cs:mcp-outil2-top);
            % outil 2 (analyse) → fin
            \draw[->,thick] ([xshift=3pt]pic cs:mcp-analyse2-end) -- ([xshift=-3pt]pic cs:mcp-fin);
        }
    \end{tcolorbox}
\end{Exemple}
\begin{Exemple}[Architecture d'agent avec A2A]
    \begin{tcolorbox}[
        title=Mode agentique A2A,
        colback=green!5!white,
        colframe=green!50!black,
        width=\textwidth,
        boxrule=0.4pt,
        left=2pt,right=2pt,top=2pt,bottom=2pt
    ]
        \begin{tcbraster}[
            raster columns=5,
            raster equal height=rows,
            raster column skip=1cm,
            raster row skip=8pt
        ]
            % utilisateur
            \begin{tcolorbox}[
                blankest,
                halign=center,
                valign=center,
                height=4cm
            ]
                \tikzmark{a2a-user}utilisateur\tikzmark{a2a-user-end}
            \end{tcolorbox}
            % agent
            \begin{tcolorbox}[
                blankest,
                halign=center,
                valign=center
            ]
                \tikzmark{a2a-agent}agent\tikzmark{a2a-agent-end}
            \end{tcolorbox}            
            % Colonne opérateurs
            \begin{tcolorbox}[
                blankest
            ]
                \begin{tcbraster}[
                    raster columns=1,
                    raster row skip=2pt
                ]
                    \begin{tcolorbox}[
                        blankest,
                        halign=center,
                        height=1cm
                    ]
                        \tikzmark{a2a-op1}opérateur IA 1\tikzmark{a2a-op1-end}
                    \end{tcolorbox}
                    \begin{tcolorbox}[
                        blankest,
                        halign=center,
                        height=1cm
                    ]
                        \tikzmark{a2a-op2}opérateur IA 2\tikzmark{a2a-op2-end}
                    \end{tcolorbox}
                    \begin{tcolorbox}[
                        blankest,
                        halign=center,
                        height=1cm
                    ]
                        \tikzmark{a2a-op3}opérateur IA 3\tikzmark{a2a-op3-end}
                    \end{tcolorbox}
                \end{tcbraster}
            \end{tcolorbox}
            % analyse
            \begin{tcolorbox}[
                blankest,
                halign=center,
                valign=center
            ]
                \tikzmark{a2a-analyse}\parbox{3cm}{\centering analyse et traitement des\\réponses\\multiples}\tikzmark{a2a-analyse-end}
            \end{tcolorbox}
            % fin
            \begin{tcolorbox}[
                blankest,
                halign=center,
                valign=center
            ]
                \tikzmark{a2a-fin}Fin de la tâche
            \end{tcolorbox}
        \end{tcbraster}        
        % Flèches
        \tikz[remember picture,overlay] {
            % utilisateur → agent
            \draw[->,thick] ([xshift=3pt]pic cs:a2a-user-end) -- ([xshift=-3pt]pic cs:a2a-agent);
            % agent → opérateurs (divergence)
            \draw[->,thick] ([xshift=3pt]pic cs:a2a-agent-end) to[out=0,in=180] ([xshift=-3pt]pic cs:a2a-op1);
            \draw[->,thick] ([xshift=3pt]pic cs:a2a-agent-end) to[out=0,in=180] ([xshift=-3pt]pic cs:a2a-op2);
            \draw[->,thick] ([xshift=3pt]pic cs:a2a-agent-end) to[out=0,in=180] ([xshift=-3pt]pic cs:a2a-op3);
            % opérateurs → analyse (convergence)
            \draw[->,thick] ([xshift=6pt]pic cs:a2a-op1-end) to[out=0,in=180] ([xshift=0pt,yshift=10pt]pic cs:a2a-analyse);
            \draw[->,thick] ([xshift=6pt]pic cs:a2a-op2-end) to[out=0,in=180] ([xshift=0pt]pic cs:a2a-analyse);
            \draw[->,thick] ([xshift=6pt]pic cs:a2a-op3-end) to[out=0,in=180] ([xshift=0pt,yshift=-10pt]pic cs:a2a-analyse);
            % analyse → fin
            \draw[->,thick] ([xshift=3pt]pic cs:a2a-analyse-end) -- ([xshift=-3pt]pic cs:a2a-fin);
        }
        \vspace{-1.5cm}\begin{center}
            \textit{chaque opérateur peut utiliser des MCP}
        \end{center}
    \end{tcolorbox}
\end{Exemple}