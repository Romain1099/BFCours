\subsection{Mise en place}

La section suivante suit une structure différente du reste de cette formation. 

Il s'agit de \acc{méthodes} permettant d'utiliser l'intelligence artificielle pour toute opération concernant des documents LaTeX. 

\bcattention Bien que le monde de l'IA évolue très rapidement, si vous n'utilisez pas l'IA en général, ce qui suit constituera un point d'entrée \acc{robuste}. 

Les guides de bonne pratiques utilisés sont issus des instructions \acc{Anthropic} afin d'obtenir le meilleur du potentiel des modèles de langage.

On se restreindra aux pratiques validées par la communauté et qui semblent nécessaires dans le contexte précis d'utilisation de \LaTeX. 

\begin{EXO}{Utiliser l'IA pour LaTeX}{IA-1}
    Après lecture de cette section :
    \begin{tcbenumerate}[2]
        \tcbitem Choisir un modèle d'IA.
        \tcbitem Brainstormer quelques idées d'agents IA adaptés à vos besoins pour des documents spécifiques. 
        \tcbitem Concevoir un prompt \acc{respectant les bonnes pratiques} pour que l'IA réalise l'\acc{un de ces documents}
        \tcbitem Préparer quelques demandes tester l'efficacité du modèle. 
        \tcbitem[raster multicolumn=2] Produire un court rapport ( éventuellement documenté avec images ) de vos expérimentations et de vos retours.
    \end{tcbenumerate}
\end{EXO}