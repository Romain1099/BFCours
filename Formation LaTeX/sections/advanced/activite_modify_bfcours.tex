\subsection{Adapter bfcours à ses besoins}

\begin{Methode}[Créer votre propre package]
Il convient de créer votre propre package pour apporter vos modifications de façon globale. ( voir la section \acc{Ajouter un localtexmf})

Il suffira alors de l'utiliser dans vos documents : 

\showcmd{usepackage}[\{bfcours\}]

\showcmd{usepackage}[\{adapt-bfcours\}]\mycomment{Votre package qui modifiera bfcours}
\end{Methode}
\begin{Methode}[Personnaliser un package]
Puisque \LaTeX\ est orienté vers la personnalisation, il est possible d'adapter n'importe quel package à vos besoins. 
Il y a plusieurs façons de procéder : 
\begin{tcbenumerate}[2]
    \tcbitem Créer vos propres commandes qui simplifient l'utilisation de celles données dans les packages utilisés. 

    \showcmd{newcommand}[\{$\backslash$bonjour\}\{bonjour\}] \mycomment{définition d'origine}

    \showcmd{newcommand}[\{$\backslash$mybonjour\}\{Bonjour !\}] \mycomment{définition que vous utiliserez}

    \tcbitem Réécrire certaines commandes pour qu'elles agissent différemment. 

    C'est un peu plus compliqué, il faut retrouver le code d'origine de la commande, recopier son contenu et modifier la copie. 

    \showcmd{newcommand}[\{$\backslash$bonjour\}\{bonjour\}] \mycomment{définition d'origine}

    \showcmd{renewcommand}[\{$\backslash$bonjour\}\{Bonjour !\}] \mycomment{réécrit la définition de la commande}
\end{tcbenumerate}
\end{Methode}
