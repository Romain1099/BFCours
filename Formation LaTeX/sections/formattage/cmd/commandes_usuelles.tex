\subsection{Commandes usuelles}

\begin{Methode}[Formattage du texte]
    Les commandes suivantes permettent d'effectuer la plupart des opérations sur le texte. 
    \begin{tcbenumerate}[2]
        \tcbitem  \showcmd[red]{underline\{texte\}}\ Souligner.
        \tcbitem  \showcmd[green]{hl\{texte\}}\ Surligner.
        \tcbitem  \showcmd[blue]{textbf\{texte\}}\ Mettre en gras.
        \tcbitem  \{\showcmd[purple]{color\{couleur\}texte}\}\ Mettre en couleur.
        \tcbitem  \showcmd[orange]{acc[couleur]\{texte\}}\ Accentuer (bf).
        \tcbitem  \showcmd[brown]{voc[couleur]\{texte\}}\ Vocabulaire (bf).
        \tcbitem  \showcmd[cyan]{textsc\{texte\}}\ Vocabulaire.
        \tcbitem  \showcmd[magenta]{fbox\{texte\}}\ Encadrer.
        \tcbitem  \showcmd[olive]{emph\{texte\}}\ Mettre en italique.
        \tcbitem  \showcmd[teal]{frquote\{texte\}}\ Citer.
        \tcbitem  \showcmd[violet]{surligner[couleur]\{texte\}}\ Surligner (bf).
        \tcbitem  \showcmd[pink]{encadrer[couleur]\{texte\}}\ Encadrer (bf).
    \end{tcbenumerate}
\end{Methode}


\begin{EXO}{Formattage du texte}{FT-1}
    \begin{tcbenumerate}
        \tcbitem Ouvrir le fichier 
        
        \displayFilePath{fichiers\_de\_la\_formation/1.Exercices\_formattage/premier\_document/premier\_document.tex}.
        \tcbitem \tcbitempoint{2}Reproduire la phrase suivante dans laquelle \acc{chaque commande} est utilisée \encadrer[green]{une seule fois} : 

        \begin{center}
            En mathématiques, on peut \underline{souligner} les éléments importants, 
            \textbf{mettre en gras} ou \acc[red]{accentuer} des mots-clés, 
            \emph{mettre en italique} les théorèmes, 
            \hl{surligner} - ou bien \surligner[purple]{surligner} - des résultats, 
            utiliser les \textsc{petites capitales} ou la {\color{green!75!black}commande} \frquote{voc} pour le\voc{vocabulaire}, 
            et \fbox{encadrer}, ou encore \encadrer[red]{encadrer} les formules essentielles.
        \end{center}

    \end{tcbenumerate}
    
\exocorrection

%le meme texte mais avec des showcmd de différentes couleurs ( correspondant aux couleurs de la méthode )
En mathématiques, on peut \showcmd[red]{underline\{souligner\}} les éléments importants, 
\showcmd[blue]{textbf\{mettre en gras\}} ou \showcmd[orange]{acc[red]\{accentuer\}} des mots-clés, 
\showcmd[olive]{emph\{mettre en italique\}} les théorèmes, 
\showcmd[green]{hl\{surligner\}} - ou bien \showcmd[violet]{surligner[purple]\{surligner\}} - des résultats, 
utiliser les \showcmd[cyan]{textsc\{petites capitales\}} ou la \{\showcmd[green]{color\{green!75!black\}commande}\} \showcmd[teal]{frquote\{voc\}} pour le\showcmd[brown]{voc\{vocabulaire\}}, 
et \showcmd[magenta]{fbox\{encadrer\}}, ou encore \showcmd[pink]{encadrer[red]\{encadrer\}} les formules essentielles.

\begin{Remarque}
\begin{tcbenumerate}
    \tcbitem  Les commandes \frquote{acc}, \frquote{encadrer}, \frquote{surligner} sont relatives au package \bfcours.
    \tcbitem  La commande color doit être entourée par des crochets. Dans le cas contraire, la commande agit sur tout le paragraphe. 
\end{tcbenumerate}
\end{Remarque}
\end{EXO}

