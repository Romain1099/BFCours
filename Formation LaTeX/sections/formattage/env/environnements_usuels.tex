\subsection{Environnements usuels}

\begin{Methode}[Formattage avec environnements]
    Les environnements suivants permettent d'effectuer la plupart des opérations de mise en page.
    \begin{tcbenumerate}[2]
        \tcbitem  \showenv[red]{center}\ Centrer un texte/contenu.
        \tcbitem  \showenv[green]{flushleft}\ Aligner à gauche.
        \tcbitem  \showenv[blue]{flushright}\ Aligner à droite.
        \tcbitem  \showenv[purple]{multicols}[\{n\}][
            Contenu gauche\\
            \showcmd{columnbreak}\\
            Contenu droit
        ]\ Affichage sur $n$ colonnes avec les packages standard.
        \tcbitem  \showenv[brown]{tcolorbox}[[options]]\ une boite.
        \tcbitem  \showenv[orange]{minipage}[\{0.475$\backslash$textwidth\}]\ une petite page dans la page.
        \tcbitem  \showenv[cyan]{itemize}[[label=\$ $\backslash$bullet\$]]\ Listes à puces.
        \tcbitem  \showenv[magenta]{enumerate}\ Listes numérotées.
        \tcbitem  \showenv[purple]{tcbenumerate}[[n][i]]\ Listes numérotées sur $n$ colonnes à partir de l'indice $i$ de bfcours.
        \tcbitem  \showenv[teal]{tabular}[[titre]\{structure\}]\ Tableaux.
        \tcbitem  \showenv[teal]{align*}\ Mode maths aligné ( séparateur \& ).
        \tcbitem  \showenv[teal]{tcbtab}[[titre]\{structure\}]\  Tableaux encadrés de bfcours.
        \tcbitem  \showenv[violet]{crep}\ Cadre de réponse (bf).
        \tcbitem  \showenv[pink]{MultiColonnes}[\{n\}[options]]\ Disposition en $n$ colonnes avec boites de style \frquote{options} (bf).
    \end{tcbenumerate}
\end{Methode}

\newpage

\begin{EXO}{Mise en page avec environnements}{ENV-1}
    \tcbitempoint{2}Reproduire la mise en page suivante en utilisant les environnements adéquats :

\begin{center}
\fbox{\begin{minipage}{0.9\textwidth}
\begin{center}
\textbf{Les différents types d'alignements}
\end{center}

\vspace{-0.25cm}\begin{multicols}{2}
\begin{flushleft}
Ce texte est aligné à gauche grâce à l'environnement \texttt{flushleft}.
\end{flushleft}

\columnbreak

\begin{flushright}
Ce texte est aligné à droite grâce à l'environnement \texttt{flushright}.
\end{flushright}
\end{multicols}

\vspace{-1cm}\begin{center}
Ce texte est centré grâce à l'environnement \texttt{center}.
\end{center}

\begin{tcbtab}[Résumé des environnements d'alignement]{|l|c|r|}
\hline
\textbf{Environnement} & \textbf{Description} & \textbf{Utilisation} \\
\hline
flushleft & Aligne à gauche & Texte courant \\
center & Centre le texte & Titres, équations \\
flushright & Aligne à droite & Signature, date \\
\hline
\end{tcbtab}
\end{minipage}}
\end{center}

\exocorrection

\showenv{center}[][
	 \showenv{minipage}[\{0.9$\backslash$textwidth\}][
		 \showenv{center}[][
		 	\showcmd{textbf}[Les différents types d'alignements]
		]\\
		\showenv{multicols}[\{2\}][
			\showenv{flushleft}[][
				Ce texte est aligné à gauche grâce à l'environnement \encadrer{flushleft}.
			]\\
			\showcmd{columnbreak}[]\\
			\showenv{flushright}[][
		 		Ce texte est aligné à droite grâce à l'environnement \encadrer{flushright}.
			]
		]\\
        \showenv{center}[][
            \showenv{tcbtab}[[Résumé des environnements d'alignement]\{|l|c|r|\}][
                \showcmd{hline}\\
                flushleft \& Aligne à gauche \& Texte courant $\backslash$$\backslash$\\
                center \& Centre le texte \& Titres, équations $\backslash$$\backslash$\\
                flushright \& Aligne à droite \& Signature, date $\backslash$$\backslash$\\
                \showcmd{hline}
            ]
        ]
	]
]
\end{EXO}
