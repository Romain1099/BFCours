\subsection{Mise en page avancée}

\begin{Methode}[Grilles et structures]
    Pour créer des structures complexes, utiliser :
    \begin{tcbenumerate}[2]
        \tcbitem La commande suivante est à la base des colonnes avec \frquote{tcolorbox} : 
        
        \showenv[red]{tcbraster}[[options]]

        Options principales:
        \begin{itemize}[label=$\bullet$]
            \item \acc{raster columns=n} : Nombre de colonnes.
            \item \acc{size=fbox} : Taille des cellules.
            \item \acc{raster equal height} : Hauteur égale pour toutes les cellules.
            \item \acc{raster column skip=1pt} : Espacement entre colonnes.
            \item \acc{raster row skip=1pt} : Espacement entre lignes.
        \end{itemize}
        
        Dans un \texttt{tcbraster}, chaque élément doit être une \texttt{tcolorbox} ou une commande qui génère une \texttt{tcolorbox}.
        
        \tcbitem  On utilisera une version pratique de \bfcours\ qui simplifie la structure en colonnes :

        \showenv[green]{Multicolonnes}[\{NbColonnes\}[Options]][
            \showcmd{tcbitem}[[options]] Contenu
        ]\\

        Commentaires : 
        \begin{itemize}[label=$\bullet$]
            \item Dispose le contenu en \acc{grille} en le stockant dans des \acc{tcolorbox}.
            \item La commande \showcmd{tcbitem} permet de changer de boite - et donc de colonne. 
            \item Les options de MultiColonnes modifient de façon globale le style des boites. 
            \item Les options de tcbitem modifie de façon locale le style de la boite. 
        \end{itemize}

    \end{tcbenumerate}
\end{Methode}

\begin{Remarque}
    Les options ci-dessus sont celles utilisables dans le package \vocref{https://ctan.org/pkg/tcolorbox}{tcolorbox}.    
    
    Puisque \acc{tout est boite}, la documentation de ce package est \acc{l'outil par exellence} pour produire du contenu de qualité et utiliser correctement les nombreuses options disponibles. 
\end{Remarque}
%\newpage
\begin{EXO}{Structure en grille}{ENV-3}
    \tcbitempoint{2}Créer une structure en grille avec 3 colonnes contenant différentes propriétés mathématiques :

\begin{tcolorbox}[blankest]
    \begin{tcbraster}[
            raster columns=3, 
            size=fbox,
            raster width=0.99\textwidth, 
            raster equal height, 
            raster column skip=1pt, 
            raster row skip=1pt
        ]
        \begin{tcolorbox}[title=Produit remarquable]
            \begin{center}
            $(a+b)^2 = a^2 + 2ab + b^2$
            \end{center}
        \end{tcolorbox}
        \begin{tcolorbox}[title=Identité trigonométrique]
            \begin{center}
            $\sin^2(\alpha) + \cos^2(\alpha) = 1$
            \end{center}
        \end{tcolorbox}
        \begin{tcolorbox}[title=Dérivée d'un produit]
            \begin{center}
            $(u \times v)' = u' \times v + u \times v'$
            \end{center}
        \end{tcolorbox}
    \end{tcbraster}
\end{tcolorbox}

\exocorrection

% Solution avec showcmd et showenv
\showenv[orange]{tcolorbox}[[blankest]][
    \showenv[red]{tcbraster}[][[\textcolor{green!75!black}{\% Options du raster}\\
                    \phantom{AAA}raster columns=3,\textcolor{green!75!black}{\% Trois colonnes}\\
                    \phantom{AAA}size=fbox,\textcolor{green!75!black}{\% Style compact pour les boites}\\
                    \phantom{AAA}raster equal height=rows,\textcolor{green!75!black}{\% Même hauteur par ligne} \\
                    \phantom{AAA}raster width=0.99\showcmd{textwidth}\textcolor{green!75!black}{\% Taille}, \\
                    \phantom{AAA}raster column skip=1pt,\textcolor{green!75!black}{\% Marge entre chaque colonne} \\
                    \phantom{AAA}raster row skip=1pt\textcolor{green!75!black}{\% Marge entre chaque ligne}\\
                    ]\\
        \showenv[blue]{bfbox}[[title=Produit remarquable]][
            \showenv[blue]{center}[][
                \$(a+b)\^\ 2 = a\^\ 2 + 2ab + b\^\ 2\$\\
            ]
        ]\\
        \showenv[purple]{tcolorbox}[[title=Identité trigonométrique]][
            \showenv[blue]{center}[][
                \$\showcmd{sin}\^\ 2(\showcmd{alpha}) + \showcmd{cos}\^\ 2(\showcmd{alpha}) = 1\$
            ]
        ]\\
        \showenv[green]{tcolorbox}[[title=dérivée d'un produit]][
            \showenv[blue]{center}[][
                \$(u \showcmd{times} v)' = u' \showcmd{times} v + u \showcmd{times} v'\$
            ]
        ]
    ]
]
\end{EXO}

\newpage

\vspace{-0.7cm}\begin{Exemple}[Environnement MultiColonnes]
    \vspace{-0.35cm}\showenv{MultiColonnes}[\{2\}[colframe=black,boxrule=0.4pt,halign=center]\textcolor{green!75!black}{\textit{\% Default [blank]}}][
  \showcmd{tcbitem}[[valign=bottom]] Hauteur adaptée pour \acc{chaque ligne}.\\
  \showcmd{tcbitem}[[title=La seule boite titrée,colframe=black,boxrule=0.4pt]] Bonjour !\\
  \showcmd{tcbitem}[[raster multicolumn=2,halign=center]]Fusion facile..\\
  \showcmd{tcbitem}[[colback=green!25!white]]Ligne 3 colonne 1\\
  \showcmd{tcbitem}[] Utilisation basique.
]

\vspace{0.2cm}\hrule

\vspace{0.2cm}

\acc{Rendu du code : }

\begin{MultiColonnes}{2}[colframe=black,boxrule=0.4pt,halign=center]%
  \tcbitem[valign=bottom] Alignement disponible.
  \tcbitem[title=La seule boite titrée,colframe=black,boxrule=0.4pt] Hauteur adaptée \acc{par ligne}.
    \tcbitem[raster multicolumn=2,halign=center] Fusion facile..
  \tcbitem[colback=green!25!white] Ligne 3 colonne 1
  \tcbitem Utilisation basique.
\end{MultiColonnes}

\vspace{0.2cm}\hrule

\vspace{0.2cm}

\acc{Version utilisant les options par défaut : } ( enlever les \acc{options})

\begin{MultiColonnes}{2}%
    \tcbitem[valign=bottom] Alignement disponible.
    \tcbitem[title=La seule boite titrée,colframe=black,boxrule=0.4pt] Hauteur adaptée \acc{par ligne}.
    \tcbitem[raster multicolumn=2,halign=center] Fusion facile..
  \tcbitem[colback=green!25!white] Ligne 3 colonne 1
  \tcbitem Utilisation basique.
\end{MultiColonnes}

\vspace{0.2cm}\hrule

\vspace{0.2cm}

\acc{Style modifié : } en utilisant un \showcmd[orange]{tcbset} précis. 

\begin{MultiColonnes}{4}%
\tcbitem[raster multicolumn=2] \showcmd[orange]{tcbset}\{\\
  \phantom{A}ColonnesBaseStyle/.style=\{\\
    \phantom{AA}top=0pt,\\
    \phantom{AA}bottom=0pt,\\
    \phantom{AA}left=0pt,\\
    \phantom{AA}right=0pt,\\
    \phantom{AA}colback=blue!5!white,\\
    \phantom{AA}colframe=blue!75!black,\\
    \phantom{AA}before title={\showcmd{dimcoloredsquare}[\{white\}\{1.5\}]},\\
    \phantom{AA}after title=\showcmd{hfill}\showcmd{today}\},\\
    \phantom{AA}boxrule=0.4pt\\
    \phantom{AA}\}\\
\}
\tcbitem[raster multicolumn=2] 
    \tcbset{
    ColonnesBaseStyle/.style={
        top=0pt,
        bottom=0pt,
        left=0pt,
        right=0pt,
        colback=blue!5!white,
        colframe=blue!75!black,
        before title={\dimcoloredsquare{white}{1.5}\ },
        after title={\ \today},
        boxrule=0.4pt
    }
    }
    \begin{MultiColonnes}{2}%
        \tcbitem[valign=bottom] Alignement disponible.
        \tcbitem[title=La seule boite titrée,colframe=black,boxrule=0.4pt] Hauteur adaptée \acc{par ligne}.
        \tcbitem[raster multicolumn=2,halign=center] Fusion facile..
        \tcbitem[colback=green!25!white] Ligne 3 colonne 1
        \tcbitem Utilisation basique.
    \end{MultiColonnes}
\end{MultiColonnes}
\end{Exemple}
\vspace{-0.7cm}