\subsection{La géométrie}
\begin{Methode}[Géométrie et \LaTeX]
        
        L'utilisation de la géométrie repose essentiellement sur le package \acc{TikZ}. 

        Son utilisation est \acc{omniprésente} en \LaTeX\ - la bordure de cet environnement est \acc{dessinée} avec une figure TikZ.

        Malheureusement, la maîtrise de ce package nécessiterait une formation à part entière (- cf. sa documentation CTAN et les ouvrages associés \vocnoindexref{http://math.et.info.free.fr/TikZ/index.html}{TikZ pour l'impatient} ou bien le package \vocnoindexref{https://ctan.org/pkg/tkz-euclide}{tkz-euclid}.\\


        Néanmoins, le professeur de mathématiques sera ravi d'apprendre que le logiciel de géométrie dynamique \acc{Geogebra} ou d'autres comme celui du groupe \acc{coopmaths} permettent un \acc{export TikZ} des figures réalisées. 

        \vspace{-0.45cm}\begin{center}
            \bouton{Fichier}$\longrightarrow$\bouton{Exporter}$\longrightarrow$\bouton{Graphique vers PGF/TikZ}
        \end{center}

        \vspace{-0.3cm}Il suffit ensuite de se laisser guider par l'interface proposée par Geogebra.

        On veillera à copier coller le contenu généré \acc{entre les bornes} \showcmd{begin\{document\}} et \showcmd{end\{document\}}.

        L'utilisation d'un \acc{script de reformattage} des figures TikZ ainsi générée est \acc{hautement conseillé} - les outils de \bfcours\ proposent un tel programme adapté à plusieurs situations. 
    \end{Methode}