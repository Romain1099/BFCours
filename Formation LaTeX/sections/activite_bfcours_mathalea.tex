\def\rdifficulty{2}
\begin{EXO}{Utiliser Mathalea}{A-1}
    \itempoint{2}Construire une fiche d'exercice sur le thème des fractions pour le niveau $6^\text{ème}$.
    
    Pour cela : 
    \begin{tcbenumerate}
        \tcbitem Ouvrir le répertoire \displayFilePath{fichiers\_de\_la\_formation/Atelier\_exercices\_mathalea} dans VScode.
        \tcbitem Aller sur \href{https://coopmaths.fr/alea/}{https://coopmaths.fr/alea/} et \acc{élaborer} une série d'exercices.
        \tcbitem \acc{Copier} le \acc{contenu seul} généré dans le fichier
        
        \displayFilePath{fichiers\_de\_la\_formation/Atelier\_exercices\_mathalea/enonce.tex}
        \tcbitem \bcattention Le document généré n'est pas adapté à la présentation de \bfcours. 

        Il faut le formatter à l'aide du programme \displayFilePath{fichiers\_de\_la\_formation/programmes/mathalea\_adapter.exe} ( double cliquer )

        Ensuite, il suffit de choisir le fichier \displayFilePath{fichiers\_de\_la\_formation/Atelier\_exercices\_mathalea/enonce.tex}  dans lequel vous aurez collé le contenu généré par MathAlea.
        \tcbitem Le programme produit un fichier \displayFilePath{fichiers\_de\_la\_formation/Atelier\_exercices\_mathalea/enonce\_TOOLS.tex}. 

        Il convient de \acc{modifier} la ligne : 
        \showcmd[red]{input\{enonce\}} en \showcmd[green]{input\{enonce\_TOOLS\}} dans le fichier principal.

        \tcbitem \acc{Explorer} les possibilités offertes par \LaTeX\ en modifiant certains exercices et / ou leurs corrections de sorte à \acc{utiliser} les \acc{environnements de réponse} de \bfcours.
    \end{tcbenumerate}
\end{EXO}