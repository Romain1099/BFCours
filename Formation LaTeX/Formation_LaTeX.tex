\documentclass[a4paper,11pt,fleqn]{article}

\usepackage[left=1cm,right=1cm,top=0.5cm,bottom=2cm]{geometry}

\usepackage{bfcours}
%\usepackage{bfcours-fonts}
\usepackage{fourier}
%\usepackage{bfcours-quatrieme}
\usepackage{verbatim}

\tcbuselibrary{documentation}


\def\rdifficulty{1}
\setrdexo{%left skip=1cm,
display exotitle,
exo header = tcolorbox,
%display tags,
skin = bouyachakka,
lower ={box=crep},
display score,
display level,
save lower,
score=\points,
level=\rdifficulty,
overlay={\node[inner sep=0pt,
anchor=west,rotate=90, yshift=0.3cm]
at (frame.south west) {\thetags[0]} ;}
%]%obligatoire
}
\setrdcrep{seyes, correction=true, correction color=monrose, correction font = \large\bfseries}

%\usepackage{bfcours-fonts-dys}
\usepackage{listings} % Pour afficher le code
\lstdefinestyle{tcblatex}{
    language=[LaTeX]TeX,
    basicstyle=\ttfamily\small,
    numbers=none,
    keywordstyle=\bfseries\color{blue!70!black},
    morekeywords={begin,end,documentclass,usepackage,newcommand,renewcommand,section,subsection,item,maCommandeExemple,tcbitem,boite},
    commentstyle=\itshape\color{green!60!black},
    stringstyle=\color{purple!70!black},
    texcsstyle=*\color{red!70!black},
    emphstyle=\color{orange!90!black},
    emph={figure,table,tabular,array,displaymath,equation,eqnarray,align,gather,verbatim,quote,enumerate,itemize,proof,theorem,Exercice,Methode,Definition,Theorem,EXOEVAL,Propriete,tcbenumerate,Colonnes,tcbraster,tcolorbox,None,tcbtab,tcbtabx,},
    showstringspaces=false,
    breaklines=true,
    breakatwhitespace=false,
    keepspaces=true,
    columns=flexible,
    tabsize=2
}

%\tcbuselibrary{minted}
\tcbuselibrary{listings}
\tcbset{listing engine=listings}
\usepackage{soul}
\newcommand{\bfcours}{\acc{BFcours}}
\newcommand{\mathalea}{\acc{MathAléa}}

\newcommand{\maCommandeExemple}[1]{Bonjour #1 !}

\NewTotalTCBox{\cmverb}{ O{blue} v !O{} }{ fontupper=\ttfamily,nobeforeafter,tcbox raise base,arc=0pt,outer arc=0pt,
top=0pt,bottom=0pt,left=0mm,right=0mm,
leftrule=0pt,rightrule=0pt,toprule=0.3mm,bottomrule=0.3mm,boxsep=0.5mm,
colback=#1!10!white,colframe=#1!50!black,#3}{#2}
\NewTotalTCBox{\cmbox}{ O{blue} m !O{} }{ fontupper=\ttfamily,nobeforeafter,tcbox raise base,arc=0pt,outer arc=0pt,
top=0pt,bottom=0pt,left=0mm,right=0mm,
leftrule=0pt,rightrule=0pt,toprule=0.3mm,bottomrule=0.3mm,boxsep=0.5mm,
colback=#1!10!white,colframe=#1!50!black,#3}{#2}

 \newcommand{\bs}{$\backslash$}

\NewDocumentCommand{\showcmd}{O{blue} m O{}}{%
  \cmbox[#1]{\bs #2#3}
}

\NewDocumentCommand{\showcmdpar}{O{purple} m O{}}{%
  \cmbox[#1]{$\backslash$#2}\{
	\hspace{0.5cm}\begin{minipage}{\textwidth-0.5cm}#3\end{minipage}%
   \}
}
% Définition de la boîte colorée pour l'affichage de code
\NewTotalTCBox{\envbox}{ O{red} m !O{} }{ 
  fontupper=\ttfamily,
  nobeforeafter,
  tcbox raise base,
  arc=0pt,
  outer arc=0pt,
  top=0pt,
  bottom=0pt,
  left=0mm,
  right=0mm,
  leftrule=0pt,
  rightrule=0pt,
  toprule=0.3mm,
  bottomrule=0.3mm,
  boxsep=0.5mm,
  colback=#1!10!white,
  colframe=#1!50!black,
  #3
}{#2}

% Commande améliorée pour afficher la syntaxe d'un environnement
\NewDocumentCommand{\showenv}{O{red} m O{} O{Contenu}}{%
  \showcmd[#1]{begin\{#2\}#3}
	
	\hspace{0.5cm}\begin{minipage}{0.9\textwidth}#4\end{minipage}

  \showcmd[#1]{end\{#2\}}
}


% Configuration pour JSON
\definecolor{stringcolor}{RGB}{42,0,255}
\definecolor{numbercolor}{RGB}{0,128,0}
\definecolor{bracescolor}{RGB}{150,0,0}

\lstdefinelanguage{json}{
    basicstyle=\ttfamily\footnotesize,
    showstringspaces=false,
    breaklines=true,
    frame=lines,
    stringstyle=\color{stringcolor},
    keywordstyle=\color{bracescolor},
    morekeywords={},
    literate=
     *{0}{{{\color{numbercolor}0}}}{1}
      {1}{{{\color{numbercolor}1}}}{1}
      {2}{{{\color{numbercolor}2}}}{1}
      {3}{{{\color{numbercolor}3}}}{1}
      {4}{{{\color{numbercolor}4}}}{1}
      {5}{{{\color{numbercolor}5}}}{1}
      {6}{{{\color{numbercolor}6}}}{1}
      {7}{{{\color{numbercolor}7}}}{1}
      {8}{{{\color{numbercolor}8}}}{1}
      {9}{{{\color{numbercolor}9}}}{1}
      {:}{{{\color{bracescolor}{:}}}}{1}
      {,}{{{\color{bracescolor}{,}}}}{1}
      {\{}{{{\color{bracescolor}{\{}}}}{1}
      {\}}{{{\color{bracescolor}{\}}}}}{1}
      {[}{{{\color{bracescolor}{[}}}}{1}
      {]}{{{\color{bracescolor}{]}}}}{1},
}

\renewcommand\SequenceItem[5][]{
\stepcounter{tmp}%
%\xdef\mycommentary{#4}
%\xdef\SequenceItemription{#3}
\node[myshape,#2] (foo\thetmp) {};
\node[font=\color{white}] at (foo\thetmp) {\begin{minipage}{2.5cm}\begin{center}#3\end{center}\end{minipage}};
\node[mynewtext,anchor=north west] (desc\thetmp) at (foo\thetmp.north west) 
  {%
  \tcbsidebyside[
    bicolor,
    height=#1,
    width=15cm,
    nobeforeafter,
    bottom=2pt,
    colback=myblue!25!white,
    colbacklower=myblue!10!white,
    fonttitle=\bfseries,
    center title,
    drop lifted shadow,
    colframe=myblue,
    sharp corners=northwest,
  ]{
        #4
   }{
        #5
    }
  };
}
%\currentAccentColor
\newcommand{\relierhautdepart}[2][depart]{
  \begin{tikzpicture}[remember picture, overlay]
    \draw[->, thick, dashed, color=monrose, line width=1.5pt, opacity=0.7] 
        ($(pic cs:#1) + (0.03, 0.15)$) to[out=45, in=90, looseness=1.8] ($(pic cs:#2) + (0.1, 0.1)$);
  \end{tikzpicture}
}
\newcommand{\relierbasdepart}[2][depart]{
  \begin{tikzpicture}[remember picture, overlay]
    \draw[->, thick, dashed, color=\currentAccentColor, line width=1.2pt, opacity=0.7] 
          (pic cs:#1) to[out=-90, in=90, looseness=3] (pic cs:#2);
  \end{tikzpicture}
}

% Pour l'activité bfcours-1 seulement 
% Commandes pour les flèches de connexion
\newcommand{\relierhaut}[3][monrose]{
  \begin{tikzpicture}[remember picture, overlay]
    \draw[->, thick, dashed, color=#1, line width=1.5pt, opacity=0.8] 
        (#2) to[out=45, in=180, looseness=1.2] (#3);
  \end{tikzpicture}
}

\newcommand{\relierbas}[3][monbleu]{
  \begin{tikzpicture}[remember picture, overlay]
    \draw[->, thick, dashed, color=#1, line width=1.5pt, opacity=0.8] 
        (#2) to[out=-45, in=180, looseness=1.2] (#3);
  \end{tikzpicture}
}

%

\usepackage{dirtree}
\newtcbox{\pcadre}[1][red]{on line,
arc=7pt,colback=#1!10!white,colframe=#1!50!black,
before upper={\rule[-3pt]{0pt}{10pt}},boxrule=1pt,
boxsep=0pt,left=6pt,right=6pt,top=2pt,bottom=2pt}
\newcommand{\tikzinclude}[1]{%
    \stepcounter{tikzfigcounter}%
    \csname tikzfig#1\endcsname
}
\newcommand{\figureLongueurCercle}{
    \definecolor{ttzzqq}{rgb}{0.2,0.6,0.}
\definecolor{qqqqff}{rgb}{0.,0.,1.}
\definecolor{xdxdff}{rgb}{0.49019607843137253,0.49019607843137253,1.}
\definecolor{ffwwqq}{rgb}{1.,0.4,0.}
\definecolor{ududff}{rgb}{0.30196078431372547,0.30196078431372547,1.}
\definecolor{uuuuuu}{rgb}{0.26666666666666666,0.26666666666666666,0.26666666666666666}
\definecolor{yqqqyq}{rgb}{0.5019607843137255,0.,0.5019607843137255}
\begin{tikzpicture}[line cap=round,line join=round,>=triangle 45,x=1.0cm,y=1.0cm]
\clip(-1.3,-2.) rectangle (6.,1.);
\draw [line width=0.8pt,color=yqqqyq] (0.,0.) circle (0.75cm);
\draw [line width=0.8pt,color=yqqqyq] (-1.,-1.5)-- (3.7123889803846897,-1.5);
\draw [line width=0.8pt,dash pattern=on 1pt off 1pt,color=ffwwqq] (-0.75,0.)-- (0.75,0.);
\draw [line width=0.8pt,dash pattern=on 1pt off 1pt,color=ffwwqq] (-0.02306940646219445,0.05382861507845321) -- (-0.02306940646219445,-0.05382861507845321);
\draw [line width=0.8pt,dash pattern=on 1pt off 1pt,color=ffwwqq] (0.023069406462194016,0.05382861507845321) -- (0.023069406462194016,-0.05382861507845321);
\draw [line width=0.8pt,dash pattern=on 1pt off 1pt,color=qqqqff] (0.,0.)-- (0.4867299416577562,0.5706084155476829);
\draw [line width=0.8pt,dash pattern=on 1pt off 1pt,color=qqqqff] (0.1994863125988175,0.32273277731191763) -- (0.2872436290589386,0.2478756382357651);
\draw [line width=0.8pt,dash pattern=on 1pt off 1pt,color=ffwwqq] (-1.,-1.2)-- (0.5,-1.2);
\draw [line width=0.8pt,dash pattern=on 1pt off 1pt,color=ffwwqq] (-0.2730694064621942,-1.1461713849215465) -- (-0.2730694064621942,-1.253828615078453);
\draw [line width=0.8pt,dash pattern=on 1pt off 1pt,color=ffwwqq] (-0.2269305935378057,-1.1461713849215465) -- (-0.2269305935378057,-1.253828615078453);
\draw [line width=0.8pt,dash pattern=on 1pt off 1pt,color=ffwwqq] (0.5,-1.2)-- (2.,-1.2);
\draw [line width=0.8pt,dash pattern=on 1pt off 1pt,color=ffwwqq] (1.2269305935378059,-1.1461713849215465) -- (1.2269305935378059,-1.253828615078453);
\draw [line width=0.8pt,dash pattern=on 1pt off 1pt,color=ffwwqq] (1.2730694064621944,-1.1461713849215465) -- (1.2730694064621944,-1.253828615078453);
\draw [line width=0.8pt,dash pattern=on 1pt off 1pt,color=ffwwqq] (2.,-1.2)-- (3.5,-1.2);
\draw [line width=0.8pt,dash pattern=on 1pt off 1pt,color=ffwwqq] (2.726930593537806,-1.1461713849215465) -- (2.726930593537806,-1.253828615078453);
\draw [line width=0.8pt,dash pattern=on 1pt off 1pt,color=ffwwqq] (2.773069406462194,-1.1461713849215465) -- (2.773069406462194,-1.253828615078453);
\draw [line width=0.8pt,dash pattern=on 1pt off 1pt,color=ttzzqq] (3.7123889803846897,-1.)-- (3.7123889803846892,-1.6);
\draw [line width=0.8pt,dash pattern=on 1pt off 1pt,color=ttzzqq] (3.5,-1.)-- (3.5,-1.6);
\draw (1.0529201296198805,0.538595212100376) node[anchor=north west] {$r = 0.75$};
\draw (1.0606099317739452,0.1387255000890093) node[anchor=north west] {$c = {\color{blue}r} \times {\color{red}\pi} \approx 4.71$};
\draw (0.7607076477654201,0.8615669025710952) node[anchor=north west] {$\textbf{Circonférence du cercle} \mathcal{C} :$};
\draw (-0.738803772277205,0.8615669025710952) node[anchor=north west,xshift=-0.5cm] {$\mathcal{C}$};
\begin{scriptsize}
\draw[color=yqqqyq] (1.2451651834714994,-1.6952923136554323) node {$4.71$};
\draw [color=uuuuuu] (0.75,0.)-- ++(-3.0pt,0 pt) -- ++(6.0pt,0 pt) ++(-3.0pt,-3.0pt) -- ++(0 pt,6.0pt);
\draw [color=ududff] (-1.,-1.2)-- ++(-3.0pt,0 pt) -- ++(6.0pt,0 pt) ++(-3.0pt,-3.0pt) -- ++(0 pt,6.0pt);
\draw[color=ffwwqq] (0.030176443129269655,-0.1957808936128071) node {$1.5$};
\draw [color=xdxdff] (0.4867299416577562,0.5706084155476829)-- ++(-3.0pt,0 pt) -- ++(6.0pt,0 pt) ++(-3.0pt,-3.0pt) -- ++(0 pt,6.0pt);
%\draw[color=qqqqff] (0.41466655083250686,0.20408881839855963) node {$0.75$};
\draw [color=uuuuuu] (0.5,-1.2)-- ++(-3.0pt,0 pt) -- ++(6.0pt,0 pt) ++(-3.0pt,-3.0pt) -- ++(0 pt,6.0pt);
\draw[color=ffwwqq] (0.09938466251585235,-1.1031775477924468) node {$1.5$};
\draw [color=uuuuuu] (2.,-1.2)-- ++(-3.0pt,0 pt) -- ++(6.0pt,0 pt) ++(-3.0pt,-3.0pt) -- ++(0 pt,6.0pt);
\draw[color=ffwwqq] (1.4297204351690531,-1.041659130559929) node {$1.5$};
\draw [color=uuuuuu] (3.5,-1.2)-- ++(-3.0pt,0 pt) -- ++(6.0pt,0 pt) ++(-3.0pt,-3.0pt) -- ++(0 pt,6.0pt);
\draw[color=ffwwqq] (2.9446114595198076,-1.0493489327139938) node {$1.5$};
\end{scriptsize}
\end{tikzpicture}
}

\newcommand{\figureAireCarre}{
    \definecolor{ffwwqq}{rgb}{1.,0.4,0.}
    \definecolor{zzttqq}{rgb}{0.6,0.2,0.}
    \begin{tikzpicture}[line cap=round,line join=round,>=triangle 45,x=1.0cm,y=1.0cm]
    \clip(-0.5,-0.5) rectangle (6.5,3.);
    \fill[line width=0.8pt,color=zzttqq,fill=zzttqq,fill opacity=0.10000000149011612] (0.,0.) -- (2.,0.) -- (2.,2.) -- (0.,2.) -- cycle;
    \draw [line width=0.8pt,color=zzttqq] (0.,0.)-- (2.,0.);
    \draw [line width=0.8pt,color=zzttqq] (2.,0.)-- (2.,2.);
    \draw [line width=0.8pt,color=zzttqq] (2.,2.)-- (0.,2.);
    \draw [line width=0.8pt,color=zzttqq] (0.,2.)-- (0.,0.);
    \draw [line width=0.8pt,dash pattern=on 1pt off 1pt,color=ffwwqq] (0.,2.1990485449122428)-- (2.,2.2);
    \draw (2.3265445321089646,2.0241005763481263) node[anchor=north west] {$\mathcal{A}_{\text{Carré}} = c \times c $};
    \draw (2.3357854785212573,1.515848523672014) node[anchor=north west] {$\mathcal{A}_{\text{Carré}} = 2 \times 2 = 4 $};
    \begin{scriptsize}
    \draw[color=ffwwqq] (0.9866073023264862,2.4260817452828696) node {$2$};
    \end{scriptsize}
    \end{tikzpicture}
}

\newcommand{\solEquation}{
    \begin{center}
        \tdplotsetmaincoords{60}{120}
        \begin{tikzpicture}[tdplot_main_coords, scale=0.8]
            % Paramètres de vue en 3D
            
            %\begin{tdplotpicture}[tdplot main coords]
        
                % Axes en 3D
                \draw[thick,->] (0,0,0) -- (4,0,0) node[anchor=north east]{$x$};
                \draw[thick,->] (0,0,0) -- (0,4,0) node[anchor=north west]{$y$};
                \draw[thick,->] (0,0,0) -- (0,0,4) node[anchor=south]{$z$};
        
                % Points pour le plan (2x + y - z = 3)
                \coordinate (A) at (0,3.5,0);   % Point d'intersection avec y
                \coordinate (B) at (2,1.5,0); % Point d'intersection avec x
                \coordinate (C) at (0,0,3);  % Point d'intersection avec z
        
                % Tracer le plan en utilisant les points
                \filldraw[fill=blue!20,opacity=0.5] (A) -- (B) -- (C) -- cycle;
        
                % Lignes principales du plan
                \draw[thick, blue] (A) -- (B) -- (C) -- cycle;
        
                % Hachures pour la région de solutions (en dessous du plan)
                \begin{scope}
                    \clip (A) -- (B) -- (C) -- cycle;
                    \foreach \z in {-3,-2.5,...,0} {
                        \draw[dashed, blue!60,opacity=0.7] (0,3,\z) -- (1.5,0,\z);
                    }
                \end{scope}
        
                % Étiquettes des points
                \node at (A) [anchor=south west] {$(0;3{,}5;0)$};
                \node at (B) [anchor=north] {$(2;1{,}5;0)$};
                \node at (C) [anchor=east] {$(0;0;3)$};
        
        \end{tikzpicture}
    \end{center}
}

\hypersetup{
    pdfauthor={R.Deschamps},
    pdfsubject={},
    pdfkeywords={},
    pdfproducer={LuaLaTeX},
    pdfcreator={Boum Factory}
}

\newtcbox{\displayFilePath}[1][]{%
  enhanced,
  nobeforeafter,
  tcbox raise base,
  fontupper=\ttfamily\footnotesize,
  colback=black,
  colframe=black,
  coltext=white,
  arc=1pt,
  boxsep=2pt,
  left=2pt,
  right=2pt,
  top=2pt,
  bottom=2pt,
  #1
}


% Définition de la commande \bouton
\newcommand{\bouton}[1]{%
  \tikz[baseline=(text.base)]{%
    \node[
      draw=gray!60,
      fill=gray!10,
      rounded corners=2pt,
      inner sep=2pt,
      inner ysep=1pt,
      text height=1.5ex,
      text depth=.25ex
    ] (text) {#1};
  }%
}
\displayitempointsfalse % Ne pas afficher les boîtes

\newcommand{\mycomment}[1]{
  \textcolor{green!75!black}{\% #1}
}

% Commandes robustes pour afficher du XML
\newcommand{\safett}[1]{{\ttfamily\detokenize{#1}}}
\newcommand{\xmltag}[1]{\texttt{<}\safett{#1}\texttt{>}}
\newcommand{\xmlctag}[1]{\texttt{</}\safett{#1}\texttt{>}}
\newcommand{\xmlattr}[2]{\safett{#1}\texttt{="}\safett{#2}\texttt{"}}

% Boîte pour afficher du code XML
\NewTotalTCBox{\xmlbox}{ O{blue} m !O{} }{ 
  fontupper=\ttfamily,
  nobeforeafter,
  tcbox raise base,
  arc=0pt,
  outer arc=0pt,
  top=0pt,
  bottom=0pt,
  left=0mm,
  right=0mm,
  leftrule=0pt,
  rightrule=0pt,
  toprule=0.3mm,
  bottomrule=0.3mm,
  boxsep=0.5mm,
  colback=#1!5!white,
  colframe=#1!30!black,
  #3
}{#2}



\begin{document}

\setcounter{pagecounter}{0}
\setcounter{ExoMA}{0}
\setcounter{prof}{0}

\def\points{1}
\def\rdifficulty{1}
\chapitre[
    Prof% : $\mathbf{6^{\text{ème}}}$,$\mathbf{5^{\text{ème}}}$,$\mathbf{4^{\text{ème}}}$,$\mathbf{3^{\text{ème}}}$,$\mathbf{2^{\text{nde}}}$,$\mathbf{1^{\text{ère}}}$,$\mathbf{T^{\text{Le}}}$,
    ]{
    \LaTeX% : ,Equations
    }{
    Boum% : Collège,Lycée
    }{
    Factory% : Amadis Jamyn,Eugène Belgrand
    }{
    % : ,\tableauPresenteEvalSixieme{}{10},\tableofcontents
    }{
    Formation % : Cours,Exercices
    }

    \tableofcontents
    
    \newpage 
    \phantom{a}
    \vfill
    \tableaurecapsequence{
        \centering\textbf{Nombre d'heures} & \multicolumn{4}{c}{$6$ \ heures}\\ 
	\hline
    
        \enteteContenu%
    
        \competence{Se familiariser avec les environnements de travail \LaTeX}
        \competence{Comprendre le fonctionnement général d'un document \LaTeX}
        \competence{Utiliser l'environnement EXO}
        \competence{Utiliser les exercices générés par la plateforme \mathalea}
        \competence{Utiliser les environnements didactiques de \bfcours}
        \competence{Utiliser des outils numériques pour faciliter l'utilisation de \LaTeX}
        \competence{Mettre en place un agent IA dédié à \LaTeX}
        \competence{Créer son propre prompt optimisé}
        \competence{Créer son propre package}
    }
    
    \vfill
    \printvocindex
    \vfill
    \newpage
    
\section{Introduction}
\begin{tikzpicture}[]
    \SequenceItem[3cm]{}{Introduction \\
    
    \vspace{0.2cm}\hspace{-0.7cm}\overlaychrono{30}}{%Rappels de calculs littéral :
        \begin{itemize}[label=$\bullet$]
            \item Plan.
            \item Comment fonctionne \LaTeX.
            \item Téléchargement des logiciels\voc{MikTeX} et\voc{VSCode}.
        \end{itemize}
    }{\`A faire : \begin{itemize}[label=$\bullet$]
        \item Se connecter à un point d'accès mobile. 
        \item Télécharger les ressources de la formation.
    \end{itemize}
    }
    \SequenceItem[6cm]{below = 1.7cm of desc1.south west}{Activité\\
    
    \vspace{0.2cm}\hspace{-0.7cm}\overlaychrono{60}}{
        \begin{itemize}[label=$\bullet$,itemsep=0em]
            \item Setup des logiciels.
            \item Point théorique sur la structure d'un document \LaTeX.
            \item Le fameux \frquote{Hello World !}
        \end{itemize}
    }{
        \`A faire : 
        \begin{itemize}[label=$\bullet$]
            \item Télécharger l'extension\voc{LaTeX Workshop} et\voc{PDF Viewer} sur VSCode.
            \item[\bcattention] Télécharger le fichier \\\frquote{setup\_vscode.json} pour configurer LaTeX Workshop.
            \item Construire son premier document \LaTeX.
        \end{itemize}
    }
    \SequenceItem[4cm]{below = 1.7cm of desc2.south west}{Point théorique\\
    
    \vspace{0.2cm}\hspace{-0.7cm}\overlaychrono{30}}{%Les bouteilles :\\
    \begin{itemize}[itemsep=1.5em]%[label=$\bullet$,itemsep=0em]
        \item[$\bullet$]\voc{Commandes} et\voc{Environnements}
        \item[\bclampe] \underline{Formattage} 
        $\overbrace{\text{du}}^\text{ou des\tikzmark{depart}}$
            \pcadre[blue]{\textbf{\color{red}te}{\Huge x}{\scriptsize te\tikzmark{ici}}}%
        %
        .\relierhautdepart{ici}
        \item[$\bullet$]\voc{CTAN} et\voc{LaTeX Stack Exchange}
    \end{itemize}
    }{
        \`A faire :
        \begin{itemize}[label=\bcoutil,itemsep=0em]
            \item Se familiariser avec les commandes basiques.
            \item Savoir où\voc{se documenter}.
        \end{itemize}
    }
        \SequenceItem[4cm]{below = 1.7cm of desc3.south west}{BFcours\\
    
        \vspace{0.2cm}\hspace{-0.7cm}\overlaychrono{60}}{%Nature d'une égalité
        \begin{itemize}[label=$\bullet$,itemsep=0em]
            \item Une pause s'impose ! 
            \item Point théorique sur le package\voc{BFcours}.
            \item Premier document avec \acc{BFcours}.
        \end{itemize}
    }{
        \`A faire : 
        \begin{itemize}[label=$\bullet$]
            \item Téléchargement du package \acc{BFcours}.
            \item Compiler un premier document avec le package \bfcours.
        \end{itemize}
    }
    \SequenceItem[4cm]{below = 1.7cm of desc4.south west}{Ateliers\\
    
    \vspace{0.2cm}\hspace{-0.7cm}\overlaychrono{90}}{%Distributivité
    \begin{itemize}[label=$\bullet$,itemsep=0em]
        \item Utiliser\voc{MathAléa}.
        \item Utiliser les outils \bfcours.
        \item Construire une séance avec \bfcours.
        \item Construire une évaluation avec \bfcours.
    \end{itemize}
    }{
        \`A faire : 
        \begin{itemize}[label=$\bullet$]
            \item Construire une fiche d'exercices.
            \item Construire une séance de cours.
            \item Utiliser les outils \bfcours.
        \end{itemize}
    }
\end{tikzpicture}

\begin{tikzpicture}[]

    \SequenceItem[4cm]{below = 1.7cm of desc5.south west}{Ateliers avancés\\
    
    \vspace{0.2cm}\hspace{-0.7cm}\overlaychrono{90}}{%Réduire et factoriser
    \begin{itemize}[label=$\bullet$,itemsep=0em]
        \item Adapter \bfcours\ ( et \LaTeX en général ) à ses besoins.
        \item Utiliser le générateur de questions Flash de \bfcours.
        \item Utiliser Python pour générer des questions aléatoires.
    \end{itemize}
    }{
        \`A faire : \begin{itemize}[label=$\bullet$]
            \item Utiliser les fonctionnalités avancées de \LaTeX.
            \item Explorer le potentiel des combinaisons de\voc{Python} et de \LaTeX.
        \end{itemize}
    }
\end{tikzpicture}

\newpage
On peut utiliser le mode maths de plusieurs manières : 
\begin{tcolorbox}[blank]
\begin{tcbenumerate}[2]
    \tcbitem Inline via \textcolor{red}{\$ contenu maths \$} : $\left(\text{\Large{E}}\right)~~5x + 3 = 2^\frac{3}{x}$
    \tcbitem Inline via le mode \acc{display} \textcolor{red}{$\backslash$( contenu maths $\backslash$)} : \(\left(\text{\Large{E}}\right)~~5x + 3 = 2^\frac{3}{x}\)
    \tcbitem En valeur via le mode \acc{centré} \textcolor{red}{$\backslash$[ contenu maths $\backslash$]} : \[\left(\text{\Large{E}}\right)~~5x + 3 = 2^\frac{3}{x}\]
    \tcbitem En mode \acc{align} ( énuméré ) ou \acc{align*} ( non énuméré ) :  
    
    \showenv{align}[][contenu maths]
    \begin{align}
        \left(\text{\Large{E}}\right)~~5x + 3 &= 2^\frac{3}{x}\\
        &= e^{\frac{3}{x}\ln(2)}
    \end{align}
\end{tcbenumerate}

Dans le mode mathématiques, si l'on veut écrire du texte, il faut l'appeler \acc{dans la commande text} : 

\showcmd{text}[\{Texte à écrire\}].


Le mode mathématique peut donc aussi être utilisé pour formatter du texte : 

$6^{\text{ème}}$
\end{tcolorbox}% : On peut utiliser le mode maths de plusieurs manières : 
\begin{tcolorbox}[blank]
\begin{tcbenumerate}[2]
    \tcbitem Inline via \textcolor{red}{\$ contenu maths \$} : $\left(\text{\Large{E}}\right)~~5x + 3 = 2^\frac{3}{x}$
    \tcbitem Inline via le mode \acc{display} \textcolor{red}{$\backslash$( contenu maths $\backslash$)} : \(\left(\text{\Large{E}}\right)~~5x + 3 = 2^\frac{3}{x}\)
    \tcbitem En valeur via le mode \acc{centré} \textcolor{red}{$\backslash$[ contenu maths $\backslash$]} : \[\left(\text{\Large{E}}\right)~~5x + 3 = 2^\frac{3}{x}\]
    \tcbitem En mode \acc{align} ( énuméré ) ou \acc{align*} ( non énuméré ) :  
    
    \showenv{align}[][contenu maths]
    \begin{align}
        \left(\text{\Large{E}}\right)~~5x + 3 &= 2^\frac{3}{x}\\
        &= e^{\frac{3}{x}\ln(2)}
    \end{align}
\end{tcbenumerate}

Dans le mode mathématiques, si l'on veut écrire du texte, il faut l'appeler \acc{dans la commande text} : 

\showcmd{text}[\{Texte à écrire\}].


Le mode mathématique peut donc aussi être utilisé pour formatter du texte : 

$6^{\text{ème}}$
\end{tcolorbox},

\newpage

\section{Correction des exercices}

\rdexocorrection{0}

\end{document}