% Exemple d'extension : Cartes supplémentaires

% Pour ajouter des cartes, copiez ce modèle dans enonce.tex
% Respectez toujours l'alternance énoncés/solutions

% ===== PAGE X : ÉNONCÉS 17-20 (Systèmes d'équations) =====
\begin{pageenonces}
    \begin{tcolorbox}[carteenonce={17}]
        \eqcarte{
            \begin{cases}
                x + y = 5 \\
                x - y = 1
            \end{cases}
        }
    \end{tcolorbox}
    \begin{tcolorbox}[carteenonce={18}]
        \eqcarte{
            \begin{cases}
                2x + 3y = 12 \\
                x - y = 1
            \end{cases}
        }
    \end{tcolorbox}
    \begin{tcolorbox}[carteenonce={19}]
        \eqcarte{|x - 3| = 5}
    \end{tcolorbox}
    \begin{tcolorbox}[carteenonce={20}]
        \eqcarte{\sqrt{x + 1} = 3}
    </end{tcolorbox}
\end{pageenonces}

% ===== PAGE X+1 : SOLUTIONS 17-20 =====
\begin{pagesolutions}
    \begin{tcolorbox}[cartesolution={17}]
        \solcarte{
            \textbf{Méthode par addition :}\\[0.3cm]
            \begin{align*}
                x + y &= 5 \\
                x - y &= 1
            \end{align*}
            En additionnant : $2x = 6$\\
            Donc $x = 3$\\[0.3cm]
            En remplaçant : $3 + y = 5$\\
            Donc $y = 2$\\[0.3cm]
            \textbf{Solution :} $(x, y) = (3, 2)$
        }
    \end{tcolorbox}
    \begin{tcolorbox}[cartesolution={18}]
        \solcarte{
            \textbf{Méthode par substitution :}\\[0.3cm]
            De la 2e équation : $x = y + 1$\\[0.3cm]
            Dans la 1re : $2(y + 1) + 3y = 12$\\
            $2y + 2 + 3y = 12$\\
            $5y = 10$\\
            $y = 2$\\[0.3cm]
            Donc $x = 2 + 1 = 3$\\[0.3cm]
            \textbf{Solution :} $(x, y) = (3, 2)$
        }
    \end{tcolorbox}
    \begin{tcolorbox}[cartesolution={19}]
        \solcarte{
            \textbf{Définition de la valeur absolue :}\\[0.3cm]
            $|x - 3| = 5$ signifie :\\
            $x - 3 = 5$ ou $x - 3 = -5$\\[0.3cm]
            Si $x - 3 = 5$ : $x = 8$\\
            Si $x - 3 = -5$ : $x = -2$\\[0.3cm]
            \textbf{Solutions :} $x = 8$ ou $x = -2$
        }
    \end{tcolorbox}
    \begin{tcolorbox}[cartesolution={20}]
        \solcarte{
            \begin{align*}
                \sqrt{x + 1} &= 3 \\
                x + 1 &= 9 \\
                x &= 8
            \end{align*}
            \textbf{Vérification :} $\sqrt{8 + 1} = \sqrt{9} = 3$ ✓\\[0.3cm]
            \textbf{Solution :} $x = 8$
        }
    \end{tcolorbox}
\end{pagesolutions}