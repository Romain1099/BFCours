% Cartes auto-correctives - Équations
% Structure : Pages d'énoncés alternées avec pages de solutions

% ===== PAGE 1 : ÉNONCÉS 1-4 =====
\begin{pageenonces}
    \begin{tcolorbox}[carteenonce={1}]
        2x + 4 = -3x + 7
    \end{tcolorbox}
    \begin{tcolorbox}[carteenonce={2}]
        2(x - 3) = -(5x - 4)
    \end{tcolorbox}
    \begin{tcolorbox}[carteenonce={3}]
        10x - 8 = x - 7
    \end{tcolorbox}
    \begin{tcolorbox}[carteenonce={4}]
        4 - (-5x + 9) = 3x + 2
    \end{tcolorbox}
\end{pageenonces}

% ===== PAGE 2 : SOLUTIONS 1-4 =====
\begin{pagesolutions}
    \begin{tcolorbox}[cartesolution={1}]
        \begin{align*}
            2x + 4 &= -3x + 7 \\
            2x + 4 \modif{+ 3x} &= -3x + 7 \modif{+ 3x} \\
            \reduc{5x} + 4 &= 7 \\
            5x + 4 \modif{- 4} &= 7 \modif{- 4} \\
            5x &= \reduc{3} \\
            \frac{5x}{\modif{5}} &= \frac{3}{\modif{5}} \\
            x &= \reduc{\frac{3}{5}} = \reduc{0,6}
        \end{align*}
    \end{tcolorbox}
    \begin{tcolorbox}[cartesolution={2}]
        \begin{align*}
            2(x - 3) &= -(5x - 4) \\
            \reduc{2x - 6} &= \reduc{-5x + 4} \\
            2x - 6 \modif{+ 5x} &= -5x + 4 \modif{+ 5x} \\
            \reduc{7x} - 6 &= 4 \\
            7x - 6 \modif{+ 6} &= 4 \modif{+ 6} \\
            7x &= \reduc{10} \\
            x &= \reduc{\frac{10}{7}}
        \end{align*}
    \end{tcolorbox}
    \begin{tcolorbox}[cartesolution={3}]
        \begin{align*}
            10x - 8 &= x - 7 \\
            10x - 8 \modif{- x} &= x - 7 \modif{- x} \\
            \reduc{9x} - 8 &= -7 \\
            9x - 8 \modif{+ 8} &= -7 \modif{+ 8} \\
            9x &= \reduc{1} \\
            x &= \reduc{\frac{1}{9}}
        \end{align*}
    \end{tcolorbox}
    \begin{tcolorbox}[cartesolution={4}]
        \begin{align*}
            4 - (-5x + 9) &= 3x + 2 \\
            4 \reduc{+ 5x - 9} &= 3x + 2 \\
            \reduc{5x - 5} &= 3x + 2 \\
            5x - 5 \modif{- 3x} &= 3x + 2 \modif{- 3x} \\
            \reduc{2x} - 5 &= 2 \\
            2x - 5 \modif{+ 5} &= 2 \modif{+ 5} \\
            2x &= \reduc{7} \\
            x &= \reduc{\frac{7}{2}}
        \end{align*}
    \end{tcolorbox}
\end{pagesolutions}
% ===== PAGE 3 : ÉNONCÉS 5-8 =====
\begin{pageenonces}
    \begin{tcolorbox}[carteenonce={5}]
        6(x + 7) = 5x + 4
    \end{tcolorbox}
    \begin{tcolorbox}[carteenonce={6}]
        8 - (-2x + 3) = x + 5
    \end{tcolorbox}
    \begin{tcolorbox}[carteenonce={7}]
        3x + 5 = 13x + 6
    \end{tcolorbox}
    \begin{tcolorbox}[carteenonce={8}]
        12x + 6 = 6x + 4
    \end{tcolorbox}
\end{pageenonces}

% ===== PAGE 4 : SOLUTIONS 5-8 =====
\begin{pagesolutions}
    \begin{tcolorbox}[cartesolution={5}]
        \begin{align*}
            6(x + 7) &= 5x + 4 \\
            \reduc{6x + 42} &= 5x + 4 \\
            6x + 42 \modif{- 5x} &= 5x + 4 \modif{- 5x} \\
            \reduc{x} + 42 &= 4 \\
            x + 42 \modif{- 42} &= 4 \modif{- 42} \\
            x &= \reduc{-38}
        \end{align*}
    \end{tcolorbox}
    \begin{tcolorbox}[cartesolution={6}]
        \begin{align*}
            8 - (-2x + 3) &= x + 5 \\
            8 \reduc{+ 2x - 3} &= x + 5 \\
            \reduc{5 + 2x} &= x + 5 \\
            5 + 2x \modif{- x} &= x + 5 \modif{- x} \\
            5 + \reduc{x} &= 5 \\
            5 + x \modif{- 5} &= 5 \modif{- 5} \\
            x &= \reduc{0}
        \end{align*}
    \end{tcolorbox}
    \begin{tcolorbox}[cartesolution={7}]
        \begin{align*}
            3x + 5 &= 13x + 6 \\
            3x + 5 \modif{- 3x} &= 13x + 6 \modif{- 3x} \\
            5 &= \reduc{10x} + 6 \\
            5 \modif{- 6} &= 10x + 6 \modif{- 6} \\
            \reduc{-1} &= 10x \\
            \frac{-1}{\modif{10}} &= \frac{10x}{\modif{10}} \\
            \reduc{-\frac{1}{10}} &= x
        \end{align*}
    \end{tcolorbox}
    \begin{tcolorbox}[cartesolution={8}]
        \begin{align*}
            12x + 6 &= 6x + 4 \\
            12x + 6 \modif{- 6x} &= 6x + 4 \modif{- 6x} \\
            \reduc{6x} + 6 &= 4 \\
            6x + 6 \modif{- 6} &= 4 \modif{- 6} \\
            6x &= \reduc{-2} \\
            \frac{6x}{\modif{6}} &= \frac{-2}{\modif{6}} \\
            x &= \reduc{-\frac{2}{6}} = \reduc{-\frac{1}{3}}
        \end{align*}
    \end{tcolorbox}
\end{pagesolutions}
% ===== PAGE 5 : ÉNONCÉS 9-12 =====
\begin{pageenonces}
    \begin{tcolorbox}[carteenonce={9}]
        11x + 7 = 7x + 8
    \end{tcolorbox}
    \begin{tcolorbox}[carteenonce={10}]
        2(2x - 5) = -6x + 8
    \end{tcolorbox}
    \begin{tcolorbox}[carteenonce={11}]
        4 - (9x - 7) = -5x - 4
    \end{tcolorbox}
    \begin{tcolorbox}[carteenonce={12}]
        10x - 1 = 11
    \end{tcolorbox}
\end{pageenonces}

% ===== PAGE 6 : SOLUTIONS 9-12 =====
\begin{pagesolutions}
    \begin{tcolorbox}[cartesolution={9}]
        \begin{align*}
            11x + 7 &= 7x + 8 \\
            11x + 7 \modif{- 7x} &= 7x + 8 \modif{- 7x} \\
            \reduc{4x} + 7 &= 8 \\
            4x + 7 \modif{- 7} &= 8 \modif{- 7} \\
            4x &= \reduc{1} \\
            \frac{4x}{\modif{4}} &= \frac{1}{\modif{4}} \\
            x &= \reduc{\frac{1}{4}} = \reduc{0,25}
        \end{align*}
    \end{tcolorbox}
    \begin{tcolorbox}[cartesolution={10}]
        \begin{align*}
            2(2x - 5) &= -6x + 8 \\
            \reduc{4x - 10} &= -6x + 8 \\
            4x - 10 \modif{+ 6x} &= -6x + 8 \modif{+ 6x} \\
            \reduc{10x} - 10 &= 8 \\
            10x - 10 \modif{+ 10} &= 8 \modif{+ 10} \\
            10x &= \reduc{18} \\
            \frac{10x}{\modif{10}} &= \frac{18}{\modif{10}} \\
            x &= \reduc{\frac{18}{10}} = \reduc{1,8}
        \end{align*}
    \end{tcolorbox}
    \begin{tcolorbox}[cartesolution={11}]
        \begin{align*}
            4 - (9x - 7) &= -5x - 4 \\
            4 \reduc{- 9x + 7} &= -5x - 4 \\
            \reduc{-9x + 11} &= -5x - 4 \\
            -9x + 11 \modif{+ 5x} &= -5x - 4 \modif{+ 5x} \\
            \reduc{-4x} + 11 &= -4 \\
            -4x + 11 \modif{- 11} &= -4 \modif{- 11} \\
            -4x &= \reduc{-15} \\
            \frac{-4x}{\modif{-4}} &= \frac{-15}{\modif{-4}} \\
            x &= \reduc{\frac{15}{4}} = \reduc{3,75}
        \end{align*}
    \end{tcolorbox}
    \begin{tcolorbox}[cartesolution={12}]
        \begin{align*}
            10x - 1 &= 11 \\
            10x - 1 \modif{+ 1} &= 11 \modif{+ 1} \\
            10x &= \reduc{12} \\
            \frac{10x}{\modif{10}} &= \frac{12}{\modif{10}} \\
            x &= \reduc{\frac{12}{10}} = \reduc{1,2}
        \end{align*}
    \end{tcolorbox}
\end{pagesolutions}
% ===== PAGE 7 : ÉNONCÉS 13-16 (Équations du second degré) =====
\begin{pageenonces}
    \begin{tcolorbox}[carteenonce={13}]
        $9x^2 - 4 = 0$
    \end{tcolorbox}
    \begin{tcolorbox}[carteenonce={14}]
        $(4x + 3)(2x + 5) = 8x^2$
    \end{tcolorbox}
    \begin{tcolorbox}[carteenonce={15}]
        $x^2 + 7x = 0$
    \end{tcolorbox}
    \begin{tcolorbox}[carteenonce={16}]
        $32x^2 + 4 = 8x(4x + 9)$
    \end{tcolorbox}
\end{pageenonces}

% ===== PAGE 8 : SOLUTIONS 13-16 =====
\begin{pagesolutions}
    \begin{tcolorbox}[cartesolution={13}]
        \begin{align*}
            9x^2 - 4 &= 0 \\
            9x^2 &= \modif{4} \\
            x^2 &= \reduc{\frac{4}{9}} \\
            x &= \pm\sqrt{\frac{4}{9}} \\
            x &= \reduc{\pm\frac{2}{3}}
        \end{align*}
        \vspace{0.3cm}
        \textbf{Solutions :} $x = \frac{2}{3}$ ou $x = -\frac{2}{3}$
    \end{tcolorbox}
    \begin{tcolorbox}[cartesolution={14}]
        \begin{align*}
            (4x + 3)(2x + 5) &= 8x^2 \\
            \reduc{8x^2 + 20x + 6x + 15} &= 8x^2 \\
            8x^2 + \reduc{26x} + 15 &= 8x^2 \\
            8x^2 + 26x + 15 \modif{- 8x^2} &= 8x^2 \modif{- 8x^2} \\
            26x + 15 &= \reduc{0} \\
            26x &= \modif{-15} \\
            x &= \reduc{-\frac{15}{26}}
        \end{align*}
    \end{tcolorbox}
    \begin{tcolorbox}[cartesolution={15}]
        \begin{align*}
            x^2 + 7x &= 0 \\
            \reduc{x(x + 7)} &= 0
        \end{align*}
        \vspace{0.3cm}
        Un produit est nul si l'un des facteurs est nul :\\[0.2cm]
        \textbf{Soit} $x = 0$ \quad \textbf{Soit} $x + 7 = 0$\\
        \phantom{Soit $x = 0$} \quad\quad\quad $x = -7$\\[0.3cm]
        \textbf{Solutions :} $x = -7$ ou $x = 0$
    \end{tcolorbox}
    \begin{tcolorbox}[cartesolution={16}]
        \begin{align*}
            32x^2 + 4 &= 8x(4x + 9) \\
            32x^2 + 4 &= \reduc{32x^2 + 72x} \\
            32x^2 + 4 \modif{- 32x^2} &= 32x^2 + 72x \modif{- 32x^2} \\
            4 &= \reduc{72x} \\
            \frac{4}{\modif{72}} &= \frac{72x}{\modif{72}} \\
            x &= \reduc{\frac{4}{72}} = \reduc{\frac{1}{18}}
        \end{align*}
    \end{tcolorbox}
\end{pagesolutions}
% ===== PAGE 9 : ÉNONCÉS 17-20 (Équations produits) =====
\begin{pageenonces}
    \begin{tcolorbox}[carteenonce={17}]
        (x + 13)(x - 11) = 0
    \end{tcolorbox}
    \begin{tcolorbox}[carteenonce={18}]
        (x - 10)(3x + 9) = 0
    \end{tcolorbox}
    \begin{tcolorbox}[carteenonce={19}]
        (x - 1)(x + 17) = 0
    \end{tcolorbox}
    \begin{tcolorbox}[carteenonce={20}]
        (2x + 4)(x + 9) = 0
    \end{tcolorbox}
\end{pageenonces}

% ===== PAGE 10 : SOLUTIONS 17-20 =====
\begin{pagesolutions}
    \begin{tcolorbox}[cartesolution={17}]
        $(x + 13)(x - 11) = 0$\\[0.5cm]
        Un produit est nul si l'un de ses facteurs est nul.\\[0.3cm]
        $\bullet$ \textbf{Soit} $x + 13 = 0$\\
        \phantom{$\bullet$ Soit} $x = \reduc{-13}$\\[0.3cm]
        $\bullet$ \textbf{Soit} $x - 11 = 0$\\
        \phantom{$\bullet$ Soit} $x = \reduc{11}$\\[0.5cm]
        \textbf{Les solutions sont} $-13$ et $11$.
    \end{tcolorbox}
    \begin{tcolorbox}[cartesolution={18}]
        $(x - 10)(3x + 9) = 0$\\[0.5cm]
        Un produit est nul si l'un de ses facteurs est nul.\\[0.3cm]
        $\bullet$ \textbf{Soit} $x - 10 = 0$\\
        \phantom{$\bullet$ Soit} $x = \reduc{10}$\\[0.3cm]
        $\bullet$ \textbf{Soit} $3x + 9 = 0$\\
        \phantom{$\bullet$ Soit} $3x = \modif{-9}$\\
        \phantom{$\bullet$ Soit} $x = \reduc{-\frac{9}{3}} = \reduc{-3}$\\[0.5cm]
        \textbf{Les solutions sont} $10$ et $-3$.
    \end{tcolorbox}
    \begin{tcolorbox}[cartesolution={19}]
        $(x - 1)(x + 17) = 0$\\[0.5cm]
        Un produit est nul si l'un de ses facteurs est nul.\\[0.3cm]
        $\bullet$ \textbf{Soit} $x - 1 = 0$\\
        \phantom{$\bullet$ Soit} $x = \reduc{1}$\\[0.3cm]
        $\bullet$ \textbf{Soit} $x + 17 = 0$\\
        \phantom{$\bullet$ Soit} $x = \reduc{-17}$\\[0.5cm]
        \textbf{Les solutions sont} $-17$ et $1$.
    \end{tcolorbox}
    \begin{tcolorbox}[cartesolution={20}]
        $(2x + 4)(x + 9) = 0$\\[0.5cm]
        Un produit est nul si l'un de ses facteurs est nul.\\[0.3cm]
        $\bullet$ \textbf{Soit} $2x + 4 = 0$\\
        \phantom{$\bullet$ Soit} $2x = \modif{-4}$\\
        \phantom{$\bullet$ Soit} $x = \reduc{-\frac{4}{2}} = \reduc{-2}$\\[0.3cm]
        $\bullet$ \textbf{Soit} $x + 9 = 0$\\
        \phantom{$\bullet$ Soit} $x = \reduc{-9}$\\[0.5cm]
        \textbf{Les solutions sont} $-9$ et $-2$.
    \end{tcolorbox}
\end{pagesolutions}