% Exemples d'utilisation avancée des cartes auto-correctives

% Pour utiliser ces exemples, décommentez les sections souhaitées
% et ajoutez-les à votre fichier enonce.tex

% ====================================
% Cartes avec indices progressifs
% ====================================

% Définition d'un style pour les cartes avec indices
\tcbset{
    carteindice/.style={
        enhanced,
        width=\largeurcarte,
        height=\hauteurcarte,
        arc=3mm,
        boxrule=0.8pt,
        colback=orange!5!white,
        colframe=carteorange,
        halign=center,
        valign=center,
        fontupper=\normalsize,
        title={\faLightbulbO\ Indice},
        fonttitle=\bfseries\color{white},
        colbacktitle=carteorange,
    }
}

% Exemple de carte avec indice
\begin{comment}
\begin{tcbraster}[pagecarte]
    % Recto : Question
    \begin{tcolorbox}[carterecto]
        \eqcarte{(x + 2)^2 - 16 = 0}
    \end{tcolorbox}
    
    % Indice
    \begin{tcolorbox}[carteindice]
        Pense à l'identité remarquable :\\[0.3cm]
        $a^2 - b^2 = (a-b)(a+b)$\\[0.3cm]
        Ici : $a = x + 2$ et $b^2 = 16$
    \end{tcolorbox}
    
    % Verso : Solution
    \begin{tcolorbox}[carteverso]
        \solcarte{
            \begin{align*}
                (x + 2)^2 - 16 &= 0 \\
                (x + 2)^2 - 4^2 &= 0 \\
                [(x+2) - 4][(x+2) + 4] &= 0 \\
                (x - 2)(x + 6) &= 0
            \end{align*}
            \textbf{Solutions :} $x = 2$ ou $x = -6$
        }
    \end{tcolorbox}
\end{tcbraster}
\end{comment}