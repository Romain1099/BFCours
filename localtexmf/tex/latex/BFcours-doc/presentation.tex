Le package BFcours est constitué de tous les outils que j'utilise au quotidien pour développer mes cours. \\

L'objectif étant de modifier l'utilisation standard de LaTeX afin d'apporter une surcouche de style. \\

La philosophie générale consiste à produire et écrire du code LaTeX simple permettant de se concentrer sur le contenu. \\

Les commandes et environnements ainsi que leurs particularité sont décrites dans la suite de ce document. \\
Le lecteur pourra trouver de nombreux exemples d'utilisation du package dans la partie Exemples. \\
On pourra également se diriger vers le repo contenant les cours pour se fournir en exemples d'utilisations. 


\subsection{Mes habitudes}

Pour produire des documents de façon aisée, il est nécessaire de réfléchir à une structure des fichiers. \\

On pourra s'inspirer du fonctionnement de mes \frquote{notes de stages} en herboristerie : \vocref{https://github.com/Romain1099/Herboristerie-2024.git}{Stage Arsimed 2024}\\

L'IDE \textbf{VScode} offre de nombreuses extensions dédiées à \LaTeX à explorer. De simples modules d'autocomplétion permettront une bien meilleure expérience. \\

Extension \textbf{PDF Viewer} qui permet de lire des documents pdf directement dans VScode.\\
Il doit y avoir un module permettant de compiler directement des documents latex dans VScode mais je n'ai plus la référence en tete. 
