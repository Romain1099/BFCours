\subsection{Commandes d'accentuation du texte}

La package BFcours propose deux commandes d'accentuation du texte : 
\begin{itemize}[label = $\bullet$]
    \item \verb+\acc+ $\rightarrow$ met en gras et colore un mot
    \item \verb+\voc+ $\rightarrow$ comme acc, mais place le mot à l'index
\end{itemize}
Les deux commandes obéissent à la couleur \verb+\currentAccentColor+
\begin{tcolorbox}[colback=yellow!10!white, title=Exemple d'utilisation pour \texttt{voc et acc}]
    \begin{minipage}{0.45\textwidth}
    \begin{lstlisting}[breaklines]
    \voc{mot de vocabulaire}\\
    \acc{Un autre mot}
    \end{lstlisting}
    \end{minipage}
    \hfill
    \begin{minipage}{0.45\textwidth}
    \phantom{a}\\
    \voc{mot de vocabulaire}\\
    \acc{Un autre mot}
    \end{minipage}
\end{tcolorbox}

\begin{tcolorbox}[colback=yellow!10!white, title=Exemple d'utilisation pour \texttt{printvocindex}]
    \begin{minipage}{0.45\textwidth}
        \begin{lstlisting}[breaklines]
        \printvocindex % Affiche les mots de vocabulaire dans une boite.
        \end{lstlisting}
        \end{minipage}
        \hfill
        \begin{minipage}{0.45\textwidth}
        \phantom{a}\\
        \printvocindex
    \end{minipage}
\end{tcolorbox}   

\begin{tcolorbox}[colback=yellow!10!white, title=Exemple d'utilisation pour \texttt{newcommand}]
\begin{minipage}{0.45\textwidth}
\begin{lstlisting}[breaklines]
\pascalc
\end{lstlisting}
\end{minipage}
\hfill
\begin{minipage}{0.45\textwidth}
\phantom{a}\\
\pascalc
\end{minipage}
\end{tcolorbox}


\subsection{Commandes relatives au package de Régis Deleuze}

Les packages \acc{rdexo} et \acc{rdcrep} fournissent de nombreuses fonctionnalités. 
\begin{tcolorbox}[colback=yellow!10!white, title=Exemple d'utilisation pour \texttt{rdexo}]
    \begin{minipage}{0.45\textwidth}
    \begin{lstlisting}[breaklines]
        \def\points{2} % Commande permettant d'attribuer les points aux exercices
        \def\rdifficulty{2} % Permet de définir la difficulté estimée d'un exercice ( par défaut 1 )

        \begin{EXO}{Titre}{Code}
            Ceci est un exercice.
        \end{EXO}
    \end{lstlisting}
    \end{minipage}
    \hfill
    \begin{minipage}{0.45\textwidth}
    \phantom{a}\\
    \def\points{2} % Commande permettant d'attribuer les points aux exercices
        \def\rdifficulty{2} % Permet de définir la difficulté estimée d'un exercice ( par défaut 1 )

        \begin{EXO}{Titre}{Code}
            Ceci est un exercice.
        \end{EXO}
    \end{minipage}
    \end{tcolorbox}

\begin{tcolorbox}[colback=yellow!10!white, title=Exemple d'utilisation pour \texttt{points}]
\begin{minipage}{0.45\textwidth}
\begin{lstlisting}[breaklines]
    \def\points{2} % Commande permettant d'attribuer les points aux exercices
    \points
\end{lstlisting}
\end{minipage}
\hfill
\begin{minipage}{0.45\textwidth}
\phantom{a}\\
\def\points{2} % Commande permettant d'attribuer les points aux exercices
    \points
\end{minipage}
\end{tcolorbox}




\begin{tcolorbox}[colback=yellow!10!white, title=Exemple d'utilisation pour \texttt{tccrep}]
\begin{minipage}{0.45\textwidth}
\begin{lstlisting}[breaklines]
    \setrdcrep{seyes, correction=false, correction color=monrose, correction font = \large\bfseries}

    \tccrep[seyes=false]{1.2cm}{123}
    \tccrep[seyes=false]{1.2cm}{123}

    ----

    \setrdcrep{seyes, correction=true, correction color=monrose, correction font = \large\bfseries}
    \tccrep[seyes=false]{1.2cm}{123}
    \tccrep[seyes=false]{1.2cm}{123}
\end{lstlisting}
\end{minipage}
\hfill
\begin{minipage}{0.45\textwidth}
\phantom{a}\\
    \setrdcrep{seyes, correction=false, correction color=monrose, correction font = \large\bfseries}
    \tccrep[seyes=false]{1.2cm}{123}
    \tccrep[seyes=true]{1.2cm}{123}\\

    ----\\

    \setrdcrep{seyes, correction=true, correction color=monrose, correction font = \large\bfseries}
    \tccrep[seyes=false]{1.2cm}{123}
    \tccrep[seyes=true]{1.2cm}{123}
\end{minipage}
\end{tcolorbox}
\begin{tcolorbox}[colback=yellow!10!white, title=Exemple d'utilisation pour l'environnement\texttt{crep}]
    \begin{minipage}{0.45\textwidth}
    \begin{lstlisting}[breaklines]
        \setrdcrep{seyes, correction=false, correction color=monrose, correction font = \large\bfseries}
    
        \begin{crep}
            Ceci est une réponse
        \end{crep}
    
        ----
    
        \setrdcrep{seyes, correction=true, correction color=monrose, correction font = \large\bfseries}
        
        \begin{crep}
            Ceci est une réponse
        \end{crep}
    \end{lstlisting}
    \end{minipage}
    \hfill
    \begin{minipage}{0.45\textwidth}
    \phantom{a}\\
    \setrdcrep{seyes, correction=false, correction color=monrose, correction font = \large\bfseries}
    
    \begin{crep}
        Ceci est une réponse
    \end{crep}

    ----

    \setrdcrep{seyes, correction=true, correction color=monrose, correction font = \large\bfseries}
    
    \begin{crep}
        Ceci est une réponse
    \end{crep}
    \end{minipage}
    \end{tcolorbox}

\begin{tcolorbox}[colback=yellow!10!white, title=Exemple d'utilisation pour \texttt{mysquare}]
\begin{minipage}{0.45\textwidth}
\begin{lstlisting}[breaklines]
\mysquare
\end{lstlisting}
\end{minipage}
\hfill
\begin{minipage}{0.45\textwidth}
\phantom{a}\\
\mysquare
\end{minipage}
\end{tcolorbox}




\begin{tcolorbox}[colback=yellow!10!white, title=Exemple d'utilisation pour \texttt{tableaucompetence}]
\begin{minipage}{0.32\textwidth}
\begin{lstlisting}[breaklines]
\tableaucompetence{}
\end{lstlisting}
\end{minipage}
\hfill
\begin{minipage}{0.65\textwidth}
\phantom{a}\\
\tableaucompetence{}
\end{minipage}
\end{tcolorbox}



\begin{tcolorbox}[colback=yellow!10!white, title=Exemple d'utilisation pour \texttt{competence}]
\begin{minipage}{0.32\textwidth}
\begin{lstlisting}[breaklines]
    \tableaucompetence{
        \competence{C1-1}
        \competence{C1-1}
    }
\end{lstlisting}
\end{minipage}
\hfill
\begin{minipage}{0.65\textwidth}
\phantom{a}\\
    \tableaucompetence{
        \competence{C1-1}
        \competence{C1-1}
    }
\end{minipage}
\end{tcolorbox}

\subsection{Commandes relatives aux exercices et évaluations}

La commande  \texttt{itempoint} permet le calcul automatique des points d'un exercice en cours. \\
Elle permet en outre d'obtenir le total des points stocké dans un fichier annexe.\\

Il est nécessaire de définir \texttt{\backslash displaybaremepointstrue} \textbf{avant le début du document} afin de profiter de l'affichage des points.\\
Les totaux par exercice et pour le document seront tout de même calculés.
\begin{tcolorbox}[colback=yellow!10!white, title=Exemple d'utilisation pour \texttt{competence}]
\begin{minipage}{0.65\textwidth}
\begin{lstlisting}[breaklines]
    \displaybaremepointsfalse % Active par défaut l'affichage des points
	\itempoint{4}[0.5]aa\\
	\displaybaremepointstrue % Active par défaut l'affichage des points
	\itempoint{2}bb\\
	\itempoint{1.5}[-0.5]cc
\end{lstlisting}
\end{minipage}\\

\end{tcolorbox}
\displaybaremepointsfalse % Active par défaut l'affichage des points
	\itempoint{4}[0.5]aa\\
	\displaybaremepointstrue % Active par défaut l'affichage des points
\itempoint{2}bb\\
\itempoint{1.5}[-0.5]cc

\begin{tcolorbox}[colback=yellow!10!white, title=Exemple d'utilisation pour \texttt{competence}]
\begin{minipage}{0.65\textwidth}
\begin{lstlisting}[breaklines]
	Il y a \getsavedtotalpoints dans ce document.
\end{lstlisting}
\end{minipage}\\
	Il y a \getsavedtotalpoints points dans ce document.
\end{tcolorbox}

\subsection{Commandes d'impression}


\begin{tcolorbox}[colback=yellow!10!white, title=Exemple d'utilisation pour \texttt{imp}]
\begin{minipage}{0.45\textwidth}
\begin{lstlisting}[breaklines]
    \renewcommand{\impressFileName}{commands_doc_bfcours-commands-to-print}
    \immediate\openout\imprimfile=\impressFileName.tex
    \begingroup
    \imp{
        Ceci est un texte à imprimer
        }
    \endgroup
    \immediate\closeout\imprimfile

    \noindent \begin {tikzpicture}[remember picture, overlay]
\draw [dashed] (-1cm,0) -- (\paperwidth ,0);
\node at (\paperwidth -1,0) {\ding {36}};
\end {tikzpicture}\par 
\noindent \begin {tikzpicture}[overlay]
\node [draw,inner sep=3pt, xshift=-0.5cm,yshift=0.18cm] at (0,0) {\faPrint 1};
\end {tikzpicture}
 Ceci est un texte à imprimer 


\end{lstlisting}
\end{minipage}
\hfill
\begin{minipage}{0.45\textwidth}
    \phantom{a}\\
    \renewcommand{\impressFileName}{commands_doc_bfcours-commands-to-print}
    \immediate\openout\imprimfile=\impressFileName.tex
    \begingroup
    \imp{
        Ceci est un texte à imprimer
        }
    \endgroup
    \immediate\closeout\imprimfile

    \noindent \begin {tikzpicture}[remember picture, overlay]
\draw [dashed] (-1cm,0) -- (\paperwidth ,0);
\node at (\paperwidth -1,0) {\ding {36}};
\end {tikzpicture}\par 
\noindent \begin {tikzpicture}[overlay]
\node [draw,inner sep=3pt, xshift=-0.5cm,yshift=0.18cm] at (0,0) {\faPrint 1};
\end {tikzpicture}
 Ceci est un texte à imprimer 


\end{minipage}
\end{tcolorbox}



\begin{tcolorbox}[colback=yellow!10!white, title=Exemple d'utilisation pour \texttt{repfill}]
\begin{minipage}{0.45\textwidth}
\begin{lstlisting}[breaklines]
    \setrdcrep{seyes, correction=false, correction color=monrose, correction font = \large\bfseries}

    \repfill{Une réponse}\\

    ----\\

    \setrdcrep{seyes, correction=true, correction color=monrose, correction font = \large\bfseries}
    \repfill{Une réponse}

\end{lstlisting}
\end{minipage}
\hfill
\begin{minipage}{0.45\textwidth}
\phantom{a}\\
\setrdcrep{seyes, correction=false, correction color=monrose, correction font = \large\bfseries}

    \repfill{Une réponse}\\

    ----\\

    \setrdcrep{seyes, correction=true, correction color=monrose, correction font = \large\bfseries}
    \repfill{Une réponse}
\end{minipage}
\end{tcolorbox}


\begin{tcolorbox}[colback=yellow!10!white, title=Exemple d'utilisation pour \texttt{repsim}]
\begin{minipage}{0.45\textwidth}
\begin{lstlisting}[breaklines]
\repsim{123}
\repsim[6cm]{Longue réponse}
\end{lstlisting}
\end{minipage}
\hfill
\begin{minipage}{0.45\textwidth}
\phantom{a}\\
\repsim{123}
\repsim[6cm]{Longue réponse}
\end{minipage}
\end{tcolorbox}


\begin{tcolorbox}[colback=yellow!10!white, title=Exemple d'utilisation pour \texttt{frquote}]
\begin{minipage}{0.45\textwidth}
\begin{lstlisting}[breaklines]
\frquote{Citons !}
\end{lstlisting}
\end{minipage}
\hfill
\begin{minipage}{0.45\textwidth}
\phantom{a}\\
\frquote{Citons !}
\end{minipage}
\end{tcolorbox}


\begin{tcolorbox}[colback=yellow!10!white, title=Exemple d'utilisation pour \texttt{rdifficulty}]
\begin{minipage}{0.45\textwidth}
\begin{lstlisting}[breaklines]
\def\rdifficulty{1}
\rdifficulty
\end{lstlisting}
\end{minipage}
\hfill
\begin{minipage}{0.45\textwidth}
\phantom{a}\\
\def\rdifficulty{1}%Difficulté prise en compte pour les environnements d'exercices
\rdifficulty
\end{minipage}
\end{tcolorbox}


\begin{tcolorbox}[colback=yellow!10!white, title=Exemple d'utilisation pour \texttt{setrdexo}]
\begin{minipage}{0.45\textwidth}
\begin{lstlisting}[breaklines]
\setrdexo{%left skip=1cm,
display exotitle,
exo header = tcolorbox,
%display tags,
skin = bouyachakka,
lower ={box=crep},
display score,
display level,
breakable,
score=\points,
level=\rdifficulty,
overlay={\node[inner sep=0pt,
anchor=west,rotate=90, yshift=0.3cm]%,xshift=-3em], yshift=0.45cm
at (frame.south west) {\thetags[0]} ;}
]%obligatoire}
\end{lstlisting}
\end{minipage}
\hfill
\begin{minipage}{0.45\textwidth}
Setup des options retenues pour le package rdexo
\end{minipage}
\end{tcolorbox}


\begin{tcolorbox}[colback=yellow!10!white, title=Exemple d'utilisation pour \texttt{setrdcrep}]
\begin{minipage}{0.45\textwidth}
\begin{lstlisting}[breaklines]
    \definecolor{monrose}{HTML}{FF1493}
    \setrdcrep{seyes, correction=true, correction color=monrose, correction font = \large\bfseries}
    \setrdcrep{seyes, correction=true, correction color=monrose, correction font = \large\bfseries}
\end{lstlisting}
\end{minipage}
\hfill
\begin{minipage}{0.45\textwidth}
\phantom{a}\\
Setup des options retenues pour le package rdcrep.
\end{minipage}
\end{tcolorbox}

\begin{tcolorbox}[colback=yellow!10!white, title=Exemple d'utilisation pour \texttt{filmt}]
\begin{minipage}{0.45\textwidth}
\begin{lstlisting}[breaklines]
\filmt{%
	\begin{center} \begin{tikzpicture}
        \node at (0, 0) {
            \begin{tabular}{r|l}
                1 & 24 \\
                 &  \\
                 &  \\
                 &  \\
            \end{tabular}
        };
        % Ligne de division verticale
        %\draw[thick] (0, 0.8) -- (0, -1.2);
    \end{tikzpicture}\end{center}
	}{
        $24$ est égal à une fois $24$.
}%
\end{lstlisting}
\end{minipage}
\hfill
\begin{minipage}{0.45\textwidth}
\phantom{a}\\
\filmt{%
	\begin{center} \begin{tikzpicture}
        \node at (0, 0) {
            \begin{tabular}{r|l}
                1 & 24 \\
                 &  \\
                 &  \\
                 &  \\
            \end{tabular}
        };
        % Ligne de division verticale
        %\draw[thick] (0, 0.8) -- (0, -1.2);
    \end{tikzpicture}\end{center}
	}{
        $24$ est égal à une fois $24$.
}%
\end{minipage}
\end{tcolorbox}
