\subsection{Philosophie générale des environnements}

Les environnements de BFcours fonctionnent de façon simple et n'admettent qu'un paramètre optionnel : l'intitulé de l'environnement.\\

Ces environnements agissent sur plusieurs aspects : 
\begin{enumerate}
    \item Mise en page basée sur le package \acc{tcolorbox} très flexible. 
    \item Modification des couleurs : \acc{surlignage} du texte, \acc{tableaux}, \acc{items}
    \item Référencement dans l'index sur un niveau personnalisé. 
\end{enumerate}
\subsection{Environnements standards de bfcours}


\newcommand{\contenuExempleEnv}{
    \begin{minipage}{0.3\linewidth}
        Hello World ! \\
        \begin{enumerate}
            \item This is an \acc{accentued text}
            \item This is a \voc{vocabulary text}
        \end{enumerate}
    \end{minipage}
    \hfill
    \begin{minipage}{0.65\linewidth}
        \begin{tcbtabx}[My table]{X|X|X}{\textwidth}
            One & Two & Three \\\hline\hline
            1000.00 & 2000.00 & 3000.00 \\\hline
            2000.00 & 3000.00 & 4000.00
        \end{tcbtabx}
    \end{minipage}\\
    \begin{center}       
        \begin{tcbtab}[Carrés parfaits à connaître]{c|*{12}{c|}c}%{0.8\textwidth}
            nombre $a$&0&1&2&3&4&5&6&7&8&9&10&11&12\\
            \hline
            $a$ \frquote{au carré}&0&1&4&9&16&25&36&49&64&81&100&121&144\\
        \end{tcbtab}
    \end{center}
}
On définit içi la macro contenant nos contenus de tests permettant de visualiser les effets des environnements de bfcours sur les colorations du document. 
\begin{tcblisting}{
    colback=white,
    colframe=black,
    left=6mm,
    listing options={
        style=tcblatex,
        numbers=left,
        numberstyle=\tiny\color{red!75!black}
    }
}
\renewcommand{\contenuExempleEnv}{
    \begin{minipage}{0.3\linewidth}
        Hello World ! \\
        \begin{enumerate}
            \item This is an \acc{accentued text}
            \item This is a \voc{vocabulary text}
        \end{enumerate}
    \end{minipage}
    \hfill
    \begin{minipage}{0.65\linewidth}
        \begin{tcbtabx}[My table]{X|X|X}{\textwidth}
            One & Two & Three \\\hline\hline
            1000.00 & 2000.00 & 3000.00 \\\hline
            2000.00 & 3000.00 & 4000.00
        \end{tcbtabx}
    \end{minipage}\\
    \begin{center}       
        \begin{tcbtab}[Carrés parfaits à connaître]{c|*{12}{c|}c}%{0.8\textwidth}
            nombre $a$&0&1&2&3&4&5&6&7&8&9&10&11&12\\
            \hline
            $a$ \frquote{au carré}&0&1&4&9&16&25&36&49&64&81&100&121&144\\
        \end{tcbtab}
    \end{center}
}

\tcblower

\contenuExempleEnv
\end{tcblisting}

\newpage
\begin{tcblisting}{
    colback=white,
    colframe=black,
    left=6mm,
    listing options={
        style=tcblatex,
        numbers=left,
        numberstyle=\tiny\color{red!75!black}
    }
}
\begin{Theoreme}[titre]

	\contenuExempleEnv

\end{Theoreme}
\end{tcblisting}
	This is an \acc{accentued text} and this is a \voc{vocabulary text}
%Définitions
\newpage
\begin{lstlisting}
    \begin{Exemple}[titre]

        \contenuExempleEnv
    
    \end{Exemple}
\end{lstlisting}
\begin{Exemple}[titre]

    \contenuExempleEnv

\end{Exemple}

\begin{tcblisting}{
    colback=white,
    colframe=black,
    left=6mm,
    listing options={
        style=tcblatex,
        numbers=left,
        numberstyle=\tiny\color{red!75!black}
    }
}
\begin{Definition}[titre]

	\contenuExempleEnv

\end{Definition}
\end{tcblisting}
%
\newpage

%
\newpage
\begin{tcblisting}{
    colback=white,
    colframe=black,
    left=6mm,
    listing options={
        style=tcblatex,
        numbers=left,
        numberstyle=\tiny\color{red!75!black}
    }
}
\begin{Notation}[titre]

	\contenuExempleEnv

\end{Notation}
\end{tcblisting}
%
\newpage
\begin{tcblisting}{
    colback=white,
    colframe=black,
    left=6mm,
    listing options={
        style=tcblatex,
        numbers=left,
        numberstyle=\tiny\color{red!75!black}
    }
}
\begin{Vocabulaire}[titre]

	\contenuExempleEnv

\end{Vocabulaire}
\end{tcblisting}
%
\newpage
\begin{tcblisting}{
    colback=white,
    colframe=black,
    left=6mm,
    listing options={
        style=tcblatex,
        numbers=left,
        numberstyle=\tiny\color{red!75!black}
    }
}
\begin{Aide}[titre]

	\contenuExempleEnv

\end{Aide}
\end{tcblisting}
%
\newpage
\begin{tcblisting}{
    colback=white,
    colframe=black,
    left=6mm,
    listing options={
        style=tcblatex,
        numbers=left,
        numberstyle=\tiny\color{red!75!black}
    }
}
\begin{Demonstration}[titre]

	\contenuExempleEnv
\end{Demonstration}
\end{tcblisting}
%
\newpage
\begin{tcblisting}{
    colback=white,
    colframe=black,
    left=6mm,
    listing options={
        style=tcblatex,
        numbers=left,
        numberstyle=\tiny\color{red!75!black}
    }
}
\begin{Remarque}[titre]

	\contenuExempleEnv

\end{Remarque}
\end{tcblisting}
%
\newpage
\begin{tcblisting}{
    colback=white,
    colframe=black,
    left=6mm,
    listing options={
        style=tcblatex,
        numbers=left,
        numberstyle=\tiny\color{red!75!black}
    }
}
\begin{Propriete}[titre]

	\contenuExempleEnv

\end{Propriete}
\end{tcblisting}
%
\newpage
\begin{tcblisting}{
    colback=white,
    colframe=black,
    left=6mm,
    listing options={
        style=tcblatex,
        numbers=left,
        numberstyle=\tiny\color{red!75!black}
    }
}
\begin{Activite}[titre]

	\contenuExempleEnv
\end{Activite}
\end{tcblisting}
%
\newpage
\begin{tcblisting}{
    colback=white,
    colframe=black,
    left=6mm,
    listing options={
        style=tcblatex,
        numbers=left,
        numberstyle=\tiny\color{red!75!black}
    }
}
\begin{Methode}[titre]

	\contenuExempleEnv

\end{Methode}
\end{tcblisting}


\subsection{Les exercices}
\begin{tcblisting}{
    colback=white,
    colframe=black,
    left=6mm,
    listing options={
        style=tcblatex,
        numbers=left,
        numberstyle=\tiny\color{red!75!black}
    }
}
\def\rdifficulty{1}
\begin{EXO}{Calculer en respectant les priorités opératoires}{5C12}


    \itempoint{1} Un exercice\\
    \itempoint{2.5} Plusieurs questions mais le total est calculé automatiquement
    
    
    \exocorrection
    
        
    La correction de l'exercice


\end{EXO}
\end{tcblisting}

\begin{tcblisting}{
    colback=white,
    colframe=black,
    left=6mm,
    listing options={
        style=tcblatex,
        numbers=left,
        numberstyle=\tiny\color{red!75!black}
    }
}
\def\rdifficulty{1}
\displaybaremepointstrue % Active par défaut l'affichage des points

\begin{EXOEVAL}{Calculer en respectant les priorités opératoires}{5C12}

\itempoint{1} Un exercice d'évaluation\\
\itempoint{2.5} Plusieurs questions mais le total est calculé automatiquement


\exocorrection

    
La correction de l'exercice évalué

\end{EXOEVAL}

\end{tcblisting}