\documentclass[a4paper,11pt,fleqn]{article}

\usepackage[left=1cm,right=0.5cm,top=0.5cm,bottom=2cm]{geometry}

\usepackage{bfcours}
\usepackage{bfcours-quatrieme}
\usepackage{verbatim}

\def\rdifficulty{1}
\setrdexo{%left skip=1cm,
display exotitle,
exo header = tcolorbox,
%display tags,
skin = bouyachakka,
lower ={box=crep},
display score,
display level,
save lower,
score=\points,
level=\rdifficulty,
overlay={\node[inner sep=0pt,
anchor=west,rotate=90, yshift=0.3cm]%,xshift=-3em], yshift=0.45cm
at (frame.south west) {\thetags[0]} ;}
]%obligatoire
}
\setrdcrep{seyes, correction=true, correction color=monrose, correction font = \large\bfseries}
\newcommand{\tikzinclude}[1]{%
    \stepcounter{tikzfigcounter}%
    \csname tikzfig#1\endcsname
}
\newcommand{\figureLongueurCercle}{
    \definecolor{ttzzqq}{rgb}{0.2,0.6,0.}
\definecolor{qqqqff}{rgb}{0.,0.,1.}
\definecolor{xdxdff}{rgb}{0.49019607843137253,0.49019607843137253,1.}
\definecolor{ffwwqq}{rgb}{1.,0.4,0.}
\definecolor{ududff}{rgb}{0.30196078431372547,0.30196078431372547,1.}
\definecolor{uuuuuu}{rgb}{0.26666666666666666,0.26666666666666666,0.26666666666666666}
\definecolor{yqqqyq}{rgb}{0.5019607843137255,0.,0.5019607843137255}
\begin{tikzpicture}[line cap=round,line join=round,>=triangle 45,x=1.0cm,y=1.0cm]
\clip(-1.3,-2.) rectangle (6.,1.);
\draw [line width=0.8pt,color=yqqqyq] (0.,0.) circle (0.75cm);
\draw [line width=0.8pt,color=yqqqyq] (-1.,-1.5)-- (3.7123889803846897,-1.5);
\draw [line width=0.8pt,dash pattern=on 1pt off 1pt,color=ffwwqq] (-0.75,0.)-- (0.75,0.);
\draw [line width=0.8pt,dash pattern=on 1pt off 1pt,color=ffwwqq] (-0.02306940646219445,0.05382861507845321) -- (-0.02306940646219445,-0.05382861507845321);
\draw [line width=0.8pt,dash pattern=on 1pt off 1pt,color=ffwwqq] (0.023069406462194016,0.05382861507845321) -- (0.023069406462194016,-0.05382861507845321);
\draw [line width=0.8pt,dash pattern=on 1pt off 1pt,color=qqqqff] (0.,0.)-- (0.4867299416577562,0.5706084155476829);
\draw [line width=0.8pt,dash pattern=on 1pt off 1pt,color=qqqqff] (0.1994863125988175,0.32273277731191763) -- (0.2872436290589386,0.2478756382357651);
\draw [line width=0.8pt,dash pattern=on 1pt off 1pt,color=ffwwqq] (-1.,-1.2)-- (0.5,-1.2);
\draw [line width=0.8pt,dash pattern=on 1pt off 1pt,color=ffwwqq] (-0.2730694064621942,-1.1461713849215465) -- (-0.2730694064621942,-1.253828615078453);
\draw [line width=0.8pt,dash pattern=on 1pt off 1pt,color=ffwwqq] (-0.2269305935378057,-1.1461713849215465) -- (-0.2269305935378057,-1.253828615078453);
\draw [line width=0.8pt,dash pattern=on 1pt off 1pt,color=ffwwqq] (0.5,-1.2)-- (2.,-1.2);
\draw [line width=0.8pt,dash pattern=on 1pt off 1pt,color=ffwwqq] (1.2269305935378059,-1.1461713849215465) -- (1.2269305935378059,-1.253828615078453);
\draw [line width=0.8pt,dash pattern=on 1pt off 1pt,color=ffwwqq] (1.2730694064621944,-1.1461713849215465) -- (1.2730694064621944,-1.253828615078453);
\draw [line width=0.8pt,dash pattern=on 1pt off 1pt,color=ffwwqq] (2.,-1.2)-- (3.5,-1.2);
\draw [line width=0.8pt,dash pattern=on 1pt off 1pt,color=ffwwqq] (2.726930593537806,-1.1461713849215465) -- (2.726930593537806,-1.253828615078453);
\draw [line width=0.8pt,dash pattern=on 1pt off 1pt,color=ffwwqq] (2.773069406462194,-1.1461713849215465) -- (2.773069406462194,-1.253828615078453);
\draw [line width=0.8pt,dash pattern=on 1pt off 1pt,color=ttzzqq] (3.7123889803846897,-1.)-- (3.7123889803846892,-1.6);
\draw [line width=0.8pt,dash pattern=on 1pt off 1pt,color=ttzzqq] (3.5,-1.)-- (3.5,-1.6);
\draw (1.0529201296198805,0.538595212100376) node[anchor=north west] {$r = 0.75$};
\draw (1.0606099317739452,0.1387255000890093) node[anchor=north west] {$c = {\color{blue}r} \times {\color{red}\pi} \approx 4.71$};
\draw (0.7607076477654201,0.8615669025710952) node[anchor=north west] {$\textbf{Circonférence du cercle} \mathcal{C} :$};
\draw (-0.738803772277205,0.8615669025710952) node[anchor=north west,xshift=-0.5cm] {$\mathcal{C}$};
\begin{scriptsize}
\draw[color=yqqqyq] (1.2451651834714994,-1.6952923136554323) node {$4.71$};
\draw [color=uuuuuu] (0.75,0.)-- ++(-3.0pt,0 pt) -- ++(6.0pt,0 pt) ++(-3.0pt,-3.0pt) -- ++(0 pt,6.0pt);
\draw [color=ududff] (-1.,-1.2)-- ++(-3.0pt,0 pt) -- ++(6.0pt,0 pt) ++(-3.0pt,-3.0pt) -- ++(0 pt,6.0pt);
\draw[color=ffwwqq] (0.030176443129269655,-0.1957808936128071) node {$1.5$};
\draw [color=xdxdff] (0.4867299416577562,0.5706084155476829)-- ++(-3.0pt,0 pt) -- ++(6.0pt,0 pt) ++(-3.0pt,-3.0pt) -- ++(0 pt,6.0pt);
%\draw[color=qqqqff] (0.41466655083250686,0.20408881839855963) node {$0.75$};
\draw [color=uuuuuu] (0.5,-1.2)-- ++(-3.0pt,0 pt) -- ++(6.0pt,0 pt) ++(-3.0pt,-3.0pt) -- ++(0 pt,6.0pt);
\draw[color=ffwwqq] (0.09938466251585235,-1.1031775477924468) node {$1.5$};
\draw [color=uuuuuu] (2.,-1.2)-- ++(-3.0pt,0 pt) -- ++(6.0pt,0 pt) ++(-3.0pt,-3.0pt) -- ++(0 pt,6.0pt);
\draw[color=ffwwqq] (1.4297204351690531,-1.041659130559929) node {$1.5$};
\draw [color=uuuuuu] (3.5,-1.2)-- ++(-3.0pt,0 pt) -- ++(6.0pt,0 pt) ++(-3.0pt,-3.0pt) -- ++(0 pt,6.0pt);
\draw[color=ffwwqq] (2.9446114595198076,-1.0493489327139938) node {$1.5$};
\end{scriptsize}
\end{tikzpicture}
}

\newcommand{\figureAireCarre}{
    \definecolor{ffwwqq}{rgb}{1.,0.4,0.}
    \definecolor{zzttqq}{rgb}{0.6,0.2,0.}
    \begin{tikzpicture}[line cap=round,line join=round,>=triangle 45,x=1.0cm,y=1.0cm]
    \clip(-0.5,-0.5) rectangle (6.5,3.);
    \fill[line width=0.8pt,color=zzttqq,fill=zzttqq,fill opacity=0.10000000149011612] (0.,0.) -- (2.,0.) -- (2.,2.) -- (0.,2.) -- cycle;
    \draw [line width=0.8pt,color=zzttqq] (0.,0.)-- (2.,0.);
    \draw [line width=0.8pt,color=zzttqq] (2.,0.)-- (2.,2.);
    \draw [line width=0.8pt,color=zzttqq] (2.,2.)-- (0.,2.);
    \draw [line width=0.8pt,color=zzttqq] (0.,2.)-- (0.,0.);
    \draw [line width=0.8pt,dash pattern=on 1pt off 1pt,color=ffwwqq] (0.,2.1990485449122428)-- (2.,2.2);
    \draw (2.3265445321089646,2.0241005763481263) node[anchor=north west] {$\mathcal{A}_{\text{Carré}} = c \times c $};
    \draw (2.3357854785212573,1.515848523672014) node[anchor=north west] {$\mathcal{A}_{\text{Carré}} = 2 \times 2 = 4 $};
    \begin{scriptsize}
    \draw[color=ffwwqq] (0.9866073023264862,2.4260817452828696) node {$2$};
    \end{scriptsize}
    \end{tikzpicture}
}

\newcommand{\solEquation}{
    \begin{center}
        \tdplotsetmaincoords{60}{120}
        \begin{tikzpicture}[tdplot_main_coords, scale=0.8]
            % Paramètres de vue en 3D
            
            %\begin{tdplotpicture}[tdplot main coords]
        
                % Axes en 3D
                \draw[thick,->] (0,0,0) -- (4,0,0) node[anchor=north east]{$x$};
                \draw[thick,->] (0,0,0) -- (0,4,0) node[anchor=north west]{$y$};
                \draw[thick,->] (0,0,0) -- (0,0,4) node[anchor=south]{$z$};
        
                % Points pour le plan (2x + y - z = 3)
                \coordinate (A) at (0,3.5,0);   % Point d'intersection avec y
                \coordinate (B) at (2,1.5,0); % Point d'intersection avec x
                \coordinate (C) at (0,0,3);  % Point d'intersection avec z
        
                % Tracer le plan en utilisant les points
                \filldraw[fill=blue!20,opacity=0.5] (A) -- (B) -- (C) -- cycle;
        
                % Lignes principales du plan
                \draw[thick, blue] (A) -- (B) -- (C) -- cycle;
        
                % Hachures pour la région de solutions (en dessous du plan)
                \begin{scope}
                    \clip (A) -- (B) -- (C) -- cycle;
                    \foreach \z in {-3,-2.5,...,0} {
                        \draw[dashed, blue!60,opacity=0.7] (0,3,\z) -- (1.5,0,\z);
                    }
                \end{scope}
        
                % Étiquettes des points
                \node at (A) [anchor=south west] {$(0;3{,}5;0)$};
                \node at (B) [anchor=north] {$(2;1{,}5;0)$};
                \node at (C) [anchor=east] {$(0;0;3)$};
        
        \end{tikzpicture}
    \end{center}
}

\hypersetup{
    pdfauthor={R.Deschamps},
    pdfsubject={},
    pdfkeywords={},
    pdfproducer={LuaLaTeX},
    pdfcreator={Boum Factory}
}
% Activer ou désactiver l'affichage des boîtes
%\displayitempointsfalse % Ne pas afficher les boîtes
\displayitempointstrue % Afficher les boîtes
\usepackage{listings} % Pour afficher le code
\usepackage{tcolorbox} % Pour créer des boîtes de rendu
%\tcbuselibrary{listingsutf8} % Support UTF-8 dans les listings
\lstdefinestyle{tcblatex}{
    language=[LaTeX]TeX,
    basicstyle=\ttfamily\small,
    numbers=left,
    numberstyle=\tiny,
    stepnumber=1,
    numbersep=10pt,
    showstringspaces=false,
    breaklines=true,
    frame=single,
    keepspaces=true,
    columns=flexible
}
%\tcbuselibrary{minted}
\tcbuselibrary{listings}
\tcbset{listing engine=listings}
% Nouvelle boîte de documentation avec code/rendu

\begin{document}

\setcounter{pagecounter}{0}
\setcounter{ExoMA}{0}
\setcounter{prof}{0}

\def\points{1}
\def\rdifficulty{1}
\chapitre[
    BF% : $\mathbf{6^{\text{ème}}}$,$\mathbf{5^{\text{ème}}}$,$\mathbf{4^{\text{ème}}}$,$\mathbf{3^{\text{ème}}}$,$\mathbf{2^{\text{nde}}}$,$\mathbf{1^{\text{ère}}}$,$\mathbf{T^{\text{Le}}}$,
    ]{
    Environnements% : ,Equations
    }{
    R.Deschamps% : Collège,Lycée
    }{
    % : Amadis Jamyn,Eugène Belgrand
    }{
    % : ,\tableauPresenteEvalSixieme{}{10},\tableofcontents
    }{
    Documentation% : Exercices
    }

\setrdcrep{seyes, correction=true, correction color=monrose, correction font = \large\bfseries}
\tableofcontents
\phantom{\voc{nobug}}

\section{Comment fonctionne \LaTeX ?}

\LaTeX est un langage de programmation développé par \vocref{https://fr.wikipedia.org/wiki/Donald_Knuth}{Donald knuth} dans les années 1980 qui permet de construire des documents pdf en gérant la \textbf{structure} du document de façon semi-automatique.\\

Le principe est simple : on crée des commandes, des environnements, des packages qui permettent d'obtenir de nombreuses fonctionnalités.\\

Le logiciel est gratuit et \textbf{open source} et dispose d'une \textbf{largre communauté} notamment scientifique. \\

N'importe quel document texte comportant l'extension \frquote{.tex} peut être considéré comme un fichier \LaTeX. \\

\subsection{Quelques liens}

\textbf{Les essentiels :}\\
\begin{itemize}[label = \bccrayon]
	\item Pour \textbf{télécharger un compilateur} \LaTeX : \vocref{https://miktex.org/download}{MikTeX}
	\item Le forum \LaTeX par excellence : \vocref{https://tex.stackexchange.com/}{LaTeX stack exchange}
	\item Le repo principal des packages en ligne. C'est là que l'on trouve la plupart des \textbf{documentations} : \vocref{https://ctan.org/}{CTAN}\\
		Je l'utilise surtout depuis un moteur de recherche externe : \frquote{<nom\_du\_package> CTAN}
	\item Pour télécharger mon package \frquote{bfcours} et ses à-côtés : \vocref{https://github.com/Romain1099/BFCours.git}{BFCours}
\end{itemize}

\textbf{Les non-moins importants :}\\

\begin{itemize}[label=\bccrayon]

	\item Document d'explications générales en \LaTeX : \vocref{https://tuteurs.ens.fr/logiciels/latex/}{site d'archives des tuteurs de l'ENS}.\\On n'a pas fait plus concis et complet pour prendre \LaTeX en main.
	\item Toutes les documentations de mon repo github : \\
		\begin{itemize}[label=\faPen]
			\item tcolorbox pour toutes les \textbf{boites} 
			\item Tikz-euclide pour les constructions géométriques. Il est tout de même bon de noter que GeoGebra permet l'export d'une figure comme code LaTeX - tikz.
			\item TikZ pour l'impatient $\longrightarrow$ TikZ est le module de \textbf{dessin} de \LaTeX par excellence. Des bases sont à \textbf{maîtriser} pour bien progresser en \LaTeX. 
			\item rdexo et rdcrep - packages pour l'enseignement ou la présentation de ressources. 
		\end{itemize}
	
\end{itemize}	

\section{Comment installer \LaTeX ?}

\subsection{Installation}

Aller sur la page de téléchargement de MikTeX et choisir la version \textbf{adaptée à votre système d'exploitation}.\\
\textbf{Cocher} l'option \textbf{installer les packages à la volée} ( on-the-fly ) pour permettre plus de souplesse dans les premières compilations.\\
\textbf{Décocher} l'option d'installation pour tous les utilisateurs. Cela rend plus simple l'utilisation de la console MikTeX.


\subsection{Setup du répertoire des packages locaux}

Suivre les étapes suivantes \textbf{une seule fois} :
\begin{enumerate}
	\item Coller le dossier \textbf{localtexmf} récupéré sur mon repo github \textbf{n'importe ou sur votre machine}. L'essentiel est qu'il reste à cet emplacement. \\
	\item Copier le chemin d'accès de ce dossier. \\
	\item Ouvrir la \frquote{console MikTeX} et aller au menu \frquote{Settings}.
	\item Aller dans l'onglet \frquote{Directories}.
	\item Appuyer sur le bouteon \frquote{+} et \textbf{coller} le chemin d'accès au dossier \textbf{localtexmf}.
	\item Confirmer les changements et quitter la console. 
\end{enumerate}

Parfait : vous pouvez utiliser le package bfcours et les packages de Régis Deleuze ( suite rd ) dans vos documents. 

\subsection{Première compilation}

Ouvrir le document \frquote{new\_document.tex}. \\
Dans l'application \textbf{TeXWork} qui s'ouvre, on accède au code source de la page qu'il faut \textbf{compiler}.\\

Pour cela, avec le package \textbf{bfcours} il est nécessaire d'utiliser le compilateur \frquote{LuaLaTeX} qui permet 
d'accéder à des programmes secondaires ( code en langage lua ) qui étendent les possibilité du logiciel.\\
Sélectionner le compilateur \frquote{LuaLaTeX} dans la barre de sélection en haut à gauche de l'écran. \\
Compiler ensuite votre premier document en cliquant sur le \textbf{triangle vert} en haut à gauche de l'écran. \\


\bcattention Les packages utilisés ne sont pas disponibles sur votre machine :
\begin{multicols}{2}
	Il suffit de cocher \frquote{autoriser le téléchargement on-the-fly} lors de l'installation, pour que LaTeX télécharge lui-même les paquets.\\

	\columnbreak

	Parfois, il faut les télécharger manuellement.\\
	Pour cela, il suffit d'ouvrir la console \frquote{MikTeX} via le sélecteur de programme de l'ordinateur.\\
	Cliquer sur l'onglet \frquote{Packages} du menu de gauche.\\
	Pour chaque package manquant, le renseigner dans la barre de recherche puis le sélectionner dans la liste et l'installer via \frquote{clic droit +Install}.
\end{multicols}

Cela peut être fastidieux, mais une fois que c'est fait, il est possible de se concentrer sur l'essentiel : le contenu.

\section{Présentation générale de BFcours}
Le package BFcours est constitué de tous les outils que j'utilise au quotidien pour développer mes cours. \\

L'objectif étant de modifier l'utilisation standard de LaTeX afin d'apporter une surcouche de style. \\

La philosophie générale consiste à produire et écrire du code LaTeX simple permettant de se concentrer sur le contenu. \\

Les commandes et environnements ainsi que leurs particularité sont décrites dans la suite de ce document. \\
Le lecteur pourra trouver de nombreux exemples d'utilisation du package dans la partie Exemples. \\
On pourra également se diriger vers le repo contenant les cours pour se fournir en exemples d'utilisations. 


\subsection{Mes habitudes}

Pour produire des documents de façon aisée, il est nécessaire de réfléchir à une structure des fichiers. \\

On pourra s'inspirer du fonctionnement de mes \frquote{notes de stages} en herboristerie : \vocref{https://github.com/Romain1099/Herboristerie-2024.git}{Stage Arsimed 2024}\\

L'IDE \textbf{VScode} offre de nombreuses extensions dédiées à \LaTeX à explorer. De simples modules d'autocomplétion permettront une bien meilleure expérience. \\

Extension \textbf{PDF Viewer} qui permet de lire des documents pdf directement dans VScode.\\
Il doit y avoir un module permettant de compiler directement des documents latex dans VScode mais je n'ai plus la référence en tete. 


\section{Présentation des environnements de BFcours}
\subsection{Philosophie générale des environnements}

Les environnements de BFcours fonctionnent de façon simple et n'admettent qu'un paramètre optionnel : l'intitulé de l'environnement.\\

Ces environnements agissent sur plusieurs aspects : 
\begin{enumerate}
    \item Mise en page basée sur le package \acc{tcolorbox} très flexible. 
    \item Modification des couleurs : \acc{surlignage} du texte, \acc{tableaux}, \acc{items}
    \item Référencement dans l'index sur un niveau personnalisé. 
\end{enumerate}
\subsection{Environnements standards de bfcours}


\newcommand{\contenuExempleEnv}{
    \begin{minipage}{0.3\linewidth}
        Hello World ! \\
        \begin{enumerate}
            \item This is an \acc{accentued text}
            \item This is a \voc{vocabulary text}
        \end{enumerate}
    \end{minipage}
    \hfill
    \begin{minipage}{0.65\linewidth}
        \begin{tcbtabx}[My table]{X|X|X}{\textwidth}
            One & Two & Three \\\hline\hline
            1000.00 & 2000.00 & 3000.00 \\\hline
            2000.00 & 3000.00 & 4000.00
        \end{tcbtabx}
    \end{minipage}\\
    \begin{center}       
        \begin{tcbtab}[Carrés parfaits à connaître]{c|*{12}{c|}c}%{0.8\textwidth}
            nombre $a$&0&1&2&3&4&5&6&7&8&9&10&11&12\\
            \hline
            $a$ \frquote{au carré}&0&1&4&9&16&25&36&49&64&81&100&121&144\\
        \end{tcbtab}
    \end{center}
}
On définit içi la macro contenant nos contenus de tests permettant de visualiser les effets des environnements de bfcours sur les colorations du document. 
\begin{tcblisting}{
    colback=white,
    colframe=black,
    left=6mm,
    listing options={
        style=tcblatex,
        numbers=left,
        numberstyle=\tiny\color{red!75!black}
    }
}
\renewcommand{\contenuExempleEnv}{
    \begin{minipage}{0.3\linewidth}
        Hello World ! \\
        \begin{enumerate}
            \item This is an \acc{accentued text}
            \item This is a \voc{vocabulary text}
        \end{enumerate}
    \end{minipage}
    \hfill
    \begin{minipage}{0.65\linewidth}
        \begin{tcbtabx}[My table]{X|X|X}{\textwidth}
            One & Two & Three \\\hline\hline
            1000.00 & 2000.00 & 3000.00 \\\hline
            2000.00 & 3000.00 & 4000.00
        \end{tcbtabx}
    \end{minipage}\\
    \begin{center}       
        \begin{tcbtab}[Carrés parfaits à connaître]{c|*{12}{c|}c}%{0.8\textwidth}
            nombre $a$&0&1&2&3&4&5&6&7&8&9&10&11&12\\
            \hline
            $a$ \frquote{au carré}&0&1&4&9&16&25&36&49&64&81&100&121&144\\
        \end{tcbtab}
    \end{center}
}

\tcblower

\contenuExempleEnv
\end{tcblisting}

\newpage
\begin{tcblisting}{
    colback=white,
    colframe=black,
    left=6mm,
    listing options={
        style=tcblatex,
        numbers=left,
        numberstyle=\tiny\color{red!75!black}
    }
}
\begin{Theoreme}[titre]

	\contenuExempleEnv

\end{Theoreme}
\end{tcblisting}
	This is an \acc{accentued text} and this is a \voc{vocabulary text}
%Définitions
\newpage
\begin{lstlisting}
    \begin{Exemple}[titre]

        \contenuExempleEnv
    
    \end{Exemple}
\end{lstlisting}
\begin{Exemple}[titre]

    \contenuExempleEnv

\end{Exemple}

\begin{tcblisting}{
    colback=white,
    colframe=black,
    left=6mm,
    listing options={
        style=tcblatex,
        numbers=left,
        numberstyle=\tiny\color{red!75!black}
    }
}
\begin{Definition}[titre]

	\contenuExempleEnv

\end{Definition}
\end{tcblisting}
%
\newpage

%
\newpage
\begin{tcblisting}{
    colback=white,
    colframe=black,
    left=6mm,
    listing options={
        style=tcblatex,
        numbers=left,
        numberstyle=\tiny\color{red!75!black}
    }
}
\begin{Notation}[titre]

	\contenuExempleEnv

\end{Notation}
\end{tcblisting}
%
\newpage
\begin{tcblisting}{
    colback=white,
    colframe=black,
    left=6mm,
    listing options={
        style=tcblatex,
        numbers=left,
        numberstyle=\tiny\color{red!75!black}
    }
}
\begin{Vocabulaire}[titre]

	\contenuExempleEnv

\end{Vocabulaire}
\end{tcblisting}
%
\newpage
\begin{tcblisting}{
    colback=white,
    colframe=black,
    left=6mm,
    listing options={
        style=tcblatex,
        numbers=left,
        numberstyle=\tiny\color{red!75!black}
    }
}
\begin{Aide}[titre]

	\contenuExempleEnv

\end{Aide}
\end{tcblisting}
%
\newpage
\begin{tcblisting}{
    colback=white,
    colframe=black,
    left=6mm,
    listing options={
        style=tcblatex,
        numbers=left,
        numberstyle=\tiny\color{red!75!black}
    }
}
\begin{Demonstration}[titre]

	\contenuExempleEnv
\end{Demonstration}
\end{tcblisting}
%
\newpage
\begin{tcblisting}{
    colback=white,
    colframe=black,
    left=6mm,
    listing options={
        style=tcblatex,
        numbers=left,
        numberstyle=\tiny\color{red!75!black}
    }
}
\begin{Remarque}[titre]

	\contenuExempleEnv

\end{Remarque}
\end{tcblisting}
%
\newpage
\begin{tcblisting}{
    colback=white,
    colframe=black,
    left=6mm,
    listing options={
        style=tcblatex,
        numbers=left,
        numberstyle=\tiny\color{red!75!black}
    }
}
\begin{Propriete}[titre]

	\contenuExempleEnv

\end{Propriete}
\end{tcblisting}
%
\newpage
\begin{tcblisting}{
    colback=white,
    colframe=black,
    left=6mm,
    listing options={
        style=tcblatex,
        numbers=left,
        numberstyle=\tiny\color{red!75!black}
    }
}
\begin{Activite}[titre]

	\contenuExempleEnv
\end{Activite}
\end{tcblisting}
%
\newpage
\begin{tcblisting}{
    colback=white,
    colframe=black,
    left=6mm,
    listing options={
        style=tcblatex,
        numbers=left,
        numberstyle=\tiny\color{red!75!black}
    }
}
\begin{Methode}[titre]

	\contenuExempleEnv

\end{Methode}
\end{tcblisting}


\subsection{Les exercices}
\begin{tcblisting}{
    colback=white,
    colframe=black,
    left=6mm,
    listing options={
        style=tcblatex,
        numbers=left,
        numberstyle=\tiny\color{red!75!black}
    }
}
\def\rdifficulty{1}
\begin{EXO}{Calculer en respectant les priorités opératoires}{5C12}


    \itempoint{1} Un exercice\\
    \itempoint{2.5} Plusieurs questions mais le total est calculé automatiquement
    
    
    \exocorrection
    
        
    La correction de l'exercice


\end{EXO}
\end{tcblisting}

\begin{tcblisting}{
    colback=white,
    colframe=black,
    left=6mm,
    listing options={
        style=tcblatex,
        numbers=left,
        numberstyle=\tiny\color{red!75!black}
    }
}
\def\rdifficulty{1}
\displaybaremepointstrue % Active par défaut l'affichage des points

\begin{EXOEVAL}{Calculer en respectant les priorités opératoires}{5C12}

\itempoint{1} Un exercice d'évaluation\\
\itempoint{2.5} Plusieurs questions mais le total est calculé automatiquement


\exocorrection

    
La correction de l'exercice évalué

\end{EXOEVAL}

\end{tcblisting}% : On peut utiliser le mode maths de plusieurs manières : 
\begin{tcolorbox}[blank]
\begin{tcbenumerate}[2]
    \tcbitem Inline via \textcolor{red}{\$ contenu maths \$} : $\left(\text{\Large{E}}\right)~~5x + 3 = 2^\frac{3}{x}$
    \tcbitem Inline via le mode \acc{display} \textcolor{red}{$\backslash$( contenu maths $\backslash$)} : \(\left(\text{\Large{E}}\right)~~5x + 3 = 2^\frac{3}{x}\)
    \tcbitem En valeur via le mode \acc{centré} \textcolor{red}{$\backslash$[ contenu maths $\backslash$]} : \[\left(\text{\Large{E}}\right)~~5x + 3 = 2^\frac{3}{x}\]
    \tcbitem En mode \acc{align} ( énuméré ) ou \acc{align*} ( non énuméré ) :  
    
    \showenv{align}[][contenu maths]
    \begin{align}
        \left(\text{\Large{E}}\right)~~5x + 3 &= 2^\frac{3}{x}\\
        &= e^{\frac{3}{x}\ln(2)}
    \end{align}
\end{tcbenumerate}

Dans le mode mathématiques, si l'on veut écrire du texte, il faut l'appeler \acc{dans la commande text} : 

\showcmd{text}[\{Texte à écrire\}].


Le mode mathématique peut donc aussi être utilisé pour formatter du texte : 

$6^{\text{ème}}$
\end{tcolorbox},
\section{Présentation des tableaux dans BFcours}

\subsection{Mode manuel}

\begin{tcblisting}{
    colback=white,
    colframe=black,
    left=6mm,
    listing options={
        style=tcblatex,
        numbers=left,
        numberstyle=\tiny\color{red!75!black}
    }
}
\tcbox[TableauBox,title=Carrés parfaits à connaître]{%
\arrayrulecolor{blue!50!black}\renewcommand{\arraystretch}{1.2}%
    \begin{tabular}{|*{14}{c|}}
    \hline
    nombre $a$&0&1&2&3&4&5&6&7&8&9&10&11&12\\
    \hline
    $a$ \frquote{au carré}&0&1&4&9&16&25&36&49&64&81&100&121&144\\
    \hline
    \end{tabular}
}
\end{tcblisting}
\subsection{Mode standard sans titre}
\begin{tcblisting}{
    colback=white,
    colframe=black,
    left=6mm,
    listing options={
        style=tcblatex,
        numbers=left,
        numberstyle=\tiny\color{red!75!black}
    }
}
\begin{tcbtab}[Carrés parfaits à connaître]{c|*{12}{c|}c}%{0.8\textwidth}
    nombre $a$&0&1&2&3&4&5&6&7&8&9&10&11&12\\
    \hline
    $a$ \frquote{au carré}&0&1&4&9&16&25&36&49&64&81&100&121&144\\
\end{tcbtab}
\end{tcblisting}
\subsection{Mode standard avec titre}
\begin{tcblisting}{
    colback=white,
    colframe=black,
    left=6mm,
    listing options={
        style=tcblatex,
        numbers=left,
        numberstyle=\tiny\color{red!75!black}
    }
}
\begin{tcbtab}[Carrés parfaits à connaître]{c|*{12}{c|}c}%{0.8\textwidth}
    nombre $a$&0&1&2&3&4&5&6&7&8&9&10&11&12\\
    \hline
    $a$ \frquote{au carré}&0&1&4&9&16&25&36&49&64&81&100&121&144\\
\end{tcbtab}
\end{tcblisting}


\subsection{Mode tabularx avec style clair}
\begin{tcblisting}{
    colback=white,
    colframe=black,
    left=6mm,
    listing options={
        style=tcblatex,
        numbers=left,
        numberstyle=\tiny\color{red!75!black}
    }
}
\begin{tcbtabx}[My table]{X|X|X}{0.5\textwidth}
    One & Two & Three \\\hline\hline
    1000.00 & 2000.00 & 3000.00 \\\hline
    2000.00 & 3000.00 & 4000.00
\end{tcbtabx}
\end{tcblisting}
\newpage
\setcounter{pagecounter}{0}
\setcounter{ExoMA}{0}
\setcounter{prof}{1}

\chapitre[
    BF%
    ]{
    Environnements%
    }{
    R.Deschamps%
    }{
    %
    }{
    %
    }{
    Commandes% : Solution,
    }

\tcbset{
    rdexo/default/.cd,correction style/.style={
        before upper=%
        \textbf{\thelabel~
        \thecorrectionnum~:~}
    }
}
\section{Présentation des commandes utilisables dans BFcours}

\subsection{Commandes d'accentuation du texte}

La package BFcours propose deux commandes d'accentuation du texte : 
\begin{itemize}[label = $\bullet$]
    \item \verb+\acc+ $\rightarrow$ met en gras et colore un mot
    \item \verb+\voc+ $\rightarrow$ comme acc, mais place le mot à l'index
\end{itemize}
Les deux commandes obéissent à la couleur \verb+\currentAccentColor+
\begin{tcolorbox}[colback=yellow!10!white, title=Exemple d'utilisation pour \texttt{voc et acc}]
    \begin{minipage}{0.45\textwidth}
    \begin{lstlisting}[breaklines]
    \voc{mot de vocabulaire}\\
    \acc{Un autre mot}
    \end{lstlisting}
    \end{minipage}
    \hfill
    \begin{minipage}{0.45\textwidth}
    \phantom{a}\\
    \voc{mot de vocabulaire}\\
    \acc{Un autre mot}
    \end{minipage}
\end{tcolorbox}

\begin{tcolorbox}[colback=yellow!10!white, title=Exemple d'utilisation pour \texttt{printvocindex}]
    \begin{minipage}{0.45\textwidth}
        \begin{lstlisting}[breaklines]
        \printvocindex % Affiche les mots de vocabulaire dans une boite.
        \end{lstlisting}
        \end{minipage}
        \hfill
        \begin{minipage}{0.45\textwidth}
        \phantom{a}\\
        \printvocindex
    \end{minipage}
\end{tcolorbox}   

\begin{tcolorbox}[colback=yellow!10!white, title=Exemple d'utilisation pour \texttt{newcommand}]
\begin{minipage}{0.45\textwidth}
\begin{lstlisting}[breaklines]
\pascalc
\end{lstlisting}
\end{minipage}
\hfill
\begin{minipage}{0.45\textwidth}
\phantom{a}\\
\pascalc
\end{minipage}
\end{tcolorbox}


\subsection{Commandes relatives au package de Régis Deleuze}

Les packages \acc{rdexo} et \acc{rdcrep} fournissent de nombreuses fonctionnalités. 
\begin{tcolorbox}[colback=yellow!10!white, title=Exemple d'utilisation pour \texttt{rdexo}]
    \begin{minipage}{0.45\textwidth}
    \begin{lstlisting}[breaklines]
        \def\points{2} % Commande permettant d'attribuer les points aux exercices
        \def\rdifficulty{2} % Permet de définir la difficulté estimée d'un exercice ( par défaut 1 )

        \begin{EXO}{Titre}{Code}
            Ceci est un exercice.
        \end{EXO}
    \end{lstlisting}
    \end{minipage}
    \hfill
    \begin{minipage}{0.45\textwidth}
    \phantom{a}\\
    \def\points{2} % Commande permettant d'attribuer les points aux exercices
        \def\rdifficulty{2} % Permet de définir la difficulté estimée d'un exercice ( par défaut 1 )

        \begin{EXO}{Titre}{Code}
            Ceci est un exercice.
        \end{EXO}
    \end{minipage}
    \end{tcolorbox}

\begin{tcolorbox}[colback=yellow!10!white, title=Exemple d'utilisation pour \texttt{points}]
\begin{minipage}{0.45\textwidth}
\begin{lstlisting}[breaklines]
    \def\points{2} % Commande permettant d'attribuer les points aux exercices
    \points
\end{lstlisting}
\end{minipage}
\hfill
\begin{minipage}{0.45\textwidth}
\phantom{a}\\
\def\points{2} % Commande permettant d'attribuer les points aux exercices
    \points
\end{minipage}
\end{tcolorbox}




\begin{tcolorbox}[colback=yellow!10!white, title=Exemple d'utilisation pour \texttt{tccrep}]
\begin{minipage}{0.45\textwidth}
\begin{lstlisting}[breaklines]
    \setrdcrep{seyes, correction=false, correction color=monrose, correction font = \large\bfseries}

    \tccrep[seyes=false]{1.2cm}{123}
    \tccrep[seyes=false]{1.2cm}{123}

    ----

    \setrdcrep{seyes, correction=true, correction color=monrose, correction font = \large\bfseries}
    \tccrep[seyes=false]{1.2cm}{123}
    \tccrep[seyes=false]{1.2cm}{123}
\end{lstlisting}
\end{minipage}
\hfill
\begin{minipage}{0.45\textwidth}
\phantom{a}\\
    \setrdcrep{seyes, correction=false, correction color=monrose, correction font = \large\bfseries}
    \tccrep[seyes=false]{1.2cm}{123}
    \tccrep[seyes=true]{1.2cm}{123}\\

    ----\\

    \setrdcrep{seyes, correction=true, correction color=monrose, correction font = \large\bfseries}
    \tccrep[seyes=false]{1.2cm}{123}
    \tccrep[seyes=true]{1.2cm}{123}
\end{minipage}
\end{tcolorbox}
\begin{tcolorbox}[colback=yellow!10!white, title=Exemple d'utilisation pour l'environnement\texttt{crep}]
    \begin{minipage}{0.45\textwidth}
    \begin{lstlisting}[breaklines]
        \setrdcrep{seyes, correction=false, correction color=monrose, correction font = \large\bfseries}
    
        \begin{crep}
            Ceci est une réponse
        \end{crep}
    
        ----
    
        \setrdcrep{seyes, correction=true, correction color=monrose, correction font = \large\bfseries}
        
        \begin{crep}
            Ceci est une réponse
        \end{crep}
    \end{lstlisting}
    \end{minipage}
    \hfill
    \begin{minipage}{0.45\textwidth}
    \phantom{a}\\
    \setrdcrep{seyes, correction=false, correction color=monrose, correction font = \large\bfseries}
    
    \begin{crep}
        Ceci est une réponse
    \end{crep}

    ----

    \setrdcrep{seyes, correction=true, correction color=monrose, correction font = \large\bfseries}
    
    \begin{crep}
        Ceci est une réponse
    \end{crep}
    \end{minipage}
    \end{tcolorbox}

\begin{tcolorbox}[colback=yellow!10!white, title=Exemple d'utilisation pour \texttt{mysquare}]
\begin{minipage}{0.45\textwidth}
\begin{lstlisting}[breaklines]
\mysquare
\end{lstlisting}
\end{minipage}
\hfill
\begin{minipage}{0.45\textwidth}
\phantom{a}\\
\mysquare
\end{minipage}
\end{tcolorbox}




\begin{tcolorbox}[colback=yellow!10!white, title=Exemple d'utilisation pour \texttt{tableaucompetence}]
\begin{minipage}{0.32\textwidth}
\begin{lstlisting}[breaklines]
\tableaucompetence{}
\end{lstlisting}
\end{minipage}
\hfill
\begin{minipage}{0.65\textwidth}
\phantom{a}\\
\tableaucompetence{}
\end{minipage}
\end{tcolorbox}



\begin{tcolorbox}[colback=yellow!10!white, title=Exemple d'utilisation pour \texttt{competence}]
\begin{minipage}{0.32\textwidth}
\begin{lstlisting}[breaklines]
    \tableaucompetence{
        \competence{C1-1}
        \competence{C1-1}
    }
\end{lstlisting}
\end{minipage}
\hfill
\begin{minipage}{0.65\textwidth}
\phantom{a}\\
    \tableaucompetence{
        \competence{C1-1}
        \competence{C1-1}
    }
\end{minipage}
\end{tcolorbox}

\subsection{Commandes relatives aux exercices et évaluations}

La commande  \texttt{itempoint} permet le calcul automatique des points d'un exercice en cours. \\
Elle permet en outre d'obtenir le total des points stocké dans un fichier annexe.\\

Il est nécessaire de définir \texttt{\backslash displaybaremepointstrue} \textbf{avant le début du document} afin de profiter de l'affichage des points.\\
Les totaux par exercice et pour le document seront tout de même calculés.
\begin{tcolorbox}[colback=yellow!10!white, title=Exemple d'utilisation pour \texttt{competence}]
\begin{minipage}{0.65\textwidth}
\begin{lstlisting}[breaklines]
    \displaybaremepointsfalse % Active par défaut l'affichage des points
	\itempoint{4}[0.5]aa\\
	\displaybaremepointstrue % Active par défaut l'affichage des points
	\itempoint{2}bb\\
	\itempoint{1.5}[-0.5]cc
\end{lstlisting}
\end{minipage}\\

\end{tcolorbox}
\displaybaremepointsfalse % Active par défaut l'affichage des points
	\itempoint{4}[0.5]aa\\
	\displaybaremepointstrue % Active par défaut l'affichage des points
\itempoint{2}bb\\
\itempoint{1.5}[-0.5]cc

\begin{tcolorbox}[colback=yellow!10!white, title=Exemple d'utilisation pour \texttt{competence}]
\begin{minipage}{0.65\textwidth}
\begin{lstlisting}[breaklines]
	Il y a \getsavedtotalpoints dans ce document.
\end{lstlisting}
\end{minipage}\\
	Il y a \getsavedtotalpoints points dans ce document.
\end{tcolorbox}

\subsection{Commandes d'impression}


\begin{tcolorbox}[colback=yellow!10!white, title=Exemple d'utilisation pour \texttt{imp}]
\begin{minipage}{0.45\textwidth}
\begin{lstlisting}[breaklines]
    \renewcommand{\impressFileName}{commands_doc_bfcours-commands-to-print}
    \immediate\openout\imprimfile=\impressFileName.tex
    \begingroup
    \imp{
        Ceci est un texte à imprimer
        }
    \endgroup
    \immediate\closeout\imprimfile

    \noindent \begin {tikzpicture}[remember picture, overlay]
\draw [dashed] (-1cm,0) -- (\paperwidth ,0);
\node at (\paperwidth -1,0) {\ding {36}};
\end {tikzpicture}\par 
\noindent \begin {tikzpicture}[overlay]
\node [draw,inner sep=3pt, xshift=-0.5cm,yshift=0.18cm] at (0,0) {\faPrint 1};
\end {tikzpicture}
 Ceci est un texte à imprimer 


\end{lstlisting}
\end{minipage}
\hfill
\begin{minipage}{0.45\textwidth}
    \phantom{a}\\
    \renewcommand{\impressFileName}{commands_doc_bfcours-commands-to-print}
    \immediate\openout\imprimfile=\impressFileName.tex
    \begingroup
    \imp{
        Ceci est un texte à imprimer
        }
    \endgroup
    \immediate\closeout\imprimfile

    \noindent \begin {tikzpicture}[remember picture, overlay]
\draw [dashed] (-1cm,0) -- (\paperwidth ,0);
\node at (\paperwidth -1,0) {\ding {36}};
\end {tikzpicture}\par 
\noindent \begin {tikzpicture}[overlay]
\node [draw,inner sep=3pt, xshift=-0.5cm,yshift=0.18cm] at (0,0) {\faPrint 1};
\end {tikzpicture}
 Ceci est un texte à imprimer 


\end{minipage}
\end{tcolorbox}



\begin{tcolorbox}[colback=yellow!10!white, title=Exemple d'utilisation pour \texttt{repfill}]
\begin{minipage}{0.45\textwidth}
\begin{lstlisting}[breaklines]
    \setrdcrep{seyes, correction=false, correction color=monrose, correction font = \large\bfseries}

    \repfill{Une réponse}\\

    ----\\

    \setrdcrep{seyes, correction=true, correction color=monrose, correction font = \large\bfseries}
    \repfill{Une réponse}

\end{lstlisting}
\end{minipage}
\hfill
\begin{minipage}{0.45\textwidth}
\phantom{a}\\
\setrdcrep{seyes, correction=false, correction color=monrose, correction font = \large\bfseries}

    \repfill{Une réponse}\\

    ----\\

    \setrdcrep{seyes, correction=true, correction color=monrose, correction font = \large\bfseries}
    \repfill{Une réponse}
\end{minipage}
\end{tcolorbox}


\begin{tcolorbox}[colback=yellow!10!white, title=Exemple d'utilisation pour \texttt{repsim}]
\begin{minipage}{0.45\textwidth}
\begin{lstlisting}[breaklines]
\repsim{123}
\repsim[6cm]{Longue réponse}
\end{lstlisting}
\end{minipage}
\hfill
\begin{minipage}{0.45\textwidth}
\phantom{a}\\
\repsim{123}
\repsim[6cm]{Longue réponse}
\end{minipage}
\end{tcolorbox}


\begin{tcolorbox}[colback=yellow!10!white, title=Exemple d'utilisation pour \texttt{frquote}]
\begin{minipage}{0.45\textwidth}
\begin{lstlisting}[breaklines]
\frquote{Citons !}
\end{lstlisting}
\end{minipage}
\hfill
\begin{minipage}{0.45\textwidth}
\phantom{a}\\
\frquote{Citons !}
\end{minipage}
\end{tcolorbox}


\begin{tcolorbox}[colback=yellow!10!white, title=Exemple d'utilisation pour \texttt{rdifficulty}]
\begin{minipage}{0.45\textwidth}
\begin{lstlisting}[breaklines]
\def\rdifficulty{1}
\rdifficulty
\end{lstlisting}
\end{minipage}
\hfill
\begin{minipage}{0.45\textwidth}
\phantom{a}\\
\def\rdifficulty{1}%Difficulté prise en compte pour les environnements d'exercices
\rdifficulty
\end{minipage}
\end{tcolorbox}


\begin{tcolorbox}[colback=yellow!10!white, title=Exemple d'utilisation pour \texttt{setrdexo}]
\begin{minipage}{0.45\textwidth}
\begin{lstlisting}[breaklines]
\setrdexo{%left skip=1cm,
display exotitle,
exo header = tcolorbox,
%display tags,
skin = bouyachakka,
lower ={box=crep},
display score,
display level,
breakable,
score=\points,
level=\rdifficulty,
overlay={\node[inner sep=0pt,
anchor=west,rotate=90, yshift=0.3cm]%,xshift=-3em], yshift=0.45cm
at (frame.south west) {\thetags[0]} ;}
]%obligatoire}
\end{lstlisting}
\end{minipage}
\hfill
\begin{minipage}{0.45\textwidth}
Setup des options retenues pour le package rdexo
\end{minipage}
\end{tcolorbox}


\begin{tcolorbox}[colback=yellow!10!white, title=Exemple d'utilisation pour \texttt{setrdcrep}]
\begin{minipage}{0.45\textwidth}
\begin{lstlisting}[breaklines]
    \definecolor{monrose}{HTML}{FF1493}
    \setrdcrep{seyes, correction=true, correction color=monrose, correction font = \large\bfseries}
    \setrdcrep{seyes, correction=true, correction color=monrose, correction font = \large\bfseries}
\end{lstlisting}
\end{minipage}
\hfill
\begin{minipage}{0.45\textwidth}
\phantom{a}\\
Setup des options retenues pour le package rdcrep.
\end{minipage}
\end{tcolorbox}

\begin{tcolorbox}[colback=yellow!10!white, title=Exemple d'utilisation pour \texttt{filmt}]
\begin{minipage}{0.45\textwidth}
\begin{lstlisting}[breaklines]
\filmt{%
	\begin{center} \begin{tikzpicture}
        \node at (0, 0) {
            \begin{tabular}{r|l}
                1 & 24 \\
                 &  \\
                 &  \\
                 &  \\
            \end{tabular}
        };
        % Ligne de division verticale
        %\draw[thick] (0, 0.8) -- (0, -1.2);
    \end{tikzpicture}\end{center}
	}{
        $24$ est égal à une fois $24$.
}%
\end{lstlisting}
\end{minipage}
\hfill
\begin{minipage}{0.45\textwidth}
\phantom{a}\\
\filmt{%
	\begin{center} \begin{tikzpicture}
        \node at (0, 0) {
            \begin{tabular}{r|l}
                1 & 24 \\
                 &  \\
                 &  \\
                 &  \\
            \end{tabular}
        };
        % Ligne de division verticale
        %\draw[thick] (0, 0.8) -- (0, -1.2);
    \end{tikzpicture}\end{center}
	}{
        $24$ est égal à une fois $24$.
}%
\end{minipage}
\end{tcolorbox}
% : ,\input{solutions},\rdexocorrection[columns=1]{0},\rdexocorrection[columns=2]{0},


%\section{Présentation des interactions avec les logiciels BFcours}
%\input{interaction}


\newpage
\setcounter{pagecounter}{0}
\setcounter{ExoMA}{0}
\setcounter{prof}{1}

%\chapitre[
%    BF%
%    ]{
%    Exemples d'utilisation%
%    }{
%    R.Deschamps%
%    }{
 %   %
 %   }{
 %   %
 %   }{
 %   BFcours% : Solution,
 %   }
%
%    \input{Exemples}
\end{document}