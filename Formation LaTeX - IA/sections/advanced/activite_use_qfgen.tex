\subsection{Logiciel comme générateur de code LaTeX}

\begin{tcolorbox}[blank]
    
    L'utilisation de logiciels qui génèrent du code \LaTeX\ est une idée permettant de reléguer le côté fastidieux au second plan. 

Le mode d'action est relativement simple : 
\begin{tcbenumerate}
    \tcbitem Choisir un langage que vous maîtrisez un peu ( Python, javascript... )
    \tcbitem Lister les morceaux de code \LaTeX\ à produire. Plus vous avez d'exemples, mieux c'est. 
    \tcbitem Produire éventuellement un modèle \LaTeX\ que le logiciel viendra \acc{modifier}.
    \tcbitem Demander à l'IA de produire un script permettant de générer un document \LaTeX\ sur base de vos instructions.
    \tcbitem Testez, débuguez, modifiez et demander à l'IA d'améliorer son code grâce à vos retours.
\end{tcbenumerate}

On peut trouver beaucoup de cas d'utilisation : 

\begin{MultiColonnes}{4}[colframe=black,boxrule=0.4pt,colback=gray!5!white]%


    \tcbitem[colback=blue!10!white] Production de cartes
    \tcbitem[colback=green!10!white] Gestion des modèles
    \tcbitem[colback=yellow!10!white] Gestion d'une banque de questions
    \tcbitem[colback=yellow!10!white] Abstraction d'exercices pour produire plusieurs versions
    \tcbitem[colback=yellow!10!white] Produire des ressources générées procéduralement ( puzzles, carrés magiques... )
    \tcbitem[colback=yellow!10!white] Analyser du code \LaTeX\ et le modifier ( mise à jour de fichiers anciens ). 
    \tcbitem[colback=green!10!white] Retrouver la définition d'une commande spécifique. 
    \tcbitem[colback=green!10!white] Produire des rapports d'analyse d'évaluation.
\end{MultiColonnes}

Code couleur :

\begin{MultiColonnes}{3}% 
    \tcbitem[colback=yellow!10!white] En cours de production par \bfcours.
    \tcbitem[colback=blue!10!white] Solution disponible dans \bfcours\ mais à personnaliser.
    \tcbitem[colback=green!10!white] Solution disponible dans \bfcours.
\end{MultiColonnes}
\end{tcolorbox}