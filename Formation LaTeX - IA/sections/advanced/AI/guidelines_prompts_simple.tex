% Version simplifiée pour éviter les problèmes de compilation

% Commandes basiques pour XML
\newcommand{\xmlopen}[1]{\texttt{\textcolor{myblue}{<#1>}}}
\newcommand{\xmlclose}[1]{\texttt{\textcolor{myblue}{</#1>}}}
\newcommand{\xmltagsimple}[1]{\texttt{<#1>}}

% Pour éviter les problèmes de mode math avec les underscores
\newcommand{\xmltagname}[1]{\texttt{\detokenize{#1}}}

\subsection{Guidelines pour prompts structurés (2025)}

\begin{Definition}[Prompts structurés modernes]
    Les \voc{prompts structurés} représentent l'évolution des techniques de prompt engineering en 2025, combinant la précision du \acc{XML} pour la structure logique avec la lisibilité du \acc{Markdown} pour le contenu.
\end{Definition}

\begin{Methode}[Structure XML + Markdown]
    \begin{tcbraster}[raster columns=2]
        \begin{tcolorbox}[title=Principe fondamental]
            \begin{itemize}[label=$\bullet$]
                \item \acc{XML} pour la structure
                \item \acc{Markdown} pour le contenu
                \item Balises \voc{sémantiques}
            \end{itemize}
        \end{tcolorbox}
        \begin{tcolorbox}[title=Exemple, colback=gray!5]
            \ttfamily\small
            \xmlopen{instructions}\\
            \hspace{1em}\#\# Tâche principale\\
            \hspace{1em}- Analyser le document\\
            \hspace{1em}- Extraire les points\\
            \xmlclose{instructions}
        \end{tcolorbox}
    \end{tcbraster}
\end{Methode}

\begin{Propriete}[Ordre des sections]
    Structure recommandée du général au spécifique :
    
    \begin{tcbenumerate}[raster columns=2]
        \tcbitem \xmltagsimple{system\_role}
        \tcbitem \xmltagsimple{context}  
        \tcbitem \xmltagsimple{data}
        \tcbitem \xmltagsimple{rules}
        \tcbitem \xmltagsimple{examples}
        \tcbitem \xmltagsimple{instructions}
        \tcbitem \xmltagsimple{output\_format}
        \tcbitem \xmltagsimple{reminders}
    \end{tcbenumerate}
\end{Propriete}