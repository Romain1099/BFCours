% Document de diagnostic pour identifier les problèmes

% Solution 1 : Utiliser \texttt avec \detokenize pour les underscores
Les underscores posent problème : \texttt{\detokenize{system_role}}

% Solution 2 : Remplacer les underscores par des tirets
Alternative sans underscore : \texttt{<system-role>}

% Solution 3 : Protéger les caractères spéciaux
\catcode`\_=12 % Rendre underscore normal
Maintenant ça marche : \texttt{system_role}
\catcode`\_=8  % Remettre underscore en mode math

% Solution 4 : Créer une commande robuste
\newcommand{\safett}[1]{{\ttfamily\detokenize{#1}}}
Test : \safett{system_role}

% Solution 5 : Pour le XML, éviter complètement les problèmes
\newcommand{\xmltag}[1]{\texttt{<}\safett{#1}\texttt{>}}
\newcommand{\xmlctag}[1]{\texttt{</}\safett{#1}\texttt{>}}

Test complet : \xmltag{system_role} et \xmlctag{system_role}

% Pour les caractères spéciaux dans le contenu
\begin{tcolorbox}[colback=gray!5]
\ttfamily
<agent\_prompt>\\
\hspace{1em}<role>Expert</role>\\
</agent\_prompt>
\end{tcolorbox}

% Recommandations finales :
\begin{itemize}
\item Utiliser \textbackslash detokenize pour les noms avec underscores
\item Éviter verbatim qui pose des problèmes dans les environnements
\item Préférer les tcolorbox avec \textbackslash ttfamily
\item Pour le XML, créer des commandes dédiées qui gèrent les caractères
\end{itemize}