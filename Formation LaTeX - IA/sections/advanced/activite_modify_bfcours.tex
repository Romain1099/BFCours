\subsection{Adapter bfcours à ses besoins}

\begin{Methode}[Créer votre propre package]
 \begin{MultiColonnes}{3}
    \tcbitem[raster multicolumn=2] Il convient de créer votre propre package pour apporter vos modifications de façon globale. 

Il suffira alors de l'utiliser dans vos documents : 

\showcmd{usepackage}[\{bfcours\}]

\showcmd{usepackage}[\{adapt-bfcours\}]\mycomment{Votre package qui modifiera bfcours}

    \tcbitem \bclampe On pourra consulter l'archive suivante : \vocnoindexref{https://tuteurs.ens.fr/logiciels/latex/nouveau_package.html}{archives tuteurs ENS}.
 \end{MultiColonnes}
\end{Methode}
\begin{Methode}[Personnaliser un package]
Puisque \LaTeX\ est orienté vers la personnalisation, il est possible d'adapter n'importe quel package à vos besoins. 
Il y a plusieurs façons de procéder : 
\begin{tcbenumerate}[2]
    \tcbitem Créer vos propres commandes qui simplifient l'utilisation de celles données dans les packages utilisés. 

    \mycomment{définition d'origine}

    \showcmd{newcommand}[\{$\backslash$bonjour\}[1]\{Bonjour, \#1\}] 

    \mycomment{définition que vous utiliserez}

    \showcmd{newcommand}[\{$\backslash$mybonjour\}[1]\{$\backslash$bonjour\{\#1\} !\}] 

    \tcbitem Réécrire certaines commandes pour qu'elles agissent différemment. 

    Cela peut être une tâche ardue, parfois il faut retrouver le code d'origine de la commande, recopier son contenu et modifier la copie. 

    \mycomment{définition d'origine}

    \showcmd{newcommand}[\{$\backslash$bonjour\}\{bonjour\}] 

    \mycomment{On réécrit la définition de la commande}

    \showcmd{renewcommand}[\{$\backslash$bonjour\}\{Bonjour !\}] 
\end{tcbenumerate}
\end{Methode}

\begin{Methode}[Naviguer dans un package]
    Il est très \acc{utile} et \acc{formateur} de lire directement le code source des packages qu'on utilise. 

    \bfcours\ propose un \acc{programme de recherche} de définition de commande \LaTeX\ dans un package donné ( configurable par l'utilisateur ).

    Il permet de \acc{saisir le nom d'une commande}, d'\acc{environnement} ou de \acc{couleur} et s'il trouve sa définition, il \acc{ouvre} VSCode à la ligne trouvée.

    Il suffit de lire le fichier suivant et d'utiliser les commandes données.
    
    \displayFilePath{fichiers\_de\_la\_formation/programmes/Commandes\_de\_recherche\_dans\_un\_package/README.md}

    \encadrer[green]{Ce programme de recherche de code à été entièrement écrit par Claude.}
\end{Methode}