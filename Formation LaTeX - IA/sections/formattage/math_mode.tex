\subsection{Le mode mathématiques}

\begin{Methode}[Mode mathématique]
    Le \voc{mode mathématique} permet l'accès aux commandes de calcul et \acc{adapte} la police aux mathématiques. 

    Pour une \acc{documentation}, on peut conseiller : \vocnoindexref{https://math.univ-lyon1.fr/irem/spip.php?article340}{LaTeX pour le prof de maths} ou encore \vocnoindexref{https://tug.ctan.org/info/short-math-guide/short-math-guide.pdf}{Petit guide des mathématiques - CTAN}. 


    On l'utilise de plusieurs manières ayant chacune leur spécificité. 

    \begin{MultiColonnes}{2}[colframe=\itemBaseColor,boxrule=0.4pt]
        \tcbitem[title=Mode basique] S'utilise via : \$ $contenu\  maths$ \$

        $\rightarrow$ S'insère dans le texte.

        $\rightarrow$ Simple à utiliser. 
        \tcbitem[title=Mode étendu] S'utilise via : $\backslash$( $contenu\  maths$ $\backslash$)

        $\rightarrow$ S'insère dans le texte.

        $\rightarrow$ Tailles plus importantes ( fractions ). 
        \tcbitem[title=Mode display centré] S'utilise via : $\backslash$[ $contenu\  maths$ $\backslash$]

        $\rightarrow$ Saute une ligne et indente.

        $\rightarrow$ Tailles plus importantes ( fractions ). 
        \tcbitem[title=Mode equation / align]  \showenv[red]{align*}[][$contenu\ \& \  maths$]
    \end{MultiColonnes}
\end{Methode}