
\section{Exemples d'environnements LaTeX}

\showenv{multicols}[\{2\}][\showenv[green]{itemize}[[label=\$$\backslash$bullet\$]]]

\showenv[red]{tcbtab}[\{c|c|c\}]

\showenv[green]{itemize}[[label=\$$\backslash$bullet\$]]

\newpage
\begin{center}
    \begin{minipage}{0.9\textwidth}
    \begin{center}
    \textbf{Les différents types d'alignements}
    \end{center}
    
    \begin{multicols}{2}
    \begin{flushleft}
    Ce texte est aligné à gauche grâce à l'environnement \texttt{flushleft}.
    \end{flushleft}
    
    \columnbreak
    
    \begin{flushright}
    Ce texte est aligné à droite grâce à l'environnement \texttt{flushright}.
    \end{flushright}
    \end{multicols}
    
    \begin{center}
    Ce texte est centré grâce à l'environnement \texttt{center}.
	\begin{tcbtab}[Résumé des environnements d'alignement]{|l|c|r|}
	    \hline
	    \textbf{Environnement} & \textbf{Description} & \textbf{Utilisation} \\
	    \hline
	    flushleft & Aligne à gauche & Texte courant \\
	    center & Centre le texte & Titres, équations \\
	    flushright & Aligne à droite & Signature, date \\
	    \hline	\end{tcbtab}
    \end{center}
    
    
    \end{minipage}
    \end{center}


\newpage

\showenv{center}[][
	 \showenv{minipage}[\{0.9$\backslash$textwidth\}][
		 \showenv{center}[][
		 	\showcmd{textbf}[Les différents types d'alignements]
		]\\
		\showenv{multicols}[\{2\}][
			\showenv{flushleft}[][
				Ce texte est aligné à gauche grâce à l'environnement \showcmd{texttt}[flushleft].
			]\\
			\showcmd{columnbreak}[]\\
			\showenv{flushright}[][
		 		Ce texte est aligné à droite grâce à l'environnement \showcmd{texttt}[flushright].
			]
		]\\
    \showenv{center}[][
        \showenv{tcbtab}[[Résumé des environnements d'alignement]\{|l|c|r|\}][
            \showcmd{hline}\\
            flushleft \& Aligne à gauche \& Texte courant $\backslash$$\backslash$\\
            center \& Centre le texte \& Titres, équations $\backslash$$\backslash$\\
            flushright \& Aligne à droite \& Signature, date $\backslash$$\backslash$\\
            \showcmd{hline}
        ]
    ]
	]
]

\newpage

\showenv{MultiColonnes}[\{2\}[colframe=black,boxrule=0.4pt,halign=center]\textcolor{green!75!black}{\textit{\% Default [blank]}}][
  \showcmd{tcbitem}[[valign=bottom]] Hauteur adaptée pour \acc{chaque ligne}.\\
  \showcmd{tcbitem}[[title=La seule boite titrée,colframe=black,boxrule=0.4pt, fonttitle=$\backslash$scriptsize$\backslash$bfseries]] Bonjour !\\
  \showcmd{tcbitem}[[raster multicolumn=2,halign=center]]Fusion facile..\\
  \showcmd{tcbitem}[[colback=green!25!white]]Ligne 2 colonne 1\\
  \showcmd{tcbitem}[] Utilisation basique.
]

\acc{Rendu du code : }
\begin{MultiColonnes}{2}[colframe=black,boxrule=0.4pt,halign=center]%
  \tcbitem[valign=bottom] Hauteur adaptée pour \acc{chaque ligne}.
  \tcbitem[title=La seule boite titrée,colframe=black,boxrule=0.4pt, fonttitle=\scriptsize\bfseries,] Bonjour !
    \tcbitem[raster multicolumn=2,halign=center] Fusion facile..
  \tcbitem[colback=green!25!white] Ligne 3 colonne 1
  \tcbitem Utilisation basique.
\end{MultiColonnes}

\acc{Version utilisant les options par défaut : } ( enlever les \acc{options})
\begin{MultiColonnes}{2}%
  \tcbitem[valign=bottom] Hauteur adaptée pour \acc{chaque ligne}.
  \tcbitem[title=La seule boite titrée,colframe=black,boxrule=0.4pt, fonttitle=\scriptsize\bfseries,] Bonjour !
    \tcbitem[raster multicolumn=2,halign=center] Fusion facile..
  \tcbitem[colback=green!25!white] Ligne 3 colonne 1
  \tcbitem Utilisation basique.
\end{MultiColonnes}
\acc{Style modifié : }

\showcmd[orange]{tcbset}\{\\
  \phantom{AA}ColonnesBaseStyle/.style=\{\\
    \phantom{AAAA}top=0pt,\\
    \phantom{AAAA}bottom=0pt,\\
    \phantom{AAAA}left=0pt,\\
    \phantom{AAAA}right=0pt,\\
    \phantom{AAAA}colback=blue!5!white,\\
    \phantom{AAAA}colframe=blue!75!black,\\
    \phantom{AAAA}before title={\showcmd{dimcoloredsquare}[\{white\}\{1.5\}$\backslash$ ]},\\
    \phantom{AAAA}after title=\{\showcmd{hfill} \showcmd{today}\},\\
    \phantom{AAAA}boxrule=0.4pt\\
    \phantom{AA}\}\\
\}
\tcbset{
  ColonnesBaseStyle/.style={
    top=0pt,
    bottom=0pt,
    left=0pt,
    right=0pt,
    colback=blue!5!white,
    colframe=blue!75!black,
    before title={\dimcoloredsquare{white}{1.5}\ },
    after title={\hfill \today},
    boxrule=0.4pt
  }
}
\begin{MultiColonnes}{2}%
  \tcbitem[valign=bottom] Hauteur adaptée pour \acc{chaque ligne}.
  \tcbitem[title=La seule boite titrée,colframe=black,boxrule=0.4pt, fonttitle=\scriptsize\bfseries,] Bonjour !
    \tcbitem[raster multicolumn=2,halign=center] Fusion facile..
  \tcbitem[colback=green!25!white] Ligne 3 colonne 1
  \tcbitem Utilisation basique.
\end{MultiColonnes}