%Insérer ici la méthode pour sauvegarder les documents. 
\begin{Methode}[Sauvegarder des documents]
    Chacun des exercices suivants demande d'effectuer des \acc{manipulations sur des fichiers}. 

    Ces fichiers seront \acc{ramassés numériquement} et évalués \acc{automatiquement}. 

    \begin{MultiColonnes}{3}
        \tcbitem[raster multicolumn=2] \begin{MultiColonnes}{2}
        \tcbitem[raster multicolumn=2,boxrule=0.4pt,colframe=red!75!black,colback=red!10!white,boxsep=5pt] Pour que le \acc{programme de correction} fonctionne correctement, les \acc{noms des fichiers} doivent respecter une \acc{structure précise}.
        \tcbitem[raster multicolumn=2] Nom du fichier $=$ \formuleTable{Exercice\_i\_nom\_prenom.ods}

        Dans lequel : 
        \begin{tcbenumerate}[2]
            \tcbitem \formuleTable{i} est le numéro de l'exercice. 
            \tcbitem \formuleTable{nom} est ton nom. 
            \tcbitem \formuleTable{prenom} est ton prénom. 
            \tcbitem \formuleTable{.ods} est l'\voc{extension} du fichier.
        \end{tcbenumerate}
        \end{MultiColonnes}
    \tcbitem \dirtree{%
                .1 \textbf{Espace des classes}.
                .2 \textbf{Ma classe}.
                .3 \textbf{Restitution de \phantom{aa}documents}.
                .4 \textbf{Tableur}.
                .5 \textbf{Fichier bien nommé.xlsx}.
            }%
    \end{MultiColonnes}%
    \begin{MultiColonnes}{2}%
        \tcbitem Les manipulations à effectuer dans les fichiers sont repérées par les \encadrer[green]{\faLaptop\ points}. 
        \tcbitem Les réponses sont à noter \acc{sur le document}.

        Elles sont repérées par les \encadrer[purple]{points}. 
    \end{MultiColonnes}%
    \encadrer[red]{En cas de problème de \acc{détection}, la note attribuée sera $0$ pour chaque fichier non trouvé. }%
\end{Methode}
