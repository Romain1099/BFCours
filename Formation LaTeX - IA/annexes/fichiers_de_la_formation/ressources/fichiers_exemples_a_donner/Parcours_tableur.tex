\documentclass[a4paper,11pt,fleqn]{article}

\usepackage[left=1cm,right=1cm,top=0.5cm,bottom=2cm]{geometry}

\usepackage{bfcours}
\usepackage{bfcours-fonts}
%\usepackage{bfcours-fonts-dys}

\def\rdifficulty{1}
\setrdexo{%left skip=1cm,
display exotitle,
exo header = tcolorbox,
%display tags,
skin = bouyachakka,
lower ={box=crep},
display score,
display level,
save lower,
score=\points,
level=\rdifficulty,
overlay={\node[inner sep=0pt,
anchor=west,rotate=90, yshift=0.3cm]%,xshift=-3em], yshift=0.45cm
at (frame.south west) {\thetags[0]} ;}
]%obligatoire
}
\setrdcrep{seyes, correction=true, correction color=monrose, correction font = \large\bfseries}

\newcommand{\tikzinclude}[1]{%
    \stepcounter{tikzfigcounter}%
    \csname tikzfig#1\endcsname
}
\newcommand{\figureLongueurCercle}{
    \definecolor{ttzzqq}{rgb}{0.2,0.6,0.}
\definecolor{qqqqff}{rgb}{0.,0.,1.}
\definecolor{xdxdff}{rgb}{0.49019607843137253,0.49019607843137253,1.}
\definecolor{ffwwqq}{rgb}{1.,0.4,0.}
\definecolor{ududff}{rgb}{0.30196078431372547,0.30196078431372547,1.}
\definecolor{uuuuuu}{rgb}{0.26666666666666666,0.26666666666666666,0.26666666666666666}
\definecolor{yqqqyq}{rgb}{0.5019607843137255,0.,0.5019607843137255}
\begin{tikzpicture}[line cap=round,line join=round,>=triangle 45,x=1.0cm,y=1.0cm]
\clip(-1.3,-2.) rectangle (6.,1.);
\draw [line width=0.8pt,color=yqqqyq] (0.,0.) circle (0.75cm);
\draw [line width=0.8pt,color=yqqqyq] (-1.,-1.5)-- (3.7123889803846897,-1.5);
\draw [line width=0.8pt,dash pattern=on 1pt off 1pt,color=ffwwqq] (-0.75,0.)-- (0.75,0.);
\draw [line width=0.8pt,dash pattern=on 1pt off 1pt,color=ffwwqq] (-0.02306940646219445,0.05382861507845321) -- (-0.02306940646219445,-0.05382861507845321);
\draw [line width=0.8pt,dash pattern=on 1pt off 1pt,color=ffwwqq] (0.023069406462194016,0.05382861507845321) -- (0.023069406462194016,-0.05382861507845321);
\draw [line width=0.8pt,dash pattern=on 1pt off 1pt,color=qqqqff] (0.,0.)-- (0.4867299416577562,0.5706084155476829);
\draw [line width=0.8pt,dash pattern=on 1pt off 1pt,color=qqqqff] (0.1994863125988175,0.32273277731191763) -- (0.2872436290589386,0.2478756382357651);
\draw [line width=0.8pt,dash pattern=on 1pt off 1pt,color=ffwwqq] (-1.,-1.2)-- (0.5,-1.2);
\draw [line width=0.8pt,dash pattern=on 1pt off 1pt,color=ffwwqq] (-0.2730694064621942,-1.1461713849215465) -- (-0.2730694064621942,-1.253828615078453);
\draw [line width=0.8pt,dash pattern=on 1pt off 1pt,color=ffwwqq] (-0.2269305935378057,-1.1461713849215465) -- (-0.2269305935378057,-1.253828615078453);
\draw [line width=0.8pt,dash pattern=on 1pt off 1pt,color=ffwwqq] (0.5,-1.2)-- (2.,-1.2);
\draw [line width=0.8pt,dash pattern=on 1pt off 1pt,color=ffwwqq] (1.2269305935378059,-1.1461713849215465) -- (1.2269305935378059,-1.253828615078453);
\draw [line width=0.8pt,dash pattern=on 1pt off 1pt,color=ffwwqq] (1.2730694064621944,-1.1461713849215465) -- (1.2730694064621944,-1.253828615078453);
\draw [line width=0.8pt,dash pattern=on 1pt off 1pt,color=ffwwqq] (2.,-1.2)-- (3.5,-1.2);
\draw [line width=0.8pt,dash pattern=on 1pt off 1pt,color=ffwwqq] (2.726930593537806,-1.1461713849215465) -- (2.726930593537806,-1.253828615078453);
\draw [line width=0.8pt,dash pattern=on 1pt off 1pt,color=ffwwqq] (2.773069406462194,-1.1461713849215465) -- (2.773069406462194,-1.253828615078453);
\draw [line width=0.8pt,dash pattern=on 1pt off 1pt,color=ttzzqq] (3.7123889803846897,-1.)-- (3.7123889803846892,-1.6);
\draw [line width=0.8pt,dash pattern=on 1pt off 1pt,color=ttzzqq] (3.5,-1.)-- (3.5,-1.6);
\draw (1.0529201296198805,0.538595212100376) node[anchor=north west] {$r = 0.75$};
\draw (1.0606099317739452,0.1387255000890093) node[anchor=north west] {$c = {\color{blue}r} \times {\color{red}\pi} \approx 4.71$};
\draw (0.7607076477654201,0.8615669025710952) node[anchor=north west] {$\textbf{Circonférence du cercle} \mathcal{C} :$};
\draw (-0.738803772277205,0.8615669025710952) node[anchor=north west,xshift=-0.5cm] {$\mathcal{C}$};
\begin{scriptsize}
\draw[color=yqqqyq] (1.2451651834714994,-1.6952923136554323) node {$4.71$};
\draw [color=uuuuuu] (0.75,0.)-- ++(-3.0pt,0 pt) -- ++(6.0pt,0 pt) ++(-3.0pt,-3.0pt) -- ++(0 pt,6.0pt);
\draw [color=ududff] (-1.,-1.2)-- ++(-3.0pt,0 pt) -- ++(6.0pt,0 pt) ++(-3.0pt,-3.0pt) -- ++(0 pt,6.0pt);
\draw[color=ffwwqq] (0.030176443129269655,-0.1957808936128071) node {$1.5$};
\draw [color=xdxdff] (0.4867299416577562,0.5706084155476829)-- ++(-3.0pt,0 pt) -- ++(6.0pt,0 pt) ++(-3.0pt,-3.0pt) -- ++(0 pt,6.0pt);
%\draw[color=qqqqff] (0.41466655083250686,0.20408881839855963) node {$0.75$};
\draw [color=uuuuuu] (0.5,-1.2)-- ++(-3.0pt,0 pt) -- ++(6.0pt,0 pt) ++(-3.0pt,-3.0pt) -- ++(0 pt,6.0pt);
\draw[color=ffwwqq] (0.09938466251585235,-1.1031775477924468) node {$1.5$};
\draw [color=uuuuuu] (2.,-1.2)-- ++(-3.0pt,0 pt) -- ++(6.0pt,0 pt) ++(-3.0pt,-3.0pt) -- ++(0 pt,6.0pt);
\draw[color=ffwwqq] (1.4297204351690531,-1.041659130559929) node {$1.5$};
\draw [color=uuuuuu] (3.5,-1.2)-- ++(-3.0pt,0 pt) -- ++(6.0pt,0 pt) ++(-3.0pt,-3.0pt) -- ++(0 pt,6.0pt);
\draw[color=ffwwqq] (2.9446114595198076,-1.0493489327139938) node {$1.5$};
\end{scriptsize}
\end{tikzpicture}
}

\newcommand{\figureAireCarre}{
    \definecolor{ffwwqq}{rgb}{1.,0.4,0.}
    \definecolor{zzttqq}{rgb}{0.6,0.2,0.}
    \begin{tikzpicture}[line cap=round,line join=round,>=triangle 45,x=1.0cm,y=1.0cm]
    \clip(-0.5,-0.5) rectangle (6.5,3.);
    \fill[line width=0.8pt,color=zzttqq,fill=zzttqq,fill opacity=0.10000000149011612] (0.,0.) -- (2.,0.) -- (2.,2.) -- (0.,2.) -- cycle;
    \draw [line width=0.8pt,color=zzttqq] (0.,0.)-- (2.,0.);
    \draw [line width=0.8pt,color=zzttqq] (2.,0.)-- (2.,2.);
    \draw [line width=0.8pt,color=zzttqq] (2.,2.)-- (0.,2.);
    \draw [line width=0.8pt,color=zzttqq] (0.,2.)-- (0.,0.);
    \draw [line width=0.8pt,dash pattern=on 1pt off 1pt,color=ffwwqq] (0.,2.1990485449122428)-- (2.,2.2);
    \draw (2.3265445321089646,2.0241005763481263) node[anchor=north west] {$\mathcal{A}_{\text{Carré}} = c \times c $};
    \draw (2.3357854785212573,1.515848523672014) node[anchor=north west] {$\mathcal{A}_{\text{Carré}} = 2 \times 2 = 4 $};
    \begin{scriptsize}
    \draw[color=ffwwqq] (0.9866073023264862,2.4260817452828696) node {$2$};
    \end{scriptsize}
    \end{tikzpicture}
}

\newcommand{\solEquation}{
    \begin{center}
        \tdplotsetmaincoords{60}{120}
        \begin{tikzpicture}[tdplot_main_coords, scale=0.8]
            % Paramètres de vue en 3D
            
            %\begin{tdplotpicture}[tdplot main coords]
        
                % Axes en 3D
                \draw[thick,->] (0,0,0) -- (4,0,0) node[anchor=north east]{$x$};
                \draw[thick,->] (0,0,0) -- (0,4,0) node[anchor=north west]{$y$};
                \draw[thick,->] (0,0,0) -- (0,0,4) node[anchor=south]{$z$};
        
                % Points pour le plan (2x + y - z = 3)
                \coordinate (A) at (0,3.5,0);   % Point d'intersection avec y
                \coordinate (B) at (2,1.5,0); % Point d'intersection avec x
                \coordinate (C) at (0,0,3);  % Point d'intersection avec z
        
                % Tracer le plan en utilisant les points
                \filldraw[fill=blue!20,opacity=0.5] (A) -- (B) -- (C) -- cycle;
        
                % Lignes principales du plan
                \draw[thick, blue] (A) -- (B) -- (C) -- cycle;
        
                % Hachures pour la région de solutions (en dessous du plan)
                \begin{scope}
                    \clip (A) -- (B) -- (C) -- cycle;
                    \foreach \z in {-3,-2.5,...,0} {
                        \draw[dashed, blue!60,opacity=0.7] (0,3,\z) -- (1.5,0,\z);
                    }
                \end{scope}
        
                % Étiquettes des points
                \node at (A) [anchor=south west] {$(0;3{,}5;0)$};
                \node at (B) [anchor=north] {$(2;1{,}5;0)$};
                \node at (C) [anchor=east] {$(0;0;3)$};
        
        \end{tikzpicture}
    \end{center}
}

\hypersetup{
    pdfauthor={R.Deschamps},
    pdfsubject={},
    pdfkeywords={},
    pdfproducer={LuaLaTeX},
    pdfcreator={Boum Factory}
}
% Activer ou désactiver l'affichage des boîtes pour les points
%\displayitempointsfalse % Ne pas afficher les boîtes
\displayitempointstrue % Afficher les boîtes pour les points

\usepackage{dirtree}
\begin{document}

\setcounter{pagecounter}{0}
\setcounter{ExoMA}{0}
\setcounter{prof}{0}

\def\points{\phantom{AAA}}
\def\difficulty{\phantom{AAA}}
\chapEval[
    \faLaptop%$\mathbf{5^{\text{ème}}}$% : $\mathbf{6^{\text{ème}}}$,$\mathbf{5^{\text{ème}}}$,$\mathbf{4^{\text{ème}}}$,$\mathbf{3^{\text{ème}}}$,$\mathbf{2^{\text{nde}}}$,$\mathbf{1^{\text{ère}}}$,$\mathbf{T^{\text{Le}}}$,
    ]{
    Parcours tableur% : ,Equations
    }{
    Collège% : Collège,Lycée
    }{
    Gaston Bachelard% : Othe et Vanne,Amadis Jamyn,Eugène Belgrand
    }{
    \tableauProjetEval{}%17/06/2025}% : 10},15},30},55}%}
    }{
    Informatique :
    }

\setrdcrep{seyes, correction=false, correction color=monrose, correction font = \large\bfseries}


\begin{None}
    \begin{tcbraster}[
        raster columns     = 3,
        raster width       = \textwidth,
        %size               = fbox,
        raster equal height,
        raster column skip = 5pt,
        raster row skip    = 2pt
      ]%
        \begin{tcolorbox}[blankest,raster multicolumn=2]
                \printcompindex%
        \end{tcolorbox}
        \begin{tcolorbox}[blankest, width=\textwidth]
            %\begin{minipage}[t]{0.85\textwidth}
                \textbf{Bilans :}


                \begin{tcolorbox}[colback=white, colframe=black, width=\textwidth, top=1pt, bottom=1pt, left=1pt, right=1pt, boxrule=1pt, arc=2pt, auto outer arc, boxsep=0pt, nobeforeafter]%0.280
                    {\tiny{\bccrayon}} \textbf{- \tccrep[seyes=false]{1.5cm}{}/\getsavedtotalpoints\phantom{A}\tccrep[seyes=false]{2cm}{}} 
                  \end{tcolorbox}
                \begin{tcolorbox}[colback=white, colframe=black, width=\textwidth, top=1pt, bottom=1pt, left=1pt, right=1pt, boxrule=1pt, arc=2pt, auto outer arc, boxsep=0pt, nobeforeafter]%0.280
                    \textbf{\faLaptop\ - \tccrep[seyes=false]{1.5cm}{}/\getsavedinfototalpoints\phantom{A}\tccrep[seyes=false]{2cm}{}} 
                \end{tcolorbox}
            %\end{minipage}\\
            \begin{minipage}[t]{\textwidth}
                \textbf{Modalités} :
                \begin{itemize}[label=$\bullet$,leftmargin=10pt]
                    \item Lire les exercices avant de les compléter.% : \item Calculatrice interdite, \item Calculatrice autorisée,\item 
                    \item Les feuilles de calcul doivent \acc{toutes} être \acc{enregistrées}.
                    \item Le livret doit être \acc{complété et rendu} pour évaluation.
                \end{itemize}
            \end{minipage}
        \end{tcolorbox}
    \end{tcbraster}
    \end{None}


%Insérer ici le sommaire et l'index vocabulaire.
\vfill
\tableofcontents
\vfill
\printvocindex
\vfill

\newpage
\section{Expression littérale}
\begin{Definition}[Expression littérale]
    Une \voc{expression littérale} est une expression mathématique contenant une ou plusieurs lettres qui désignent des nombres.\\

    Les lettres sont appelées des \voc{variables}.
\end{Definition}
\begin{Exemple}[Expression numérique]
    \begin{multicols}{2}
        \[A=5+7=12\]
        \[B=(-5)\times 7=-35\]
        \[C=\dfrac{5}{2}-\dfrac{2}{3}=\dfrac{5\times 3}{2\times 3}- \dfrac{2\times 2}{3\times 2}=\dfrac{15}{6}-\dfrac{4}{6}=\dfrac{11}{6}\]
        \[D=10^{2}\times 10^{5}=10^{7}\]
    
    
    \columnbreak

        Toutes ces expressions sont composées uniquement de \acc{nombres}, ce sont des \acc{expressions numériques}
    \end{multicols}
\end{Exemple}
\begin{Exemple}[Expression littérale]
    \begin{multicols}{2}
    \begin{enumerate}
        \item La longueur $c$ d’un cercle de rayon $r$ est donnée par : \\
        $c = 2\times {\color{red}\pi} \times \color{blue}r$ 
        où $\color{red}\pi\approx \color{red}3{,}14$…
        \begin{center}
            \figureLongueurCercle
        \end{center}
        Cette formule comporte \acc{une variable}
        \columnbreak

        \item L’aire d’un carré est donné par $\mathbf{\color{blue}c\times c}$
        où $\mathbf{\color{blue}c}$ représente le \acc{côté du carré}.
        \vspace{-0.8cm}\begin{center}
            \figureAireCarre
        \end{center}
    \end{enumerate}
    
    
    \end{multicols}
    \begin{enumerate}[start=3]
        \item Dans l'égalité $6x+6y+7z=21$, il y a \acc{trois variables} : $x ; y \text{ et } z$.\\
    \end{enumerate}
    \solEquation
\end{Exemple}
\newpage
\section{Valeur d’une expression littérale}

\subsection{\'Evaluer une expression littérale}

\begin{Definition}[\'Evaluer une expression littérale]
    \voc{\'Evaluer une expression} littérale signifie \acc{attribuer} une \acc{valeur numérique} à une ou plusieurs \acc{variables}.

    Il s'agit ensuite de \acc{remplacer} ces variables par les valeurs numériques, puis d'\acc{effectuer les calculs} rendus possibles.\\
\end{Definition}

\begin{Methode}
    \begin{minipage}[t]{0.475\textwidth}
        \'Evaluer l'expression suivante pour $x = 5$
        \[A = 2x + t + 3\]

        \textbf{Solution :}\\
        On \acc{remplace} les variables $x$ par $5$ :\\
        \[A = 2\times 5 + t + 3 \quad \text{t n'est pas évaluée}\]
        \[A = 10 + t + 3\]
        \[A = 13 + t\]
    \end{minipage}
    \hfill
    \begin{minipage}[t]{0.475\textwidth}
        Evaluer l'expression suivante pour $x = -1$

        \[B = x + (1 - x) + 3x\]

        \textbf{Solution :}\\
        On \acc{remplace} les variables $x$ par $-1$ :\\
        \[B = -1 + (1 - (-1)) + 3\times ( -1 )\]
        \[B = -1 + (1 + 1) - 3\]
        \[B = -1 + 2 - 3\]
        \[B = -2\]
    \end{minipage}
\end{Methode}

\begin{Remarque}
    \bcattention Penser à rajouter les signes $\times$ lorsqu'on remplace une variable qui est multipliée à un nombre.\\
    \bcattention Les variables ayant des valeurs négatives peuvent être placées \acc{entre parenthèses} pour éviter les erreurs de calcul.
\end{Remarque}

\def\points{8}
\begin{EXO}{\'Evaluer une expression littérale}{C4L13}
    Calculer la valeur des expressions suivantes pour les valeurs données :
    \begin{enumerate}
        \begin{minipage}{0.4\textwidth}\item $A=6\times (x+3)$ lorsque $x=5$
            \vspace{-0.25cm}\begin{crep}
                \[A = 6\times (5+3)\]
                \[A = 6\times 8\]
                \[A = 48\]
            \end{crep}
        \end{minipage}
        \hfill
        \begin{minipage}{0.475\textwidth}\item $B=l\times L$ lorsque $l=3{,}5$ et $L=7$
            \vspace{-0.25cm}\begin{crep}
                \[B = l\times L\]
                \[B = 3{,}5\times 7\]
                \[B = 24{,}5\]
            \end{crep}
        \end{minipage}
        \begin{minipage}{0.4\textwidth}\item $C=5\times(6-x) + 3x -7y$ lorsque $x=2$ et $y=1$
            \vspace{-0.25cm}\begin{crep}
                \[C = 5\times(6-2) + 3\times 2 -7\times 1 \]
                \[C = 5\times(4) + 6 -7 \]
            \end{crep}
        \end{minipage}
        \hfill
        \begin{minipage}{0.475\textwidth}
            \begin{crep}
                \[C = 20 - 1\]
                \[C = 19\]
            \end{crep}
        \end{minipage}
    \end{enumerate}
    
\end{EXO}

\subsection{Nature d'une égalité}
\begin{Definition}
    Une égalité est constituée de deux expressions mathématiques appelées « \voc{membres} » séparées par un signe « = ».
\end{Definition}
\begin{Vocabulaire}[Nature d'une égalité]

    \begin{center}
        \begin{tcbtab}[Une égalité peut être :]{c|c|c|c}
            \voc{Nature de l'égalité}& \acc{Vraie} & \acc{Fausse} & \acc{Parfois vraie, parfois fausse} \\
            \hline
            \acc{Exemple} 1 & $3^2 = (-3)^2$ & $\dfrac{1}{3}=0{,}33$ & $2x=10$ \\
            \acc{Exemple} 2 & $x+x=2x$ & $x^2=-1$ & $3x+1=5x-4$ 
        \end{tcbtab}
    \end{center}
\end{Vocabulaire}
\begin{Exemple}
    \vspace{-0.25cm}\begin{enumerate}
    \item L'égalité $5\times 2=6+4$ est \repsim[3cm]{vraie}, car \repsim[6cm]{$5\times 2=10$ et $6+4=10$}.\\
    \item L'égalité $4\times 6=24+3$ est \repsim[3cm]{fausse}, car \repsim[6cm]{$4\times 6=24$ mais $24+3=27$}.
    \item L'égalité $4x + 6 +2x = 2x\times 3 +2\times 3$ est \repsim[3cm]{vraie} car :
        \begin{crep}
            \begin{minipage}[t]{0.475\textwidth}
                D'une part : \\
                $4x + 6 + 2x = 6x + 6$
            \end{minipage}
            \hfill
            \begin{minipage}[t]{0.475\textwidth}
                D'autre part :\\
                $2x \times 3 + 2 \times 3 = 6x + 6$
            \end{minipage}\\\\
            Les membres de gauche et de droite sont tous les deux égaux à la même \acc{expression littérale} $6x + 6$.\\
            On en déduit que cette égalité est vraie quelle que soit la valeur de $x$.
        \end{crep}
    \end{enumerate}
\end{Exemple}
\def\points{6}
\def\rdifficulty{2}
\begin{EXO}{Déterminer la nature d'une égalité}{C4L14}
    \vspace{-0.25cm}\begin{enumerate}
    $3x+6=2(x+5)$ est \repsim[3cm]{fausse} car :
    \begin{crep}
        \begin{minipage}[t]{0.475\textwidth}
            D'une part : \\
            $2(x + 5) = 2x + 10$
        \end{minipage}
        \hfill
        \begin{minipage}[t]{0.475\textwidth}
            D'autre part :\\
            $3x + 6 \neq 2x + 10$
        \end{minipage}\\\\
        Les deux membres ne sont pas égaux.
    \end{crep}

    \vspace{-0.25cm}$x^{2}=2x$ est \repsim[3cm]{fausse} car :
    \begin{crep}
        Les deux membres ne sont pas égaux pour toutes les valeurs de $x$ : \\
        Si x = 3 par exemple, alors : $x^2 = 3^2 = 9$ mais $2x = 2\times 3 =6$
    \end{crep}
    \end{enumerate}
\end{EXO}
\vspace{-0.25cm}\begin{Remarque}
    Parfois ces égalités, par exemple $3x+5=7$ ou $4x+4=7x+2$, peuvent être égales pour certaines valeurs de $x$, on parle d'équation.
\end{Remarque}

\newpage
\section{Développement}
\subsection{Introduction}
\begin{Activite}[Les bouteilles]
    Un restaurateur a commandé 3 caisses de jus d’orange et 5 caisses de jus de raisin.\\
    Chaque caisse contient 24 bouteilles de jus.\\
    \begin{enumerate}
        \item Répondre à la question suivante de \acc{deux façons différentes} :\\
            \textbf{Combien a-t-il commandé de bouteilles en tout ?}
        \item Que remarque-t-on ?
    \end{enumerate}
    \tcblower
    \begin{enumerate}
        \item \begin{minipage}[t]{0.475\textwidth}
            \acc{Première solution :}
            \begin{crep}
                Le restaurateur a commandé $3\times\blue{24}=72$ bouteilles de jus d'orange.\\
                Il a également commandé $5\times\blue{24}=120$ bouteilles de jus de raisin.\\
                Le nombre total de bouteilles s'écrit :\\
                $3\times \blue{24} + 5 \times \blue{24} = 72 + 120 = 192$ bouteilles.
            \end{crep}
        \end{minipage}
        \hfill
        \begin{minipage}[t]{0.475\textwidth}
            \acc{Seconde solution :}
            \begin{crep}
                Puisque les caisses contiennent le même nombre de bouteilles, il suffit de \acc{multiplier} le nombre de caisses \acc{au total} par le nombre de \acc{bouteilles} par caisse : \\
                $( 3 + 5 ) \times \blue{24} = 8 \times \blue{24} = 192$ bouteilles.
            \end{crep}
        \end{minipage}

        \item On remarque que...\begin{crep}
            L'égalité suivante est vraie : \\
            $( 3 + 5 ) \times \blue{24} = 3\times \blue{24} + 5 \times \blue{24}$
        \end{crep}
    \end{enumerate}
    
\end{Activite}
\subsection{Distributivité}
\begin{Propriete}[Distributivité simple]
    On considère trois expressions ( numériques ou littérales ) a, b et c. \\
    Les égalités suivantes sont \acc{toujours vérifiées} :\\
    \begin{minipage}[t]{0.475\textwidth}
        \begin{center}\Large$\blue{a}(\red{b}+\red{c})=\blue{a}\red{b}+\blue{a}\red{c}$\end{center}
    \end{minipage}
    \hfill
    \begin{minipage}[t]{0.475\textwidth}
        \begin{center}\Large$\blue{a}(\red{b}-\red{c})=\blue{a}\red{b}-\blue{a}\red{c}$\end{center}
    \end{minipage} 
\end{Propriete}

\begin{Exemple}[Calculs astucieux]
    \vspace{-0.35cm}
    \begin{minipage}[t]{0.3\textwidth}\bclampe Pour calculer astucieusement, on peut utiliser la \voc{distributivité}.\end{minipage}
    \hfill
    \begin{minipage}[t]{0.6\textwidth}
    Puisque $101 = 100 + 1$, on peut écrire :
    \[\blue{32} \times 101 = \blue{32} \times (\red{100} + \red{1}) = \blue{32} \times \red{100} + \blue{32} \times \red{1} = 3\,200 + 32 = \color{\currentAccentColor}{3\,232}\]
    \end{minipage}\\

    %\acc{Calculer astucieusement :}\\
    $32\times 99=\tcfillcrep{\blue{32} \times (\red{100} - \red{1}) = \blue{32} \times \red{100} - \blue{32} \times \red{1} = 3\,200 - 32 = 3\,168}$\\
    $13\times 102=\tcfillcrep{\blue{13} \times (\red{100} + \red{2}) = \blue{13} \times \red{100} + \blue{13} \times \red{2} = 1\,300 + 26 = 1\,326}$\\
    $29\times 999=\tcfillcrep{\blue{29} \times (\red{1\,000} - \red{1}) = \blue{29} \times \red{1\,000} - \blue{29} \times \red{1} = 29\,000 - 29 = 28\,971}$
\end{Exemple}
\begin{Definition}[Développer une expression]
    \voc{Développer}, c’est transformer un \acc{produit} en \acc{somme} (ou \acc{différence}).\\
    Dans la pratique, développer c’est lire la formule de distributivité \acc{de la gauche vers la droite}.
\end{Definition}
\begin{Exemple}
    \bclampe Pour \acc{développer} une expression, il suffit de lire la formule de distributivité \acc{de la gauche vers la droite}.\\
    \textbf{L'expression} $A = 4(5+x)$ est un \voc{produit}. \\
    On peut le développer en : 
    $A = \blue{4}(\red{5}+\red{x}) = \blue{4}\times\red{5}+\blue{4}\times\red{x}$
\end{Exemple}

\def\points{4}
\def\rdifficulty{1}
\begin{EXO}{Développer avec la simple distributivité}{C4L21}
    Développe les expressions suivantes :
    \vspace{-0.35cm}\begin{multicols}{2}
        \begin{enumerate}
            \item $A = 5(x-2)$\begin{crep}
            $A = \blue{5}(\red{x}-\red{2}) \\= \blue{5}\times\red{x}-\blue{5}\times\red{2}\\ = 5x - 10$
            \end{crep}
            \item $B = -6(-2x+4)$\begin{crep}
            $B = \blue{-6}(\red{-2x}+\red{4}) \\= \blue{-6}\times\red{(-2x)}+\blue{(-6)}\times\red{4} \\= 12x - 24$
            \end{crep}
        \end{enumerate}
    \end{multicols}
    \begin{multicols}{2}
        \begin{enumerate}[start=3]
            \item $C = -x(2-3x)$\begin{crep}
            $C = \blue{-x}(\red{2}-\red{3x}) \\= \blue{-x}\times\red{2}-\blue{-x}\times\red{(-3x)} \\= -2x - 3x^2$
            \end{crep}
            \item $D = -(5-x)$\begin{crep}
            $D = \blue{-}(\red{5}-\red{x}) \\= \blue{-1}\times\red{5}-\blue{(-1)}\times\red{x} \\= -5 + x$
            \end{crep}
        \end{enumerate}
    \end{multicols}
\end{EXO}

\subsection{Réduire une expression}
\begin{Definition}[Réduire une expression]
    \voc{Réduire} une expression, c’est l’écrire avec le moins de termes ou de facteurs possibles. \\
    Pour cela on \acc{regroupe} les termes de \acc{même nature}. 
\end{Definition}
\begin{Exemple}
    \vspace{-0.25cm}
    \begin{multicols}{2}
        \acc{Réduire} les expressions suivantes :
        \begin{enumerate}
            \item $A = 4x+3x = \repsim[3cm]{7x}$
            \item $B = 2a+4-3a+6-2a+8a-8\\B = \repsim{5}\times a + \repsim{2}$
        \end{enumerate}
    \end{multicols}
    \vspace{-0.25cm}
    \begin{enumerate}[start=3]
        \item $C= x^{2}+8x-7-8x+15-2x^{2}+3x = \repsim[5cm]{-x^2 + 3x + 8}$
    \end{enumerate}
\end{Exemple}
\newpage
\def\points{6}
\def\rdifficulty{2.5}
\begin{EXO}{Développer et réduire des expressions}{C4L22}
    \vspace{-0.25cm}
    \begin{multicols}{2}
        \acc{Développer et réduire} les expressions suivantes :
        \begin{enumerate}
            \item $A=7(x+2)+6(x+3)\\
             A = \repsim[5cm]{7x + 7\times 2 + 6x + 6 \times 3}\\
             A = \repsim[5cm]{13x + 14 + 18}\\
             A = \repsim[3cm]{13x + 32}$

            \columnbreak


            \item $B=-2(-x+3) + 2(x-5)\\
             B = \repsim[5cm]{2x -6 + 2x - 10}\\
             B = \repsim[3cm]{4x-16}$
            \item $C= 7-2(x-2) = \repsim[5cm]{7 - 2x + 4}\\
             C = \repsim[3cm]{-2x + 11}$
        \end{enumerate}
    \end{multicols}
\end{EXO}  

\section{Factorisation}

\begin{Definition}
    \voc{Factoriser}, c’est transformer une \voc{somme} (ou \voc{différence}) en \acc{produit}.\\
    Une expression factorisée est formée de \voc{facteurs}.\\
    Dans la pratique, \acc{factoriser} c’est lire la formule de distributivité \acc{de la droite vers la gauche} : \\
    \begin{minipage}[t]{0.475\textwidth}
        \begin{center}\Large$\blue{a}\red{b}+\blue{a}\red{c}=\blue{a}(\red{b}+\red{c})$\end{center}
    \end{minipage}
    \hfill
    \begin{minipage}[t]{0.475\textwidth}
        \begin{center}\Large$\blue{a}\red{b}-\blue{a}\red{c}=\blue{a}(\red{b}-\red{c})$\end{center}
    \end{minipage} 
\end{Definition}

\begin{Exemple}
    \bclampe \includegraphics[]{images/Factorisation.jpg}
    \acc{Factoriser} les expressions suivantes puis les \voc{simplifier} le plus possible :\\
\end{Exemple}

\begin{EXO}{Factoriser des expressions}{C4L26}
    \begin{multicols}{2}
    \begin{enumerate}
        \item $A=131\times13 + 131\times87\\A=\repsim[5cm]{131 \times (13  + 87)}\\A =\repsim[3cm]{131 \times 100} = \repsim[2.5cm]{13\,100}$
        \item $B=37\times13-37\times3\\B=\repsim[5cm]{37 \times (13  - 3)}\\B =\repsim[3cm]{37 \times 10} = \repsim[2.5cm]{370}$
        \item $C=4x-4\times 5 =\repsim[5cm]{4 \times (x  -5)}$
        \item $D=24-8x =\repsim[5cm]{8 \times (3  - x)}$
        \item $E=7x+42 =\repsim[5cm]{7 \times (x + 6)}$
        \item $F=3x-3 = \repsim[5cm]{3 \times (x  - 1)}$
        \item $G=x^{2}+3x =\repsim[5cm]{x \times (x  + 3 )}$
        \item $H=3x^{2}+6x =\repsim[5cm]{3x \times (x  + 2)}$
    \end{enumerate}
\end{multicols}
\end{EXO}

\newpage
%\vspace{-1.3cm}
\section{Distributivité double}
\begin{Propriete}[Double distributivité]
    Lorsqu'on utilise la distributivité et que les deux facteurs sont des sommes ou des différences de plusieurs termes, il est \acc{utile} de connaître l'égalité suivante : \\
    \[\Large(a+b)(c+d)=ac+ad+bc+bd\]
\end{Propriete}
\begin{Demonstration}
    \begin{crep}
        \begin{minipage}[t]{0.475\textwidth}
            Notons $m = a + b$. \\
            On a alors : $(a +b)(c + d) = m (c + d)$\\
            On applique la \acc{distributivité simple} au membre de droite de l'égalité :\\
            \[\blue{m} (c + d) = \blue{m} \times c + \blue{m} \times d
            = \underbrace{\blue{( a + b )} \times c}_{\text{terme 1}} + \underbrace{\blue{( a + b )} \times d}_{\text{terme 2}}\]\\
            
        \end{minipage}
        \hfill
        \begin{minipage}[t]{0.475\textwidth}
            On développe une nouvelle fois en utilisant la distributivité pour les termes $( a + b ) \times c$ et $( a + b ) \times d$ :\\
            $
                \red{( a + b ) \times c} + \blue{( a + b ) \times d} \\= \red{a \times c + b \times c} + \blue{a \times d + b \times d}
            $\\
            Finalement, on a montré que :\\
            $(a +b)(c + d) = a \times c + b \times c + a \times d + b \times d$
        \end{minipage}
    \end{crep}
\end{Demonstration}
\vspace{-0.4cm}\begin{Remarque}
    \bclampe Comme pour la formule de distributivité simple, il est possible de lire ces formules :
    \begin{itemize}
        \item De la gauche vers la droite pour \acc{développer}.
        \item De la droite vers la gauche pour \acc{factoriser}.
    \end{itemize}
    On a aussi les formules suivantes, qu'il est possible de démontrer en utilisant \acc{les règles des signes}.
    \vspace{-0.65cm}
    \begin{multicols}{2}
        \begin{enumerate}
            \item $(a-b)(c+d)=ac+ad-bc-bd$
            \item $(a-b)(c-d)=ac-ad-bc+bd$
        \end{enumerate}
    \end{multicols}
\end{Remarque}
\def\points{8}
\def\rdifficulty{3}
\vspace{-0.5cm}\begin{EXO}{Utiliser la double distributivité}{C4L24}
    \acc{Développe et réduis} les expressions suivantes :\\
    \vspace{-0.75cm}\begin{multicols}{2}
        \begin{enumerate}
            \item $A = (2x+3)(x+8)\\
            A = \repsim[8cm]{2x\times x + 2x \times 8 + 3 \times x + 3 \times 8}\\
            A = \repsim[8cm]{2x^2 + 19x + 24}$\\
            \item $B = (-3+x)(4-5x)\\
            B = \repsim[8cm]{-3\times 4 + (-3) \times (-5x) + x \times 4 + x \times (-5x)}\\
            B = \repsim[8cm]{-5x^2 + 19x - 12}$
        \end{enumerate}
    \end{multicols}
    \vspace{-0.75cm}\begin{multicols}{2}
        \begin{enumerate}[start=3]
            \item $C = 2(3+x)(3-2x)\\
            C = \repsim[8cm]{(6+2x)(3-2x)}\\
            C = \repsim[8cm]{6\times 3 + 6 \times (-2x) + 2x \times 3 + 2x \times (-2x)}\\
            C = \repsim[8cm]{-4x^2 - 6x + 18}$\\
            \item $D = 2x(1-x)-(x-3)(3x+2)\\
            D = \repsim[8cm]{{\red{2x - 2x^2}} - ( {\blue{3x^2 + 2x - 9x - 6}})}\\
            D = \repsim[8cm]{-2x^2 + 2x - 3x^2 - 2x + 9x + 6}\\
            D = \repsim[8cm]{-5x^2 + 9x + 6}$\\
        \end{enumerate}
    \end{multicols}
\end{EXO}% : \section{Expression littérale}
\begin{Definition}[Expression littérale]
    Une \voc{expression littérale} est une expression mathématique contenant une ou plusieurs lettres qui désignent des nombres.\\

    Les lettres sont appelées des \voc{variables}.
\end{Definition}
\begin{Exemple}[Expression numérique]
    \begin{multicols}{2}
        \[A=5+7=12\]
        \[B=(-5)\times 7=-35\]
        \[C=\dfrac{5}{2}-\dfrac{2}{3}=\dfrac{5\times 3}{2\times 3}- \dfrac{2\times 2}{3\times 2}=\dfrac{15}{6}-\dfrac{4}{6}=\dfrac{11}{6}\]
        \[D=10^{2}\times 10^{5}=10^{7}\]
    
    
    \columnbreak

        Toutes ces expressions sont composées uniquement de \acc{nombres}, ce sont des \acc{expressions numériques}
    \end{multicols}
\end{Exemple}
\begin{Exemple}[Expression littérale]
    \begin{multicols}{2}
    \begin{enumerate}
        \item La longueur $c$ d’un cercle de rayon $r$ est donnée par : \\
        $c = 2\times {\color{red}\pi} \times \color{blue}r$ 
        où $\color{red}\pi\approx \color{red}3{,}14$…
        \begin{center}
            \figureLongueurCercle
        \end{center}
        Cette formule comporte \acc{une variable}
        \columnbreak

        \item L’aire d’un carré est donné par $\mathbf{\color{blue}c\times c}$
        où $\mathbf{\color{blue}c}$ représente le \acc{côté du carré}.
        \vspace{-0.8cm}\begin{center}
            \figureAireCarre
        \end{center}
    \end{enumerate}
    
    
    \end{multicols}
    \begin{enumerate}[start=3]
        \item Dans l'égalité $6x+6y+7z=21$, il y a \acc{trois variables} : $x ; y \text{ et } z$.\\
    \end{enumerate}
    \solEquation
\end{Exemple}
\newpage
\section{Valeur d’une expression littérale}

\subsection{\'Evaluer une expression littérale}

\begin{Definition}[\'Evaluer une expression littérale]
    \voc{\'Evaluer une expression} littérale signifie \acc{attribuer} une \acc{valeur numérique} à une ou plusieurs \acc{variables}.

    Il s'agit ensuite de \acc{remplacer} ces variables par les valeurs numériques, puis d'\acc{effectuer les calculs} rendus possibles.\\
\end{Definition}

\begin{Methode}
    \begin{minipage}[t]{0.475\textwidth}
        \'Evaluer l'expression suivante pour $x = 5$
        \[A = 2x + t + 3\]

        \textbf{Solution :}\\
        On \acc{remplace} les variables $x$ par $5$ :\\
        \[A = 2\times 5 + t + 3 \quad \text{t n'est pas évaluée}\]
        \[A = 10 + t + 3\]
        \[A = 13 + t\]
    \end{minipage}
    \hfill
    \begin{minipage}[t]{0.475\textwidth}
        Evaluer l'expression suivante pour $x = -1$

        \[B = x + (1 - x) + 3x\]

        \textbf{Solution :}\\
        On \acc{remplace} les variables $x$ par $-1$ :\\
        \[B = -1 + (1 - (-1)) + 3\times ( -1 )\]
        \[B = -1 + (1 + 1) - 3\]
        \[B = -1 + 2 - 3\]
        \[B = -2\]
    \end{minipage}
\end{Methode}

\begin{Remarque}
    \bcattention Penser à rajouter les signes $\times$ lorsqu'on remplace une variable qui est multipliée à un nombre.\\
    \bcattention Les variables ayant des valeurs négatives peuvent être placées \acc{entre parenthèses} pour éviter les erreurs de calcul.
\end{Remarque}

\def\points{8}
\begin{EXO}{\'Evaluer une expression littérale}{C4L13}
    Calculer la valeur des expressions suivantes pour les valeurs données :
    \begin{enumerate}
        \begin{minipage}{0.4\textwidth}\item $A=6\times (x+3)$ lorsque $x=5$
            \vspace{-0.25cm}\begin{crep}
                \[A = 6\times (5+3)\]
                \[A = 6\times 8\]
                \[A = 48\]
            \end{crep}
        \end{minipage}
        \hfill
        \begin{minipage}{0.475\textwidth}\item $B=l\times L$ lorsque $l=3{,}5$ et $L=7$
            \vspace{-0.25cm}\begin{crep}
                \[B = l\times L\]
                \[B = 3{,}5\times 7\]
                \[B = 24{,}5\]
            \end{crep}
        \end{minipage}
        \begin{minipage}{0.4\textwidth}\item $C=5\times(6-x) + 3x -7y$ lorsque $x=2$ et $y=1$
            \vspace{-0.25cm}\begin{crep}
                \[C = 5\times(6-2) + 3\times 2 -7\times 1 \]
                \[C = 5\times(4) + 6 -7 \]
            \end{crep}
        \end{minipage}
        \hfill
        \begin{minipage}{0.475\textwidth}
            \begin{crep}
                \[C = 20 - 1\]
                \[C = 19\]
            \end{crep}
        \end{minipage}
    \end{enumerate}
    
\end{EXO}

\subsection{Nature d'une égalité}
\begin{Definition}
    Une égalité est constituée de deux expressions mathématiques appelées « \voc{membres} » séparées par un signe « = ».
\end{Definition}
\begin{Vocabulaire}[Nature d'une égalité]

    \begin{center}
        \begin{tcbtab}[Une égalité peut être :]{c|c|c|c}
            \voc{Nature de l'égalité}& \acc{Vraie} & \acc{Fausse} & \acc{Parfois vraie, parfois fausse} \\
            \hline
            \acc{Exemple} 1 & $3^2 = (-3)^2$ & $\dfrac{1}{3}=0{,}33$ & $2x=10$ \\
            \acc{Exemple} 2 & $x+x=2x$ & $x^2=-1$ & $3x+1=5x-4$ 
        \end{tcbtab}
    \end{center}
\end{Vocabulaire}
\begin{Exemple}
    \vspace{-0.25cm}\begin{enumerate}
    \item L'égalité $5\times 2=6+4$ est \repsim[3cm]{vraie}, car \repsim[6cm]{$5\times 2=10$ et $6+4=10$}.\\
    \item L'égalité $4\times 6=24+3$ est \repsim[3cm]{fausse}, car \repsim[6cm]{$4\times 6=24$ mais $24+3=27$}.
    \item L'égalité $4x + 6 +2x = 2x\times 3 +2\times 3$ est \repsim[3cm]{vraie} car :
        \begin{crep}
            \begin{minipage}[t]{0.475\textwidth}
                D'une part : \\
                $4x + 6 + 2x = 6x + 6$
            \end{minipage}
            \hfill
            \begin{minipage}[t]{0.475\textwidth}
                D'autre part :\\
                $2x \times 3 + 2 \times 3 = 6x + 6$
            \end{minipage}\\\\
            Les membres de gauche et de droite sont tous les deux égaux à la même \acc{expression littérale} $6x + 6$.\\
            On en déduit que cette égalité est vraie quelle que soit la valeur de $x$.
        \end{crep}
    \end{enumerate}
\end{Exemple}
\def\points{6}
\def\rdifficulty{2}
\begin{EXO}{Déterminer la nature d'une égalité}{C4L14}
    \vspace{-0.25cm}\begin{enumerate}
    $3x+6=2(x+5)$ est \repsim[3cm]{fausse} car :
    \begin{crep}
        \begin{minipage}[t]{0.475\textwidth}
            D'une part : \\
            $2(x + 5) = 2x + 10$
        \end{minipage}
        \hfill
        \begin{minipage}[t]{0.475\textwidth}
            D'autre part :\\
            $3x + 6 \neq 2x + 10$
        \end{minipage}\\\\
        Les deux membres ne sont pas égaux.
    \end{crep}

    \vspace{-0.25cm}$x^{2}=2x$ est \repsim[3cm]{fausse} car :
    \begin{crep}
        Les deux membres ne sont pas égaux pour toutes les valeurs de $x$ : \\
        Si x = 3 par exemple, alors : $x^2 = 3^2 = 9$ mais $2x = 2\times 3 =6$
    \end{crep}
    \end{enumerate}
\end{EXO}
\vspace{-0.25cm}\begin{Remarque}
    Parfois ces égalités, par exemple $3x+5=7$ ou $4x+4=7x+2$, peuvent être égales pour certaines valeurs de $x$, on parle d'équation.
\end{Remarque}

\newpage
\section{Développement}
\subsection{Introduction}
\begin{Activite}[Les bouteilles]
    Un restaurateur a commandé 3 caisses de jus d’orange et 5 caisses de jus de raisin.\\
    Chaque caisse contient 24 bouteilles de jus.\\
    \begin{enumerate}
        \item Répondre à la question suivante de \acc{deux façons différentes} :\\
            \textbf{Combien a-t-il commandé de bouteilles en tout ?}
        \item Que remarque-t-on ?
    \end{enumerate}
    \tcblower
    \begin{enumerate}
        \item \begin{minipage}[t]{0.475\textwidth}
            \acc{Première solution :}
            \begin{crep}
                Le restaurateur a commandé $3\times\blue{24}=72$ bouteilles de jus d'orange.\\
                Il a également commandé $5\times\blue{24}=120$ bouteilles de jus de raisin.\\
                Le nombre total de bouteilles s'écrit :\\
                $3\times \blue{24} + 5 \times \blue{24} = 72 + 120 = 192$ bouteilles.
            \end{crep}
        \end{minipage}
        \hfill
        \begin{minipage}[t]{0.475\textwidth}
            \acc{Seconde solution :}
            \begin{crep}
                Puisque les caisses contiennent le même nombre de bouteilles, il suffit de \acc{multiplier} le nombre de caisses \acc{au total} par le nombre de \acc{bouteilles} par caisse : \\
                $( 3 + 5 ) \times \blue{24} = 8 \times \blue{24} = 192$ bouteilles.
            \end{crep}
        \end{minipage}

        \item On remarque que...\begin{crep}
            L'égalité suivante est vraie : \\
            $( 3 + 5 ) \times \blue{24} = 3\times \blue{24} + 5 \times \blue{24}$
        \end{crep}
    \end{enumerate}
    
\end{Activite}
\subsection{Distributivité}
\begin{Propriete}[Distributivité simple]
    On considère trois expressions ( numériques ou littérales ) a, b et c. \\
    Les égalités suivantes sont \acc{toujours vérifiées} :\\
    \begin{minipage}[t]{0.475\textwidth}
        \begin{center}\Large$\blue{a}(\red{b}+\red{c})=\blue{a}\red{b}+\blue{a}\red{c}$\end{center}
    \end{minipage}
    \hfill
    \begin{minipage}[t]{0.475\textwidth}
        \begin{center}\Large$\blue{a}(\red{b}-\red{c})=\blue{a}\red{b}-\blue{a}\red{c}$\end{center}
    \end{minipage} 
\end{Propriete}

\begin{Exemple}[Calculs astucieux]
    \vspace{-0.35cm}
    \begin{minipage}[t]{0.3\textwidth}\bclampe Pour calculer astucieusement, on peut utiliser la \voc{distributivité}.\end{minipage}
    \hfill
    \begin{minipage}[t]{0.6\textwidth}
    Puisque $101 = 100 + 1$, on peut écrire :
    \[\blue{32} \times 101 = \blue{32} \times (\red{100} + \red{1}) = \blue{32} \times \red{100} + \blue{32} \times \red{1} = 3\,200 + 32 = \color{\currentAccentColor}{3\,232}\]
    \end{minipage}\\

    %\acc{Calculer astucieusement :}\\
    $32\times 99=\tcfillcrep{\blue{32} \times (\red{100} - \red{1}) = \blue{32} \times \red{100} - \blue{32} \times \red{1} = 3\,200 - 32 = 3\,168}$\\
    $13\times 102=\tcfillcrep{\blue{13} \times (\red{100} + \red{2}) = \blue{13} \times \red{100} + \blue{13} \times \red{2} = 1\,300 + 26 = 1\,326}$\\
    $29\times 999=\tcfillcrep{\blue{29} \times (\red{1\,000} - \red{1}) = \blue{29} \times \red{1\,000} - \blue{29} \times \red{1} = 29\,000 - 29 = 28\,971}$
\end{Exemple}
\begin{Definition}[Développer une expression]
    \voc{Développer}, c’est transformer un \acc{produit} en \acc{somme} (ou \acc{différence}).\\
    Dans la pratique, développer c’est lire la formule de distributivité \acc{de la gauche vers la droite}.
\end{Definition}
\begin{Exemple}
    \bclampe Pour \acc{développer} une expression, il suffit de lire la formule de distributivité \acc{de la gauche vers la droite}.\\
    \textbf{L'expression} $A = 4(5+x)$ est un \voc{produit}. \\
    On peut le développer en : 
    $A = \blue{4}(\red{5}+\red{x}) = \blue{4}\times\red{5}+\blue{4}\times\red{x}$
\end{Exemple}

\def\points{4}
\def\rdifficulty{1}
\begin{EXO}{Développer avec la simple distributivité}{C4L21}
    Développe les expressions suivantes :
    \vspace{-0.35cm}\begin{multicols}{2}
        \begin{enumerate}
            \item $A = 5(x-2)$\begin{crep}
            $A = \blue{5}(\red{x}-\red{2}) \\= \blue{5}\times\red{x}-\blue{5}\times\red{2}\\ = 5x - 10$
            \end{crep}
            \item $B = -6(-2x+4)$\begin{crep}
            $B = \blue{-6}(\red{-2x}+\red{4}) \\= \blue{-6}\times\red{(-2x)}+\blue{(-6)}\times\red{4} \\= 12x - 24$
            \end{crep}
        \end{enumerate}
    \end{multicols}
    \begin{multicols}{2}
        \begin{enumerate}[start=3]
            \item $C = -x(2-3x)$\begin{crep}
            $C = \blue{-x}(\red{2}-\red{3x}) \\= \blue{-x}\times\red{2}-\blue{-x}\times\red{(-3x)} \\= -2x - 3x^2$
            \end{crep}
            \item $D = -(5-x)$\begin{crep}
            $D = \blue{-}(\red{5}-\red{x}) \\= \blue{-1}\times\red{5}-\blue{(-1)}\times\red{x} \\= -5 + x$
            \end{crep}
        \end{enumerate}
    \end{multicols}
\end{EXO}

\subsection{Réduire une expression}
\begin{Definition}[Réduire une expression]
    \voc{Réduire} une expression, c’est l’écrire avec le moins de termes ou de facteurs possibles. \\
    Pour cela on \acc{regroupe} les termes de \acc{même nature}. 
\end{Definition}
\begin{Exemple}
    \vspace{-0.25cm}
    \begin{multicols}{2}
        \acc{Réduire} les expressions suivantes :
        \begin{enumerate}
            \item $A = 4x+3x = \repsim[3cm]{7x}$
            \item $B = 2a+4-3a+6-2a+8a-8\\B = \repsim{5}\times a + \repsim{2}$
        \end{enumerate}
    \end{multicols}
    \vspace{-0.25cm}
    \begin{enumerate}[start=3]
        \item $C= x^{2}+8x-7-8x+15-2x^{2}+3x = \repsim[5cm]{-x^2 + 3x + 8}$
    \end{enumerate}
\end{Exemple}
\newpage
\def\points{6}
\def\rdifficulty{2.5}
\begin{EXO}{Développer et réduire des expressions}{C4L22}
    \vspace{-0.25cm}
    \begin{multicols}{2}
        \acc{Développer et réduire} les expressions suivantes :
        \begin{enumerate}
            \item $A=7(x+2)+6(x+3)\\
             A = \repsim[5cm]{7x + 7\times 2 + 6x + 6 \times 3}\\
             A = \repsim[5cm]{13x + 14 + 18}\\
             A = \repsim[3cm]{13x + 32}$

            \columnbreak


            \item $B=-2(-x+3) + 2(x-5)\\
             B = \repsim[5cm]{2x -6 + 2x - 10}\\
             B = \repsim[3cm]{4x-16}$
            \item $C= 7-2(x-2) = \repsim[5cm]{7 - 2x + 4}\\
             C = \repsim[3cm]{-2x + 11}$
        \end{enumerate}
    \end{multicols}
\end{EXO}  

\section{Factorisation}

\begin{Definition}
    \voc{Factoriser}, c’est transformer une \voc{somme} (ou \voc{différence}) en \acc{produit}.\\
    Une expression factorisée est formée de \voc{facteurs}.\\
    Dans la pratique, \acc{factoriser} c’est lire la formule de distributivité \acc{de la droite vers la gauche} : \\
    \begin{minipage}[t]{0.475\textwidth}
        \begin{center}\Large$\blue{a}\red{b}+\blue{a}\red{c}=\blue{a}(\red{b}+\red{c})$\end{center}
    \end{minipage}
    \hfill
    \begin{minipage}[t]{0.475\textwidth}
        \begin{center}\Large$\blue{a}\red{b}-\blue{a}\red{c}=\blue{a}(\red{b}-\red{c})$\end{center}
    \end{minipage} 
\end{Definition}

\begin{Exemple}
    \bclampe \includegraphics[]{images/Factorisation.jpg}
    \acc{Factoriser} les expressions suivantes puis les \voc{simplifier} le plus possible :\\
\end{Exemple}

\begin{EXO}{Factoriser des expressions}{C4L26}
    \begin{multicols}{2}
    \begin{enumerate}
        \item $A=131\times13 + 131\times87\\A=\repsim[5cm]{131 \times (13  + 87)}\\A =\repsim[3cm]{131 \times 100} = \repsim[2.5cm]{13\,100}$
        \item $B=37\times13-37\times3\\B=\repsim[5cm]{37 \times (13  - 3)}\\B =\repsim[3cm]{37 \times 10} = \repsim[2.5cm]{370}$
        \item $C=4x-4\times 5 =\repsim[5cm]{4 \times (x  -5)}$
        \item $D=24-8x =\repsim[5cm]{8 \times (3  - x)}$
        \item $E=7x+42 =\repsim[5cm]{7 \times (x + 6)}$
        \item $F=3x-3 = \repsim[5cm]{3 \times (x  - 1)}$
        \item $G=x^{2}+3x =\repsim[5cm]{x \times (x  + 3 )}$
        \item $H=3x^{2}+6x =\repsim[5cm]{3x \times (x  + 2)}$
    \end{enumerate}
\end{multicols}
\end{EXO}

\newpage
%\vspace{-1.3cm}
\section{Distributivité double}
\begin{Propriete}[Double distributivité]
    Lorsqu'on utilise la distributivité et que les deux facteurs sont des sommes ou des différences de plusieurs termes, il est \acc{utile} de connaître l'égalité suivante : \\
    \[\Large(a+b)(c+d)=ac+ad+bc+bd\]
\end{Propriete}
\begin{Demonstration}
    \begin{crep}
        \begin{minipage}[t]{0.475\textwidth}
            Notons $m = a + b$. \\
            On a alors : $(a +b)(c + d) = m (c + d)$\\
            On applique la \acc{distributivité simple} au membre de droite de l'égalité :\\
            \[\blue{m} (c + d) = \blue{m} \times c + \blue{m} \times d
            = \underbrace{\blue{( a + b )} \times c}_{\text{terme 1}} + \underbrace{\blue{( a + b )} \times d}_{\text{terme 2}}\]\\
            
        \end{minipage}
        \hfill
        \begin{minipage}[t]{0.475\textwidth}
            On développe une nouvelle fois en utilisant la distributivité pour les termes $( a + b ) \times c$ et $( a + b ) \times d$ :\\
            $
                \red{( a + b ) \times c} + \blue{( a + b ) \times d} \\= \red{a \times c + b \times c} + \blue{a \times d + b \times d}
            $\\
            Finalement, on a montré que :\\
            $(a +b)(c + d) = a \times c + b \times c + a \times d + b \times d$
        \end{minipage}
    \end{crep}
\end{Demonstration}
\vspace{-0.4cm}\begin{Remarque}
    \bclampe Comme pour la formule de distributivité simple, il est possible de lire ces formules :
    \begin{itemize}
        \item De la gauche vers la droite pour \acc{développer}.
        \item De la droite vers la gauche pour \acc{factoriser}.
    \end{itemize}
    On a aussi les formules suivantes, qu'il est possible de démontrer en utilisant \acc{les règles des signes}.
    \vspace{-0.65cm}
    \begin{multicols}{2}
        \begin{enumerate}
            \item $(a-b)(c+d)=ac+ad-bc-bd$
            \item $(a-b)(c-d)=ac-ad-bc+bd$
        \end{enumerate}
    \end{multicols}
\end{Remarque}
\def\points{8}
\def\rdifficulty{3}
\vspace{-0.5cm}\begin{EXO}{Utiliser la double distributivité}{C4L24}
    \acc{Développe et réduis} les expressions suivantes :\\
    \vspace{-0.75cm}\begin{multicols}{2}
        \begin{enumerate}
            \item $A = (2x+3)(x+8)\\
            A = \repsim[8cm]{2x\times x + 2x \times 8 + 3 \times x + 3 \times 8}\\
            A = \repsim[8cm]{2x^2 + 19x + 24}$\\
            \item $B = (-3+x)(4-5x)\\
            B = \repsim[8cm]{-3\times 4 + (-3) \times (-5x) + x \times 4 + x \times (-5x)}\\
            B = \repsim[8cm]{-5x^2 + 19x - 12}$
        \end{enumerate}
    \end{multicols}
    \vspace{-0.75cm}\begin{multicols}{2}
        \begin{enumerate}[start=3]
            \item $C = 2(3+x)(3-2x)\\
            C = \repsim[8cm]{(6+2x)(3-2x)}\\
            C = \repsim[8cm]{6\times 3 + 6 \times (-2x) + 2x \times 3 + 2x \times (-2x)}\\
            C = \repsim[8cm]{-4x^2 - 6x + 18}$\\
            \item $D = 2x(1-x)-(x-3)(3x+2)\\
            D = \repsim[8cm]{{\red{2x - 2x^2}} - ( {\blue{3x^2 + 2x - 9x - 6}})}\\
            D = \repsim[8cm]{-2x^2 + 2x - 3x^2 - 2x + 9x + 6}\\
            D = \repsim[8cm]{-5x^2 + 9x + 6}$\\
        \end{enumerate}
    \end{multicols}
\end{EXO},



\end{document}