% Configuration des cartes auto-correctives

% Dimensions des cartes (4 par page en 2x2)
\newlength{\largeurcarte}
\newlength{\hauteurcarte}
\setlength{\largeurcarte}{9cm}  % Pour 2 colonnes sur A4
\setlength{\hauteurcarte}{13cm} % Pour 2 lignes sur A4

% Couleurs pour les corrections
\definecolor{modif}{RGB}{200,0,0}     % Rouge pour les modifications
\definecolor{reduc}{RGB}{0,150,0}     % Vert pour les réductions

% Style pour les cartes énoncés
\tcbset{
    carteenonce/.style={
        enhanced,
        sharp corners,
        width=\largeurcarte,
        height=\hauteurcarte,
        boxrule=3pt,
        colback=cartebleu!10!white,
        colframe=cartebleu,
        boxsep=0pt,
        top=0pt,
        bottom=0pt,
        left=0pt,
        right=0pt,
        overlay={
            % Numéro dans un cercle
            \node[circle, fill=white, draw=cartebleu, line width=2pt, 
                  minimum size=1.2cm, font=\Large\bfseries] 
                  at ([xshift=-7mm, yshift=-7mm]frame.north east) {#1};
            % Zone blanche pour le titre
            \fill[white, rounded corners=5pt] 
                ([xshift=2mm, yshift=-15mm]frame.north west) rectangle 
                ([xshift=-2mm, yshift=-25mm]frame.north east);
            \node[font=\large\bfseries, text=cartebleu] 
                at ([yshift=-20mm]frame.north) {Résoudre l'équation};
            % Zone blanche pour le contenu
            \fill[white, rounded corners=5pt] 
                ([xshift=2mm, yshift=2mm]frame.south west) rectangle 
                ([xshift=-2mm, yshift=-30mm]frame.north east);
        },
        halign=center,
        valign=center,
        fontupper=\Huge\bfseries,
        top=35mm,
        bottom=5mm,
        left=5mm,
        right=5mm
    }
}
% Style pour les cartes solutions
\tcbset{
    cartesolution/.style={
        enhanced,
        sharp corners,
        width=\largeurcarte,
        height=\hauteurcarte,
        boxrule=3pt,
        colback=cartevert!10!white,
        colframe=cartevert,
        boxsep=0pt,
        top=0pt,
        bottom=0pt,
        left=0pt,
        right=0pt,
        overlay={
            % Numéro dans un cercle
            \node[circle, fill=white, draw=cartevert, line width=2pt, 
                  minimum size=1.2cm, font=\Large\bfseries] 
                  at ([xshift=-7mm, yshift=-7mm]frame.north east) {#1};
            % Zone blanche pour le titre
            \fill[white, rounded corners=5pt] 
                ([xshift=2mm, yshift=-15mm]frame.north west) rectangle 
                ([xshift=-2mm, yshift=-25mm]frame.north east);
            \node[font=\large\bfseries, text=cartevert] 
                at ([yshift=-20mm]frame.north) {Solution};
            % Zone blanche pour le contenu
            \fill[white, rounded corners=5pt] 
                ([xshift=2mm, yshift=2mm]frame.south west) rectangle 
                ([xshift=-2mm, yshift=-35mm]frame.north east);
        },
        halign=center,
        valign=center,
        fontupper=\normalsize,
        top=40mm,
        bottom=5mm,
        left=8mm,
        right=8mm
    }
}

% Configuration du raster 2x2 (4 cartes par page)
\tcbset{
    rastercartes/.style={
        raster columns=2,
        raster rows=2,
        raster equal height,
        raster column skip=5mm,
        raster row skip=5mm,
        raster height=\textheight,
        raster width=\textwidth,
    }
}
% Environnement pour une page d'énoncés
\newenvironment{pageenonces}{%
    \begin{tcolorbox}[
        blanker,
        width=\textwidth,
        height=\textheight,
        left=0pt,
        right=0pt,
        top=0pt,
        bottom=0pt,
    ]
    \begin{tcbraster}[rastercartes]
}{%
    \end{tcbraster}
    \end{tcolorbox}
    \newpage
}

% Environnement pour une page de solutions
\newenvironment{pagesolutions}{%
    \begin{tcolorbox}[
        blanker,
        width=\textwidth,
        height=\textheight,
        left=0pt,
        right=0pt,
        top=0pt,
        bottom=0pt,
    ]
    \begin{tcbraster}[rastercartes]
}{%
    \end{tcbraster}
    \end{tcolorbox}
    \newpage
}

% Macro pour une équation centrée et grande
\newcommand{\eqcarte}[1]{%
    $\displaystyle #1$
}

% Macro pour la solution détaillée
\newcommand{\solcarte}[1]{%
    \begin{minipage}{0.9\textwidth}
        \centering
        #1
    \end{minipage}
}

% Commandes pour le code couleur dans les corrections
\newcommand{\modif}[1]{{\color{modif}#1}}  % Rouge pour modifications
\newcommand{\reduc}[1]{{\color{reduc}#1}}  % Vert pour réductions