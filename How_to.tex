\documentclass[a4paper,11pt,fleqn]{article}

\usepackage[left=1cm,right=0.5cm,top=0.5cm,bottom=2cm]{geometry}

\usepackage{bfcours}

\def\rdifficulty{1}
\setrdexo{%left skip=1cm,
display exotitle,
exo header = tcolorbox,
%display tags,
skin = bouyachakka,
lower ={box=crep},
display score,
display level,
save lower,
score=\points,
level=\rdifficulty,
overlay={\node[inner sep=0pt,
anchor=west,rotate=90, yshift=0.3cm]%,xshift=-3em], yshift=0.45cm
at (frame.south west) {\thetags[0]} ;}
]%obligatoire
}
\setrdcrep{seyes, correction=true, correction color=monrose, correction font = \large\bfseries}

\newcommand{\tikzinclude}[1]{%
    \stepcounter{tikzfigcounter}%
    \csname tikzfig#1\endcsname
}


\hypersetup{
    pdfauthor={R.Deschamps},
    pdfsubject={},
    pdfkeywords={},
    pdfproducer={LuaLaTeX},
    pdfcreator={Boum Factory}
}

%\newcommand{\vocref}[2]{\href{#1}{\color{monrose}#2}}
\begin{document}

\setcounter{pagecounter}{0}
\setcounter{ExoMA}{0}
\setcounter{prof}{1}

\chapitre[
    \textbf{BF}
    ]{
    Comment démarrer avec \LaTeX\ % theme
    }{
    Boum% type_etablissement
    }{
    Factory% nom_etablissement
    }{
    % supplement ( laisser à % si vide )
    }{
    Document explicatif :
    }

\section{Comment fonctionne \LaTeX\  ?}

\LaTeX\  est un langage de programmation développé par \vocref{https://fr.wikipedia.org/wiki/Donald_Knuth}{Donald knuth} dans les années 1980 qui permet de construire des documents pdf en gérant la \textbf{structure} du document de façon semi-automatique.\\

Le principe est simple : on crée des commandes, des environnements, des packages qui permettent d'obtenir de nombreuses fonctionnalités.\\

Le logiciel est gratuit et \textbf{open source} et dispose d'une \textbf{largre communauté} notamment scientifique. \\

N'importe quel document texte comportant l'extension \frquote{.tex} peut être considéré comme un fichier \LaTeX\ . \\

\subsection{Quelques liens}

\textbf{Les essentiels :}\\
\begin{itemize}[label = \bccrayon]
	\item Pour \textbf{télécharger un compilateur} \LaTeX\  : \vocref{https://miktex.org/download}{MikTeX}
	\item Le forum \LaTeX\  par excellence : \vocref{https://tex.stackexchange.com/}{LaTeX stack exchange}
	\item Le repo principal des packages en ligne. C'est là que l'on trouve la plupart des \textbf{documentations} : \vocref{https://ctan.org/}{CTAN}\\
		Je l'utilise surtout depuis un moteur de recherche externe : \frquote{<nom\_du\_package> CTAN}
	\item Pour télécharger mon package \frquote{bfcours} et ses à-côtés : \vocref{https://github.com/Romain1099/BFCours.git}{BFCours}
\end{itemize}

\textbf{Les non-moins importants :}\\

\begin{itemize}[label=\bccrayon]

	\item Document d'explications générales en \LaTeX\  : \vocref{https://tuteurs.ens.fr/logiciels/latex/}{site d'archives des tuteurs de l'ENS}.\\On n'a pas fait plus concis et complet pour prendre \LaTeX\  en main.
	\item Toutes les documentations de mon repo github : \\
		\begin{itemize}[label=\faPen]
			\item tcolorbox pour toutes les \textbf{boites} 
			\item Tikz-euclide pour les constructions géométriques. Il est tout de même bon de noter que GeoGebra permet l'export d'une figure comme code LaTeX - tikz.
			\item TikZ pour l'impatient $\longrightarrow$ TikZ est le module de \textbf{dessin} de \LaTeX\  par excellence. Des bases sont à \textbf{maîtriser} pour bien progresser en \LaTeX\ . 
			\item rdexo et rdcrep - packages pour l'enseignement ou la présentation de ressources. 
		\end{itemize}
	
\end{itemize}	

\section{Comment installer \LaTeX\  ?}

\subsection{Installation}

Aller sur la page de téléchargement de MikTeX et choisir la version \textbf{adaptée à votre système d'exploitation}.\\
\textbf{Cocher} l'option \textbf{installer les packages à la volée} ( on-the-fly ) pour permettre plus de souplesse dans les premières compilations.\\
\textbf{Décocher} l'option d'installation pour tous les utilisateurs. Cela rend plus simple l'utilisation de la console MikTeX.


\subsection{Setup du répertoire des packages locaux}

Suivre les étapes suivantes \textbf{une seule fois} :
\begin{enumerate}
	\item Coller le dossier \textbf{localtexmf} récupéré sur mon repo github \textbf{n'importe ou sur votre machine}. L'essentiel est qu'il reste à cet emplacement. \\
	\item Copier le chemin d'accès de ce dossier. \\
	\item Ouvrir la \frquote{console MikTeX} et aller au menu \frquote{Settings}.
	\item Aller dans l'onglet \frquote{Directories}.
	\item Appuyer sur le bouteon \frquote{+} et \textbf{coller} le chemin d'accès au dossier \textbf{localtexmf}.
	\item Confirmer les changements et quitter la console. 
\end{enumerate}

Parfait : vous pouvez utiliser le package bfcours et les packages de Régis Deleuze ( suite rd ) dans vos documents. 

\subsection{Première compilation}

Ouvrir le document \frquote{new\_document.tex}. \\
Dans l'application \textbf{TeXWork} qui s'ouvre, on accède au code source de la page qu'il faut \textbf{compiler}.\\

Pour cela, avec le package \textbf{bfcours} il est nécessaire d'utiliser le compilateur \frquote{LuaLaTeX} qui permet d'accéder à des programmes secondaires ( code en langage lua ).\\
Sélectionner le compilateur \frquote{LuaLaTeX} dans la barre de sélection en haut à gauche de l'écran. \\
Compiler ensuite votre premier document en cliquant sur le \textbf{triangle vert} en haut à gauche de l'écran. \\


\section{Mes habitudes}

Pour produire des documents de façon aisée, il est nécessaire de réfléchir à une structure des fichiers. \\

On pourra s'inspirer du fonctionnement de mes \frquote{notes de stages} en herboristerie : \vocref{https://github.com/Romain1099/Herboristerie-2024.git}{Stage Arsimed 2024}\\

L'IDE \textbf{VScode} offre de nombreuses extensions dédiées à \LaTeX\  à explorer. De simples modules d'autocomplétion permettront une bien meilleure expérience. \\

Extension \textbf{PDF Viewer} qui permet de lire des documents pdf directement dans VScode.\\
Il doit y avoir un module permettant de compiler directement des documents latex dans VScode mais je n'ai plus la référence en tete. 


\end{document}