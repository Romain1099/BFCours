\documentclass[a4paper,11pt,fleqn]{article}

\usepackage[left=1cm,right=0.5cm,top=0.5cm,bottom=2cm]{geometry}

\usepackage{bfcours}

\def\rdifficulty{1}
\setrdexo{%left skip=1cm,
display exotitle,
exo header = tcolorbox,
%display tags,
skin = bouyachakka,
lower ={box=crep},
display score,
display level,
save lower,
score=\points,
level=\rdifficulty,
overlay={\node[inner sep=0pt,
anchor=west,rotate=90, yshift=0.3cm]%,xshift=-3em], yshift=0.45cm
at (frame.south west) {\thetags[0]} ;}
]%obligatoire
}
\setrdcrep{seyes, correction=true, correction color=monrose, correction font = \large\bfseries}


\hypersetup{
    pdfauthor={R.Deschamps},
    pdfsubject={},
    pdfkeywords={},
    pdfproducer={LuaLaTeX},
    pdfcreator={Boum Factory}
}
\newcommand{\messageIntro}[1]{
    Rapport d'incident, \hfill #1\\
}
\renewcommand{\punition}[4]{
    Travail supplémentaire, à rendre pour #2.\\
    Ce travail consiste à :\\
    \begin{center}#4\end{center}
}
\newcommand{\retenue}[4]{
    Retenue d'une durée de #1.
    La retenue aura lieu #2 en salle #3 durant laquelle le travail suivant sera proposé :\\
    \begin{center}#4\end{center}
}
\begin{document}

\setcounter{pagecounter}{0}
\setcounter{ExoMA}{0}
\setcounter{prof}{0}
%Pour les overlay
\def\points{\phantom{AAA}}
\def\difficulty{\phantom{AAA}}
\chapitre[
    % niveau : $\mathbf{6^{\text{ème}}}$,$\mathbf{5^{\text{ème}}}$,$\mathbf{4^{\text{ème}}}$,$\mathbf{3^{\text{ème}}}$,$\mathbf{2^{\text{nde}}}$,$\mathbf{1^{\text{ère}}}$,$\mathbf{T^{\text{Le}}}$
    ]{
    % Auteur : commande(get_students_names())
    }{
    % type_etablissement : Collège,Lycée
    }{
    % nom_etablissement : Amadis Jamyn,Eugène Belgrand
    }{
    %
    }{
    % type_document : Rapport d'incident -
    }

\begin{minipage}[t]{0.3\textwidth}
    \boite{Lieu et date :}{
        \textbf{Lieu :} % Lieu : salle A1,cours de récréation,couloirs,réfectoire,dans les rangs,pendant la montée,
    
    
        \textbf{Date :} % Date
    }
\end{minipage}%
\hfill
\begin{minipage}[t]{0.65\textwidth}
    \boite{Eleves impliqués :}{
        \textbf{Auteur :} % Auteur



        \textbf{Victime :} % Victime : commande(get_students_names())



        \textbf{Témoins :} Les élèves présents dans la classe.
    }
\end{minipage}%


\boite{Description des faits :}{

\messageIntro{%
        % Date : commande(get_current_date())
    }
\vspace{2cm}
}

\boite{Réponse disciplinaire apportée :}{

\textbf{Nature de la réponse :} \% Reponse : punition, retenue
{
    % Duree_retenue : 1h
}{
    % Date_punition : Lundi, Mardi, Mercredi, Jeudi, Vendredi
}{
    % Salle_retenue : salle A1, salle de permanence
}{
    % Travail_a_faire : ,
}

\textbf{Explication de la punition :}



\vspace{2cm}
Je reste disponible par mail pour toute explication supplémentaire concernant la réponse disciplinaire apportée.
}
\end{document}