\subsection{Informations sur la frise}% Définition des constantes

\boite{Matériel :}{
    \vspace{-0.2cm}\begin{multicols}{2}
        \begin{itemize}[label=\bcoutil]
        \item Trois dés à $10$ faces ou un \href{https://www.ma-calculatrice.fr/des-virtuels-simulateur?nbre_des=3&de_couleurs_1=005BD4&de_min_1=0&de_max_1=9&de_couleurs_2=2CD400&de_min_2=0&de_max_2=9&de_couleurs_3=D46000&de_min_3=0&de_max_3=9&de_couleurs_4=000000&de_min_4=1&de_max_4=6&de_couleurs_5=000000&de_min_5=1&de_max_5=6&de_couleurs_6=000000&de_min_6=1&de_max_6=6&de_couleurs_7=000000&de_min_7=1&de_max_7=6&de_couleurs_8=000000&de_min_8=1&de_max_8=6&de_couleurs_9=000000&de_min_9=1&de_max_9=6&de_couleurs_10=000000&de_min_10=1&de_max_10=6&de_couleurs_11=000000&de_min_11=1&de_max_11=6&de_couleurs_12=000000&de_min_12=1&de_max_12=6}{simulateur}.
        \item Des \acc{curseurs} représentant chaque équipe.
        \item Des cartes \acc{défis}.
        \item Des cartes \acc{rôle}.
        
        \columnbreak

        \item Une calculatrice pour l'équipe d'arbitrage.
        \item Une \acc{frise graduée}.
        \item Une \acc{fiche équipe}.
        \item Un chronomètre : \phantom{a}\hfill \overlaychrono{15} \hfill \phantom{a}
        \end{itemize}
    \end{multicols}
}
\def\bandwidth{4.5} % Largeur d'une bande en cm
\def\bandheight{25} % Hauteur d'une bande en cm
\def\languette{0.8}
\def\echellegraphique{10}
\def\nombremax{50}
\edef\nombrepage{\fpeval{round(\nombremax/(3*2.5),0)}}
\def\reste{1}
\begin{multicols}{2}
\begin{itemize}[label=$\bullet$]
    \item La frise suivante est à découper. 
    \item Il suffit ensuite de coller les languettes, pour assembler la frise.
    \item La frise est composée de $\num{\fpeval{(\nombrepage-\reste)*3+2}}$ morceaux répartis sur $\num{\nombrepage}$ pages.
    \item Longueur totale : $\num{\fpeval{(\bandheight * (\nombrepage - \reste) * 3 +\bandheight * 2 )  / 100}}$ m.
\end{itemize}

\columnbreak

La répartition des \acc{événements} est générée comme suit : 
\begin{center}
    \begin{tcbtab}{c|c|c|c|c}
        Type & R & C & M & D \\
        \hline
        Probabilité & $\frac{1}{6}$ & $\frac{1}{6}$ & $\frac{1}{6}$ & $\frac{1}{2}$\\
    \end{tcbtab}
\end{center}
\end{multicols}

\begin{Exemple}[Fonctionnement d'un tour de jeu]
    \vspace{-0.3cm}\begin{multicols}{2}
        \begin{enumerate}[itemsep=0em]
            \item L'équipe commence son tour à l'\acc{abscisse} $\num{7.45}$.
            \item Les élèves obtiennent, \acc{dans l'ordre} $4$, $1$ et $6$. 
            \item Le responsable complète la \acc{fiche d'avancement} avec le nombre $\num{4.16}$.
            \item L'équipe complète le résultat par $\num{7.45} + \num{4.16} = \num{\fpeval{7.45+4.16}}$.
            \item Le résultat est validé par l'équipe arbitre qui a le droit d'utiliser la calculatrice. 
            \item L'équipe tombe sur une case notée \frquote{Défi}. \\
            L'équipe pioche une carte \frquote{défi} de difficulté $2$ et répond à la question correctement.
            \item Les arbitres valident la solution. L'équipe avance de $2$ unités supplémentaires et note cette étape dans la \acc{fiche d'avancement}.
        \end{enumerate}

    \end{multicols}

    \begin{tcbtab}[Extrait de la fiche d'avancement]{c|c|c|c|c|c}
        %$\phantom{\dfrac{\dfrac{1}{10}}{\dfrac{1}{10}}}$ Tour $\phantom{\dfrac{\dfrac{1}{10}}{\dfrac{1}{10}}}$ & $1$ & $\dfrac{1}{10}$ & $\dfrac{1}{100}$ & Nombre formé & Résultat \\
        Action \no & Dé \no 1 & Dé \no 2 & Dé \no 3 & Nombre formé & Résultat \\
        \hline
        \repsim[1cm]{}  & \repsim{X} & \repsim{X} & \repsim{X} & \repsim[3cm]{X} & \repsim[6cm]{$\num{7.45}$}\\
        \hline
        \repsim[1cm]{}  & \repsim{$4$} & \repsim{$1$} & \repsim{$6$} & \repsim[3cm]{$\num{4.16}$} & \repsim[6cm]{$\num{7.45} + \num{4.16} = \num{\fpeval{7.45+4.16}}$}\\
        \hline
        \repsim[1cm]{}  & \repsim{X} & \repsim{X} & \repsim{X} & \repsim[3cm]{2} & \repsim[6cm]{$\num{\fpeval{7.45+4.16}} + 2 = \num{\fpeval{7.45+4.16+2}}$}\\
        \hline
        
    \end{tcbtab}

\end{Exemple}

\newpage
