\mescartes{red}{Q.3.1}{$7 \times 7=$ $\ldots$}{S.3.1}{$7 \times 7={\color[HTML]{f15929}\boldsymbol{49}}$}%
\mescartes{red}{Q.3.2}{La moitié de $36$ est : \ldots}{S.3.2}{La moitié de $36$ est $36\div 2={\color[HTML]{f15929}\boldsymbol{18}}$.}%
\vspace{1cm}
\mescartes{red}{Q.3.3}{Complète : \\$19+\ldots =100$}{S.3.3}{$100-19={\color[HTML]{f15929}\boldsymbol{81}}$}%
\mescartes{red}{Q.3.4}{$3$ cahiers coûtent $9$\,\euro{}.\\                               $9$ cahiers coûtent $\ldots$\,\euro{}}{S.3.4}{$3$ cahiers coûtent $9$\,\euro{}.\\                      $3\times3=9$ cahiers coûtent $3\times9={\color[HTML]{f15929}\boldsymbol{27}}$\,\euro{}.}%       
\vspace{1cm}
\mescartes{red}{Q.3.5}{$2$ h $30$ min $=$ \\ $\ldots$ min}{S.3.5}{$2$ h $30$ min $=2\times 60+ 30$ min $={\color[HTML]{f15929}\boldsymbol{150}}$ min}%
\mescartes{red}{Q.3.6}{Quel est le nombre écrit sous le point d'interrogation ?\\\tikzinclude{gGnt}}{S.3.6}{Le nombre écrit sous le point d'interrogation est : ${\color[HTML]{f15929}\boldsymbol{90}}$.}%
\vspace{1cm}
\mescartes{red}{Q.3.7}{$32+19=$$\ldots$}{S.3.7}{$32+19=32+20-1=52-1={\color[HTML]{f15929}\boldsymbol{51}}$}%
\mescartes{red}{Q.3.8}{$18$ élèves se mettent par groupe de $3$. \\                       Il y a $\ldots$ groupes.}{S.3.8}{Le nombre de groupes est donné par $18\div 3={\color[HTML]{f15929}\boldsymbol{6}}$.}%
\vspace{1cm}
\mescartes{red}{Q.3.9}{Le tiers de $27$ est :  $\ldots$}{S.3.9}{Le tiers de $27$ est : $27\div 3={\color[HTML]{f15929}\boldsymbol{9}}$.}%       
\mescartes{red}{Q.3.10}{Complète :\\                            $4+9=\ldots+5$}{S.3.10}{Le nombre cherché est : $4+9-5={\color[HTML]{f15929}\boldsymbol{8}}$.}%
\vspace{1cm}
\mescartes{red}{Q.3.11}{$4{,}4\times 10=$$\ldots$}{S.3.11}{$4{,}4\times 10={\color[HTML]{f15929}\boldsymbol{44}}$}%
\mescartes{red}{Q.3.12}{Un film commence à $19$ h $35$ et se termine à $21$ h $15$.\\                     Combien de temps a duré le film ?}{S.3.12}{Pour aller à $20$ h, il faut $25$ min, et il faut ajouter $1$ heure et $15$ min pour arriver à $21$ h $15$, soit au total ${\color[HTML]{f15929}\boldsymbol{1}}$ h ${\color[HTML]{f15929}\boldsymbol{40}}$ min.}%
\vspace{1cm}
\mescartes{red}{Q.3.13}{Complète :\\$3=$$\ldots$ quarts}{S.3.13}{$3=\dfrac{12}{4}=12\times \dfrac{1}{4}$, donc ${\color[HTML]{f15929}\boldsymbol{12}}$ quarts $=3$.}%
\mescartes{red}{Q.3.14}{Ajoute $25$ min à $7$ h $50$ min.}{S.3.14}{Pour aller à $8$ h, il faut $10$ min, et il reste $15$ min à ajouter, ce qui donne ${\color[HTML]{f15929}\boldsymbol{8}}$ h et ${\color[HTML]{f15929}\boldsymbol{15}}$ min.}%
\vspace{1cm}
\mescartes{red}{Q.3.15}{Ajoute un dixième à $2{,}96$.}{S.3.15}{$1$ dixième $=0,1$, d'où $2{,}96+0,1 ={\color[HTML]{f15929}\boldsymbol{3{,}06}}$}%
\mescartes{red}{Q.3.16}{Yann a $30$ billes. Il a $8$ billes de moins que Lou.\\                                    Lou a $\ldots$ billes.}{S.3.16}{Yann a $8$ billes de moins que Lou, donc Lou en a $8$ de plus, soit $30+8={\color[HTML]{f15929}\boldsymbol{38}}$ billes.}%
\vspace{1cm}
\mescartes{red}{Q.3.17}{$0{,}2$ kg  $=$   $\ldots$ g}{S.3.17}{Comme $1$ kg $=1\,000$ g,  pour passer des "kg" au "g", on multiplie par $1\,000$.\\                      Comme : $0{,}2\times 1\,000 =200$, alors $0{,}2$ kg$={\color[HTML]{f15929}\boldsymbol{200}}$ g.}%
\mescartes{red}{Q.3.18}{Écris en chiffres : \\                            Deux-millions-deux-mille}{S.3.18}{Deux-millions-deux-mille-deux $=2\,000\,000  + 2\,000 + 2={\color[HTML]{f15929}\boldsymbol{2\,002\,000}}$.}%
\vspace{1cm}
\mescartes{red}{Q.3.19}{Complète : \\                           $10$ jours $=$$\ldots$ h}{S.3.19}{Dans une journée, il y a $24$ heures, donc dans $10$ jours, il y a $10\times 24={\color[HTML]{f15929}\boldsymbol{240}}$ heures.}%
\mescartes{red}{Q.3.20}{Complète : \\                     $9$ heures $=$$\ldots$ min}{S.3.20}{Dans une heure, il y a $60$ minutes, donc dans $9$ heures, il y a $9\times 60={\color[HTML]{f15929}\boldsymbol{540}}$ minutes.}%
\vspace{1cm}
\mescartes{red}{Q.3.21}{Combien faut-il de pièces de $10$ centimes pour avoir $5{,}80$\,\euro{}. \\}{S.3.21}{Il faut : $5{,}8\div 0,1=5{,}8\times 10={\color[HTML]{f15929}\boldsymbol{58}}$ pièces.}%
\mescartes{red}{Q.3.22}{$0{,}33+0{,}4=$$\ldots$}{S.3.22}{$0{,}33+0{,}4={\color[HTML]{f15929}\boldsymbol{0{,}73}}$}%
\vspace{1cm}
\mescartes{red}{Q.3.23}{Le double de $4{,}8$ est $\ldots$}{S.3.23}{Le double de $4{,}8$ est $2\times 4{,}8={\color[HTML]{f15929}\boldsymbol{9{,}6}}$.}%
\mescartes{red}{Q.3.24}{Compléter :\\ $ .... \times 6=36$}{S.3.24}{$ {\color[HTML]{f15929}\boldsymbol{6}} \times 6=36$}%
\vspace{1cm}
\mescartes{blue}{Q.2.1}{$9 \times 4$}{S.2.1}{$9 \times 4={\color[HTML]{f15929}\boldsymbol{36}}$}%
\mescartes{blue}{Q.2.2}{$36+29$}{S.2.2}{$36+29=36+30-1=66-1={\color[HTML]{f15929}\boldsymbol{65}}$}%
\vspace{1cm}
\mescartes{blue}{Q.2.3}{Combien y a-t-il de boules noires ? \\}{S.2.3}{Le nombre de boules noires est donné par : $8\times 3={\color[HTML]{f15929}\boldsymbol{24}}$.}%
\mescartes{blue}{Q.2.4}{La moitié de $42$}{S.2.4}{La moitié de $42$ est $42\div 2={\color[HTML]{f15929}\boldsymbol{21}}$.}%
\vspace{1cm}
\mescartes{blue}{Q.2.5}{Complète :  $\ldots \times \ldots =35$}{S.2.5}{Deux réponses possibles (avec des entiers) : \\         ${\color[HTML]{f15929}\boldsymbol{5}}\times {\color[HTML]{f15929}\boldsymbol{7}}=35$\\         ${\color[HTML]{f15929}\boldsymbol{1}}\times {\color[HTML]{f15929}\boldsymbol{35}}=35$}%
\mescartes{blue}{Q.2.6}{\Temps{;;;;35;}+ \Temps{;;;;40;}}{S.2.6}{De $40 \text{ min }$ pour aller à $1$ h, il faut $20$ min, et il reste $15$ min à ajouter.\\         On obtient  ${\color[HTML]{f15929}\boldsymbol{1}}$ h et ${\color[HTML]{f15929}\boldsymbol{15}}$ min.}%
\vspace{1cm}
\mescartes{blue}{Q.2.7}{Pour partager $30$ oeufs, combien de boites de  $6$ oeufs dois-je utiliser ?}{S.2.7}{Le nombre de boites est donné par $30\div 6={\color[HTML]{f15929}\boldsymbol{5}}$.}%
\mescartes{blue}{Q.2.8}{Écris en chiffres le nombre cinquante-deux-mille-sept.}{S.2.8}{cinquante-deux-mille-sept = $52\,000$ + 7 = ${\color[HTML]{f15929}\boldsymbol{52\,007}}$}%
\vspace{1cm}
\mescartes{blue}{Q.2.9}{Karole a $12$ ans. \\         Laurent a 5 ans de moins que Karole. Laurent a $\ldots$ ans}{S.2.9}{Puisque Laurent a 5 ans de moins que Karole, son âge est  : $12-5={\color[HTML]{f15929}\boldsymbol{7}}$ {\color[HTML]{f15929}ans}.}%
\mescartes{blue}{Q.2.10}{Donne l'écriture décimale de  $3\times 7$ centièmes.}{S.2.10}{$1$ centième $=0,01$, d'où $3\times 7$ centièmes $=3\times 7\times 0,01={\color[HTML]{f15929}\boldsymbol{0.21}}$.}%
\vspace{1cm}
\mescartes{blue}{Q.2.11}{Complète : \,\,\,  $1{,}8+\ldots =10$}{S.2.11}{Le nombre cherché est donné par : $10-1{,}8={\color[HTML]{f15929}\boldsymbol{8{,}2}}$.}%
\mescartes{blue}{Q.2.12}{Complète : \,\,\,  $405= \ldots$ dizaines  $\ldots$  unités}{S.2.12}{$405 = {\color[HTML]{f15929}\boldsymbol{40}}$ dizaines ${\color[HTML]{f15929}\boldsymbol{5}}$ unités}%
\vspace{1cm}
\mescartes{blue}{Q.2.13}{$25\div 5$}{S.2.13}{$25\div 5={\color[HTML]{f15929}\boldsymbol{5}}$}%
\mescartes{blue}{Q.2.14}{Si $2$ cahiers coûtent $8$\,\euro{}, alors $8$ cahiers coûtent  $\ldots$\,\euro{}.}{S.2.14}{$2$ cahiers coûtent $8$\,\euro{}.\\ $4\times2=8$ cahiers coûtent $4\times8={\color[HTML]{f15929}\boldsymbol{32}}$\,\euro{}.}%
\vspace{1cm}
\mescartes{blue}{Q.2.15}{$92\times 5$}{S.2.15}{$92\times 5=92\times 10 \div 2=920\div 2={\color[HTML]{f15929}\boldsymbol{460}}$}%
\mescartes{blue}{Q.2.16}{Dans $32$ combien de fois $4$ ?}{S.2.16}{Dans $32$, il y a ${\color[HTML]{f15929}\boldsymbol{8}}$ fois $4$ car $8\times 4=32$.}%
\vspace{1cm}
\mescartes{blue}{Q.2.17}{Complète : \,\,\, $7$ centaines et  $\ldots$  dizaines font  $740$.}{S.2.17}{$740 = 7$ centaines et ${\color[HTML]{f15929}\boldsymbol{4}}$ dizaines}%
\mescartes{blue}{Q.2.18}{Combien de dixièmes y a-t-il en tout dans $8{,}48$ ?}{S.2.18}{$8{,}48 = 8$ unités $4$ dixièmes $8$ centièmes.\\Or $1$ unité = $10$ dixièmes donc $8$ unités $= 80$ dixièmes.\\Finalement $8{,}48 = 84$ dixièmes $8$ centièmes.\\Il y a donc ${\color[HTML]{f15929}\boldsymbol{84}}$ dixièmes en tout dans $8{,}48$.}%
\vspace{1cm}
\mescartes{blue}{Q.2.19}{$1{,}93+ 0{,}8$}{S.2.19}{$1{,}93+ 0{,}8={\color[HTML]{f15929}\boldsymbol{2{,}73}}$}%
\mescartes{green}{Q.1.1}{ $22+13$ }{S.1.1}{$22+13=35$}%
\vspace{1cm}
\mescartes{green}{Q.1.2}{ $65-32$ }{S.1.2}{$65-32=33$}%
\mescartes{green}{Q.1.3}{ $17+25$ }{S.1.3}{$17+25=42$}%
\vspace{1cm}
\mescartes{green}{Q.1.4}{ $83-25$ }{S.1.4}{$83-25=58$}%
\mescartes{green}{Q.1.5}{ $3\times 1\,000 + 6\times 10 + 5\times 100$ }{S.1.5}{$3\times 1\,000 + 6\times 10 + 5\times 100 =3\,560$}%
\vspace{1cm}
\mescartes{green}{Q.1.6}{ $40\div5$ }{S.1.6}{$40\div5=8$}%
\mescartes{green}{Q.1.7}{ $2{,}2+3$ }{S.1.7}{$2{,}2+3=5{,}2$}%
\vspace{1cm}
\mescartes{green}{Q.1.8}{ $1{,}4+1{,}16$ }{S.1.8}{$1{,}4+1{,}16=2{,}56$}%
\mescartes{green}{Q.1.9}{ $5{,}63-2{,}2$ }{S.1.9}{$5{,}63-2{,}2=3{,}43$}%
\vspace{1cm}
\mescartes{green}{Q.1.10}{ $6{,}1-4{,}5$ }{S.1.10}{$6{,}1-4{,}5=1{,}6$}%
\mescartes{green}{Q.1.11}{ J'ai $18$ ans. Je suis $2$ fois plus âgé que Joachim.\\Quel âge a Joachim ? }{S.1.11}{L'âge de Joachim est : $18 \div 2=9$ ans.}%
\vspace{1cm}
\mescartes{green}{Q.1.12}{ Léa a $17$ ans. Sa sœur a $5$ ans.\\Quelle est leur différence d'âge ? }{S.1.12}{La différence d'âge entre Léa et sa sœur est : $17-5=12$ ans.}%
\mescartes{green}{Q.1.13}{ Joachim a couru $2$ séquences de $10$ minutes. Combien de minutes a-t-il couru en tout ? }{S.1.13}{Joachim a couru : $2 \times 10=20$ minutes.}%
\vspace{1cm}
\mescartes{green}{Q.1.14}{ $\ldots - 3=2{,}2$ }{S.1.14}{${\color[HTML]{f15929}\boldsymbol{5{,}2}} - 3=2{,}2$}%
\mescartes{green}{Q.1.15}{ $5 \times 6$ }{S.1.15}{$5 \times 6=30$}%
\vspace{1cm}
\mescartes{green}{Q.1.16}{ $6 \times 4$ }{S.1.16}{$6 \times 4=24$}%
\mescartes{green}{Q.1.17}{ On a coupé $4{,}1$ cm d'une ficelle qui en faisait $7{,}2$.\\Combien de centimètres en reste-t-il ? }{S.1.17}{$7{,}2-4{,}1=3{,}1$}%
\vspace{1cm}
\mescartes{green}{Q.1.18}{ \tikzinclude{TIsR}}{S.1.18}{${\color[HTML]{f15929}\boldsymbol{3{,}5}} + 2{,}4=5{,}9$}%
\mescartes{green}{Q.1.19}{ $8 \times 7$ }{S.1.19}{$7 \times 8=56$}%
\vspace{1cm}
\mescartes{green}{Q.1.20}{ $\ldots \times 20=190$ }{S.1.20}{Le périmètre mesure : $3 \times 5{,}3$ cm $=15{,}9$ cm.}%
\mescartes{green}{Q.1.21}{ $3$ kg de fraises coûtent $22{,}5$\,\euro{}, combien coûtent $12$ kg de fraises ? }{S.1.21}{${\color[HTML]{f15929}\boldsymbol{9{,}5}} \times 20=190$}%
\vspace{1cm}
\mescartes{green}{Q.1.22}{ $\ldots \times 4=40$ }{S.1.22}{$12$ kg de fraises coûtent : $22{,}5 \times 4 = 90$\,\euro{}.}%
\mescartes{green}{Q.1.23}{ Le diamètre d'un cercle de $60$ unités de rayon. }{S.1.23}{${\color[HTML]{f15929}\boldsymbol{10}} \times 4=40$}%     
\vspace{1cm}
\mescartes{green}{Q.1.24}{ Le film a commencé à $20$ h $30$. Il s'est terminé à $22$ h $25$.\\ Combien de minutes a-t-il duré ? }{S.1.24}{Le diamètre est le double du rayon : $2 \times 60 = 120$}%
\mescartes{green}{Q.1.25}{ En $24$ minutes, un manège fait $27$ tours.\\En $8$ minutes il fait \ldots tours. }{S.1.25}{Le film a duré $1$ h $55$ min soit $115$ minutes.}%
\vspace{1cm}