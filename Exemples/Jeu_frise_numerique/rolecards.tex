\subsection{Cartes de rôles}
\newcommand{\rolecard}[2]{%
  \begin{tcolorbox}[
    enhanced,
    width=7cm,
    colback=blue!10!white,
    colframe=blue!50!black,
    boxrule=1mm,
    arc=5mm,
    drop shadow={xshift=2mm,yshift=-2mm,opacity=0.5},
    fontupper=\small,
    fonttitle=\bfseries\Large,
    coltitle=white,
    colbacktitle=blue!50!black,
    title={#1},
    attach boxed title to top center={yshift=-3mm},
    boxed title style={
      sharp corners,
      boxrule=0mm,
      interior style={left color=blue!50!black, right color=blue!30!white}
    },
    overlay={%
      % Petit cercle en haut à gauche
      \draw[fill=yellow!80!orange, draw=yellow!50!red, line width=0.5mm] 
        ([xshift=-2mm,yshift=-2mm]frame.north west) circle (3mm);
      % Petit cercle en haut à droite
      \draw[fill=yellow!80!orange, draw=yellow!50!red, line width=0.5mm] 
        ([xshift=2mm,yshift=-2mm]frame.north east) circle (3mm);
      % Une étoile décorative centrée en haut
      \node[rotate=15, scale=1.2] at ([yshift=-1cm]frame.north) {\faStar};
    }
  ]
    #2
  \end{tcolorbox}
}



\begin{multicols}{2}
    \rolecard{Joueur 1}{
        \begin{center}\acc{Responsable des dés}\end{center}
        \begin{itemize}[label=\faPen]
            \item Lance trois dés à 10 faces au début de chaque tour.
            \item S'il les lance plusieurs fois d'affilée, le tour de l'équipe est annulé.
            \item Participe aux calculs en collaboration avec l'équipe.
            \item Participe à la résolution de problèmes.
        \end{itemize}
    }

    \rolecard{Joueur 2}{
        \begin{center}\acc{Responsable de la fiche}\end{center}
        \begin{itemize}[label=\faPen]
            \item Note le nombre formé en utilisant les valeurs obtenues avec les dés.
            \item Ajoute ce nombre au résultat précédent.
            \item Participe aux calculs en collaboration avec l'équipe.
            \item Participe à la résolution de problèmes.        
        \end{itemize}
    }
\end{multicols}

\vspace{3cm}

\begin{multicols}{2}

    \rolecard{Joueur 3}{
        \begin{center}\acc{Responsable du curseur}\end{center}
        \begin{itemize}[label=\faPen]
            \item Déplace le curseur sur la frise graduée en fonction du résultat.
            \item Vérifie si l'équipe atterrit sur un événement particulier.
            \item Participe aux calculs en collaboration avec l'équipe.
            \item Participe à la résolution de problèmes.
        \end{itemize}
    }

    \rolecard{Joueur 4}{
        \begin{center}\acc{Responsable de la communication}\end{center}
        \begin{itemize}[label=\faPen]
            \item Informe l'équipe des règles et des événements déclenchés.
            \item Vérifie et annonce les actions à réaliser en fonction des cases atteintes.
            \item Participe aux calculs en collaboration avec l'équipe.
            \item Participe à la résolution de problèmes.
        \end{itemize}
    }
\end{multicols}

\newpage

\begin{multicols}{2}
    \rolecard{Arbitre}{
        \begin{center}\phantom{a}\end{center}
        \begin{itemize}[label=\faPen]
            \item Valide les calculs des équipes en utilisant la calculatrice.
            \item Chronomètre les défis ( 30 secondes à 1 minute ).
            \item Peut \acc{mettre en pause} la partie à tout moment.
            \item Peut \acc{annuler une action} s'il estime que l'équipe n'est pas correcte. \\ ( triche, antijeu, moqueries...)
            \item Demande l'intervention du professeur en cas de problème.
        \end{itemize}
    }

    \columnbreak

    \rolecard{Arbitre}{
        \begin{center}\phantom{a}\end{center}
        \begin{itemize}[label=\faPen]
            \item Valide les calculs des équipes en utilisant la calculatrice.
            \item Chronomètre les défis ( 30 secondes à 1 minute ).
            \item Peut \acc{mettre en pause} la partie à tout moment.
            \item Peut \acc{annuler une action} s'il estime que l'équipe n'est pas correcte. \\ ( triche, antijeu, moqueries...)
            \item Demande l'intervention du professeur en cas de problème.
        \end{itemize}
    }
\end{multicols}

% Commande pour dessiner une case à cocher vide.
\newcommand{\hereismycheckbox}{%
  \tikz[scale=0.8] \draw[thick] (0,0) rectangle (0.4,0.4);%
}

\begin{center}
    % Le tableau comporte une première colonne pour le libellé puis 6 colonnes (3 pour chaque équipe)
    \begin{tcbtab}{p{5cm} |*{3}{>{\centering\arraybackslash}p{1.5cm}} |*{3}{>{\centering\arraybackslash}p{1.5cm}}}
      \hline
      \multirow{2}{*}{\centering\textbf{Infractions}} & \multicolumn{3}{c|}{\textbf{Équipe \no\repsim[1cm]{}}} & \multicolumn{3}{c|}{\textbf{Équipe \no\repsim[1cm]{}}} \\
      \cline{2-7}
      & \textbf{Niv. 1} & \textbf{Niv. 2} & \textbf{Niv. 3} & \textbf{Niv. 1} & \textbf{Niv. 2} & \textbf{Niv. 3} \\
      \hline
      % Exemple de ligne d'infraction avec 3 cases par équipe
      Triche aux dés & \hereismycheckbox & \hereismycheckbox & \hereismycheckbox & \hereismycheckbox & \hereismycheckbox & \hereismycheckbox \\
      \hline
      % Ajoutez ici d'autres infractions
      Retarder l'activité & \hereismycheckbox & \hereismycheckbox & \hereismycheckbox & \hereismycheckbox & \hereismycheckbox & \hereismycheckbox \\
      \hline
      Bavardages excessifs & \hereismycheckbox & \hereismycheckbox & \hereismycheckbox & \hereismycheckbox & \hereismycheckbox & \hereismycheckbox \\
      \hline
      Non-respect des consignes & \hereismycheckbox & \hereismycheckbox & \hereismycheckbox & \hereismycheckbox & \hereismycheckbox & \hereismycheckbox \\
      \hline
      Dégrade le matériel & \hereismycheckbox & \hereismycheckbox & \hereismycheckbox & \hereismycheckbox & \hereismycheckbox & \hereismycheckbox \\
      \hline
      Ne respecte pas l'arbitre & \hereismycheckbox & \hereismycheckbox & \hereismycheckbox & \hereismycheckbox & \hereismycheckbox & \hereismycheckbox \\
      \hline
    \end{tcbtab}
  \end{center}

  Observations : 

  \begin{crep}[extra lines=5]
    \phantom{a}

  \end{crep}

   \vspace{3cm}