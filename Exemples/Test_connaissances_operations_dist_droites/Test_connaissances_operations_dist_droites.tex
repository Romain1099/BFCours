\documentclass[a4paper,11pt,fleqn]{article}

\usepackage[left=1cm,right=0.5cm,top=0.5cm,bottom=2cm]{geometry}

\usepackage{bfcours}
\usepackage{bfcours-fonts}
\def\rdifficulty{1}
\setrdexo{%left skip=1cm,
display exotitle,
exo header = tcolorbox,
%display tags,
skin = bouyachakka,
lower ={box=crep},
display score,
display level,
save lower,
score=\points,
level=\rdifficulty,
overlay={\node[inner sep=0pt,
anchor=west,rotate=90, yshift=0.3cm]%,xshift=-3em], yshift=0.45cm
at (frame.south west) {\thetags[0]} ;}
]%obligatoire
}
\setrdcrep{seyes, correction=true, correction color=monrose, correction font = \large\bfseries}

\newcommand{\tikzinclude}[1]{%
    \stepcounter{tikzfigcounter}%
    \csname tikzfig#1\endcsname
}
%Fichier généré à l'aide du script "Tikzfig.py"

\newcommand{\tikzfiglOjG}{
\begin{tikzpicture}[baseline]

		\tikzset{
		  point/.style={
			thick,
			draw,
			cross out,
			inner sep=0pt,
			minimum width=5pt,
			minimum height=5pt,
		  },
		}
		\clip (-5,-4) rectangle (5,5);
			\draw[color ={{black}},line width = 0.625,opacity = 0.8] (-1.0625,3.0625)--(-0.9375,2.9375);\draw[color ={{black}},line width = 0.625,opacity = 0.8] (-1.0625,2.9375)--(-0.9375,3.0625);
		\draw [color={black}] (-1,3.5) node[anchor = center,scale=1, rotate = 0] {T};
		\draw[color={black}] (-44.721359099991595,-22.360679549995798)--(48.721359099991595,24.360679549995798);
		\draw [color={black}] (4,2.5) node[anchor = center,scale=1, rotate = 0] {(d)};
	
	\end{tikzpicture}
}

\newcommand{\tikzfigBNjj}{
\begin{tikzpicture}[baseline]

    \tikzset{
      point/.style={
        thick,
        draw,
        cross out,
        inner sep=0pt,
        minimum width=5pt,
        minimum height=5pt,
      },
    }
    \clip (-5,-4) rectangle (5,5);
    	\draw[color ={{black}},line width = 0.625,opacity = 0.8] (-1.0625,3.0625)--(-0.9375,2.9375);\draw[color ={{black}},line width = 0.625,opacity = 0.8] (-1.0625,2.9375)--(-0.9375,3.0625);
	\draw[color={black}] (-44.721359099991595,-22.360679549995798)--(48.721359099991595,24.360679549995798);\draw [color={black}] (0,0.5) node[anchor = center,scale=1, rotate = 0] {(d)};
	\draw[color={black}] (-1,3)--(0.4,0.2)--cycle;
	
	\draw [color={black}] (-1.22,3.45) node[anchor = center,scale=1, rotate = 0] {T};
	\draw [color={black}] (0.62,-0.25) node[anchor = center,scale=1, rotate = 0] {H};
	\draw [color={black}] (0.15,1.82) node[anchor = center,scale=1, rotate = -63.43495] {3,1 cm};
	\draw[color={black}] (0.22,0.56)--(0.58,0.74)--(0.76,0.38);

\end{tikzpicture}
}



\hypersetup{
    pdfauthor={R.Deschamps},
    pdfsubject={},
    pdfkeywords={},
    pdfproducer={LuaLaTeX},
    pdfcreator={Boum Factory}
}
% Activer ou désactiver l'affichage des boîtes pour les points
%\displayitempointsfalse % Ne pas afficher les boîtes
\displayitempointstrue % Afficher les boîtes

\begin{document}

\setcounter{pagecounter}{0}
\setcounter{ExoMA}{0}
\setcounter{prof}{0}

\def\points{\phantom{AAA}}
\def\difficulty{\phantom{AAA}}
\chapitre[
    $\mathbf{6^{\text{ème}}}$% : $\mathbf{6^{\text{ème}}}$,$\mathbf{5^{\text{ème}}}$,$\mathbf{4^{\text{ème}}}$,$\mathbf{3^{\text{ème}}}$,$\mathbf{2^{\text{nde}}}$,$\mathbf{1^{\text{ère}}}$,$\mathbf{T^{\text{Le}}}$,
    ]{
    Opérations - Distance d'un point à une droite% : ,Equations
    }{
    Collège% : Collège,Lycée
    }{
    %Amadis Jamyn% : Amadis Jamyn,Eugène Belgrand
    }{
    \tableauPresenteEvalSixieme{}{\getsavedtotalpoints}
    }{
    Devoir :
    }
%
\setrdcrep{seyes, correction=false, correction color=monrose, correction font = \large\bfseries}
%    \tableaucompetence{
%        \competence{
%		% Competencea , 
%	}
%        \competence{
%		% Competenceb , 
%	}
%	\competence{
%		%Competencec ,
%	}
%    }
%
\vspace{-0.2cm}
\begin{minipage}[t]{0.5\textwidth}%
    \printcompindex
\end{minipage}
\hfill
\begin{minipage}[t]{0.4\textwidth}
    \begin{minipage}[t]{0.75\textwidth}
        \phantom{a}\\
        Date : \repsim[3.5cm]{}\hfill \vrule \hfill \overlaychrono{10}
    \end{minipage}\\
    \begin{minipage}[t]{\textwidth}
        \textbf{Modalités} :
        \begin{itemize}[label=$\bullet$]
            \item Calculatrice interdite% : \item Calculatrice interdite, \item Calculatrice autorisée,\item 
            \item Pas de prêt de matériel
            \item Les figures doivent être réalisées au crayon à papier bien taillé
        \end{itemize}
    \end{minipage}
\end{minipage}

\def\points{2}
\def\rdifficulty{1}
\begin{EXOEVAL}{Opérations Posées - Vocabulaire}{6C20,\ 6C13-2}

    Poser et effectuer les calculs suivants.

\begin{multicols}{2}
	\begin{enumerate}[itemsep=1em]
		\item \itempoint{1}$131+69{,}23 = $\repsim[2.5cm]{\num{\fpeval{131+69.23}}}\\
		\boite{Calcul :}{\begin{crep}[extra lines = 4]
			\opadd[lineheight=\baselineskip,columnwidth=2ex,decimalsepsymbol={,},voperator=bottom,voperation=top]{131}{69.23}
		\end{crep}}

		\columnbreak

		\item \itempoint{1}$505-499{,}7= $\repsim[2.5cm]{\num{\fpeval{505-499.7}}}\\
		\boite{Calcul :}{\begin{crep}[extra lines = 5]
			\opsub[lineheight=\baselineskip,columnwidth=2ex,carrysub,lastcarry,decimalsepsymbol={,},voperator=bottom,voperation=top]{505}{499.7}
		\end{crep}}
	\end{enumerate}
\end{multicols}
\vspace{-0.4cm}\begin{multicols}{2}
	\begin{enumerate}[itemsep=1em,start=3]
		\item \itempoint{1.5}$97{,}4\times9{,}8= $\repsim[2.5cm]{\num{\fpeval{97.4*9.8}}}\\
		\boite{Calcul :}{\begin{crep}[extra lines = 6]
			\opmul[lineheight=\baselineskip,columnwidth=2ex,displayshiftintermediary=all,decimalsepsymbol={,},voperator=bottom,voperation=top]{97.4}{9.8}\hspace*{30mm}\opmul[lineheight=\baselineskip,columnwidth=2ex,displayshiftintermediary=all,decimalsepsymbol={,},voperator=bottom,voperation=top]{9.8}{97.4}
		\end{crep}}

		\columnbreak

		\item \textbf{Traduire} chaque calcul par une \textbf{phrase} utilisant le \acc{vocabulaire} du cours.

		\begin{enumerate}[itemsep=1em]
			\item \itempoint{1}$7 \times 10$\tcfillcrep{}\phantom{AAAAAA} \\ \tcfillcrep{le produit de 7 par 10}
			\item \itempoint{1}$14 - 2$\tcfillcrep{}\phantom{AAAAAA}\\ \tcfillcrep{la différence de 14 et de 2}
			\item \itempoint{1}$7+6$\tcfillcrep{}\phantom{AAAAAA}\\ \tcfillcrep{la somme de 7 et de 6}
		\end{enumerate}
	\end{enumerate}
\end{multicols}


\exocorrection

\begin{multicols}{2}
	\begin{enumerate}[itemsep=1em]
		\item $131+69{,}23 = $\repsim[2.5cm]{\num{\fpeval{131+69.23}}}\\
		\boite{Calcul :}{\begin{crep}
			\opadd[lineheight=\baselineskip,columnwidth=2ex,decimalsepsymbol={,},voperator=bottom,voperation=top]{131}{69.23}
		\end{crep}}

		\columnbreak

		\item $505-499{,}7= $\repsim[2.5cm]{\num{\fpeval{505-499.7}}}\\
		\boite{Calcul :}{\begin{crep}
			\opsub[lineheight=\baselineskip,columnwidth=2ex,carrysub,lastcarry,decimalsepsymbol={,},voperator=bottom,voperation=top]{505}{499.7}
		\end{crep}}
	\end{enumerate}
\end{multicols}
\begin{multicols}{2}
	\begin{enumerate}[itemsep=1em,start=3]
		\item $97{,}4\times9{,}8= $\repsim[2.5cm]{\num{\fpeval{97.4*9.8}}}\\
		\boite{Calcul :}{\begin{crep}
			\opmul[lineheight=\baselineskip,columnwidth=2ex,displayshiftintermediary=all,decimalsepsymbol={,},voperator=bottom,voperation=top]{97.4}{9.8}\hspace*{30mm}\opmul[lineheight=\baselineskip,columnwidth=2ex,displayshiftintermediary=all,decimalsepsymbol={,},voperator=bottom,voperation=top]{9.8}{97.4}
		\end{crep}}

		\columnbreak

		\item \textbf{Traduire} chaque calcul par une \textbf{phrase} en français.\\
			Il n'est \textbf{pas demandé d'effectuer le calcul}.

		\begin{enumerate}[itemsep=1em]
			\item $7 \times 10$\tcfillcrep{}\phantom{AAAAA} \\ \repsim[6cm]{le produit de 7 par 10}
			\item $14 - 2$\tcfillcrep{}\phantom{AAAAA}\\ \repsim[6cm]{la différence de 14 et de 2}
			\item $7+6$\tcfillcrep{}\phantom{AAAAA}\\ \repsim[6cm]{la somme de 7 et de 6}
		\end{enumerate}
	\end{enumerate}
\end{multicols}

\end{EXOEVAL}

\newpage

\def\points{1}
\def\rdifficulty{2}
\begin{EXOEVAL}{Distance d'un point à une droite}{6G53}

    

	\begin{enumerate}[itemsep=1em]
		\item \itempoint{2}\textbf{Mesurer} la distance entre le \textbf{point} $T$ et la \textbf{droite} ($d$).\\
 Distance $ = \repsim[4.5cm]{TH = 3{,}1 cm}$\\

 		\item \itempoint{1.5}\textbf{Laisser apparents} les traits de construction et \textbf{coder la figure}.\\
 	\end{enumerate}
 \begin{center}
	\tikzinclude{lOjG}
 \end{center}

\exocorrection

\begin{enumerate}[itemsep=1em]
	\item \textbf{Mesurer} la distance entre le \textbf{point} $T$ et la \textbf{droite} ($d$).\\
Distance $ = \repsim[4.5cm]{TH = 3{,}1 cm}$\\

	 \item \textbf{Laisser apparents} les traits de construction et \textbf{coder la figure}.\\
 \end{enumerate}
 \tikzinclude{BNjj}\\\\Pour mesurer la distance entre le point $T$ et la droite ($d$) :\\
      - on utilise l'équerre pour tracer la perpendiculaire à la droite ($d$)) qui passe par le point $T$\\
      - si on nomme $H$ le pied de la perpendiculaire, alors la distance entre le point $T$ et la droite ($d$) est la longueur $TH = 3{,}1 cm$


\end{EXOEVAL}
% : On peut utiliser le mode maths de plusieurs manières : 
\begin{tcolorbox}[blank]
\begin{tcbenumerate}[2]
    \tcbitem Inline via \textcolor{red}{\$ contenu maths \$} : $\left(\text{\Large{E}}\right)~~5x + 3 = 2^\frac{3}{x}$
    \tcbitem Inline via le mode \acc{display} \textcolor{red}{$\backslash$( contenu maths $\backslash$)} : \(\left(\text{\Large{E}}\right)~~5x + 3 = 2^\frac{3}{x}\)
    \tcbitem En valeur via le mode \acc{centré} \textcolor{red}{$\backslash$[ contenu maths $\backslash$]} : \[\left(\text{\Large{E}}\right)~~5x + 3 = 2^\frac{3}{x}\]
    \tcbitem En mode \acc{align} ( énuméré ) ou \acc{align*} ( non énuméré ) :  
    
    \showenv{align}[][contenu maths]
    \begin{align}
        \left(\text{\Large{E}}\right)~~5x + 3 &= 2^\frac{3}{x}\\
        &= e^{\frac{3}{x}\ln(2)}
    \end{align}
\end{tcbenumerate}

Dans le mode mathématiques, si l'on veut écrire du texte, il faut l'appeler \acc{dans la commande text} : 

\showcmd{text}[\{Texte à écrire\}].


Le mode mathématique peut donc aussi être utilisé pour formatter du texte : 

$6^{\text{ème}}$
\end{tcolorbox},

\newpage
\setcounter{pagecounter}{0}
\setcounter{ExoMA}{0}
\setcounter{prof}{1}

\chapitre[
    $\mathbf{6^{\text{ème}}}$%
    ]{
    Opérations - Distance d'un point à une droite%
    }{
    Collège%
    }{
    Amadis Jamyn%
    }{
    % supplement
    }{
    Solution
    }
\setrdcrep{seyes, correction=true, correction color=monrose, correction font = \large\bfseries}
\tcbset{
    rdexo/default/.cd,correction style/.style={
        before upper=%
        \textbf{\thelabel~
        \thecorrectionnum~:~}
    }
}

\rdexocorrection{0}



\end{document}