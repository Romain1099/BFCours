\def\points{2}
\def\rdifficulty{1}
\begin{EXOEVAL}{Opérations Posées - Vocabulaire}{6C20,\ 6C13-2}

    Poser et effectuer les calculs suivants.

\begin{multicols}{2}
	\begin{enumerate}[itemsep=1em]
		\item \itempoint{1}$131+69{,}23 = $\repsim[2.5cm]{\num{\fpeval{131+69.23}}}\\
		\boite{Calcul :}{\begin{crep}[extra lines = 4]
			\opadd[lineheight=\baselineskip,columnwidth=2ex,decimalsepsymbol={,},voperator=bottom,voperation=top]{131}{69.23}
		\end{crep}}

		\columnbreak

		\item \itempoint{1}$505-499{,}7= $\repsim[2.5cm]{\num{\fpeval{505-499.7}}}\\
		\boite{Calcul :}{\begin{crep}[extra lines = 5]
			\opsub[lineheight=\baselineskip,columnwidth=2ex,carrysub,lastcarry,decimalsepsymbol={,},voperator=bottom,voperation=top]{505}{499.7}
		\end{crep}}
	\end{enumerate}
\end{multicols}
\vspace{-0.4cm}\begin{multicols}{2}
	\begin{enumerate}[itemsep=1em,start=3]
		\item \itempoint{1.5}$97{,}4\times9{,}8= $\repsim[2.5cm]{\num{\fpeval{97.4*9.8}}}\\
		\boite{Calcul :}{\begin{crep}[extra lines = 6]
			\opmul[lineheight=\baselineskip,columnwidth=2ex,displayshiftintermediary=all,decimalsepsymbol={,},voperator=bottom,voperation=top]{97.4}{9.8}\hspace*{30mm}\opmul[lineheight=\baselineskip,columnwidth=2ex,displayshiftintermediary=all,decimalsepsymbol={,},voperator=bottom,voperation=top]{9.8}{97.4}
		\end{crep}}

		\columnbreak

		\item \textbf{Traduire} chaque calcul par une \textbf{phrase} utilisant le \acc{vocabulaire} du cours.

		\begin{enumerate}[itemsep=1em]
			\item \itempoint{1}$7 \times 10$\tcfillcrep{}\phantom{AAAAAA} \\ \tcfillcrep{le produit de 7 par 10}
			\item \itempoint{1}$14 - 2$\tcfillcrep{}\phantom{AAAAAA}\\ \tcfillcrep{la différence de 14 et de 2}
			\item \itempoint{1}$7+6$\tcfillcrep{}\phantom{AAAAAA}\\ \tcfillcrep{la somme de 7 et de 6}
		\end{enumerate}
	\end{enumerate}
\end{multicols}


\exocorrection

\begin{multicols}{2}
	\begin{enumerate}[itemsep=1em]
		\item $131+69{,}23 = $\repsim[2.5cm]{\num{\fpeval{131+69.23}}}\\
		\boite{Calcul :}{\begin{crep}
			\opadd[lineheight=\baselineskip,columnwidth=2ex,decimalsepsymbol={,},voperator=bottom,voperation=top]{131}{69.23}
		\end{crep}}

		\columnbreak

		\item $505-499{,}7= $\repsim[2.5cm]{\num{\fpeval{505-499.7}}}\\
		\boite{Calcul :}{\begin{crep}
			\opsub[lineheight=\baselineskip,columnwidth=2ex,carrysub,lastcarry,decimalsepsymbol={,},voperator=bottom,voperation=top]{505}{499.7}
		\end{crep}}
	\end{enumerate}
\end{multicols}
\begin{multicols}{2}
	\begin{enumerate}[itemsep=1em,start=3]
		\item $97{,}4\times9{,}8= $\repsim[2.5cm]{\num{\fpeval{97.4*9.8}}}\\
		\boite{Calcul :}{\begin{crep}
			\opmul[lineheight=\baselineskip,columnwidth=2ex,displayshiftintermediary=all,decimalsepsymbol={,},voperator=bottom,voperation=top]{97.4}{9.8}\hspace*{30mm}\opmul[lineheight=\baselineskip,columnwidth=2ex,displayshiftintermediary=all,decimalsepsymbol={,},voperator=bottom,voperation=top]{9.8}{97.4}
		\end{crep}}

		\columnbreak

		\item \textbf{Traduire} chaque calcul par une \textbf{phrase} en français.\\
			Il n'est \textbf{pas demandé d'effectuer le calcul}.

		\begin{enumerate}[itemsep=1em]
			\item $7 \times 10$\tcfillcrep{}\phantom{AAAAA} \\ \repsim[6cm]{le produit de 7 par 10}
			\item $14 - 2$\tcfillcrep{}\phantom{AAAAA}\\ \repsim[6cm]{la différence de 14 et de 2}
			\item $7+6$\tcfillcrep{}\phantom{AAAAA}\\ \repsim[6cm]{la somme de 7 et de 6}
		\end{enumerate}
	\end{enumerate}
\end{multicols}

\end{EXOEVAL}

\newpage

\def\points{1}
\def\rdifficulty{2}
\begin{EXOEVAL}{Distance d'un point à une droite}{6G53}

    

	\begin{enumerate}[itemsep=1em]
		\item \itempoint{2}\textbf{Mesurer} la distance entre le \textbf{point} $T$ et la \textbf{droite} ($d$).\\
 Distance $ = \repsim[4.5cm]{TH = 3{,}1 cm}$\\

 		\item \itempoint{1.5}\textbf{Laisser apparents} les traits de construction et \textbf{coder la figure}.\\
 	\end{enumerate}
 \begin{center}
	\tikzinclude{lOjG}
 \end{center}

\exocorrection

\begin{enumerate}[itemsep=1em]
	\item \textbf{Mesurer} la distance entre le \textbf{point} $T$ et la \textbf{droite} ($d$).\\
Distance $ = \repsim[4.5cm]{TH = 3{,}1 cm}$\\

	 \item \textbf{Laisser apparents} les traits de construction et \textbf{coder la figure}.\\
 \end{enumerate}
 \tikzinclude{BNjj}\\\\Pour mesurer la distance entre le point $T$ et la droite ($d$) :\\
      - on utilise l'équerre pour tracer la perpendiculaire à la droite ($d$)) qui passe par le point $T$\\
      - si on nomme $H$ le pied de la perpendiculaire, alors la distance entre le point $T$ et la droite ($d$) est la longueur $TH = 3{,}1 cm$


\end{EXOEVAL}
